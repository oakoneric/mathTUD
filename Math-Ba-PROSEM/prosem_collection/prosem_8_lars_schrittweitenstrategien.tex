\newcommand{\nat}{\mathbb{N}}				% natürliche Zahlen
\newcommand{\integer}{\mathbb{Z}}			% ganze Zahlen
\newcommand{\ratio}{\mathbb{Q}}
\newcommand{\real}{\mathbb{R}}				% reelle Zahlen
\newcommand{\complex}{\mathbb{C}}			% komplexe Zahlen
\newcommand{\korper}{\mathbb{K}}            % Körper IR oder IC


\chapter{Schrittweitenstrategien}

\vortragender{Lars Ortscheidt}

Das bisher kennengelernte Abstiegsverfahren hat in Bezug auf die Abstiegsrichtung $d$ und die Schrittweite $t$ große Freiheitsgrade. Da die Schrittweite $t_{\min}$ mit $f(x+t_{\min}d) = \min_{t\geq 0} f(x+td)$ im Allgemeinen nicht in endlich vielen Schritten berechnet werden kann, werden nun zwei Schrittweitenregeln vorgestellt, welche ''realisierbar`` sind, d.h. sie brauchen nur endlich viele Schritte zur Berechnung.


\section{Armijo-Regel}

\begin{definition}
	Seien $f: \real^n \to \real$ stetig differenzierbar und $\sigma, \beta \in (0,1)$ fest für das ganze Abstiegsverfahren. Zu $x, d \in \real^n$ mit $\nabla f(x)^{\top} d<0$ bestimme $t \defeq \max_{\ell \in \nat_0} \beta^\ell$, so dass gilt 
	\begin{align}
		f(x+td) \leq f(x) + \sigma t \nabla f(x)^{\top} d.  \label{armijo}
	\end{align}	 
	Dieses Verfahren heißt \begriff{Armijo-Regel}. \\
	Setzt man $\phi(t) := f(x+td)$ lautet \labelcref{armijo} 
	\begin{align*}
		\phi(t) \leq \phi (0) + \sigma t \phi' (0),
	\end{align*}
	die sogenannte Armijo-Goldstein-Bedingung.
\end{definition}

\begin{bemerkung}
	\begin{itemize}	
		\item Zur Bestimmung von $t$ wird \labelcref{armijo} also für $t= \beta^\ell$, $\ell=0,1,2,...$ überprüft und bei der ersten Gültigkeit abgebrochen.
		\item $T(x,d)$ besitzt bei der Armijo-Regel höchstens ein Element.
	\end{itemize}
\end{bemerkung}
\begin{satz}
	Seien $f: \real^n \to \real$ stetig differenzierbar mit $\sigma , \beta \in (0,1)$, fest. Dann existiert für alle $x,d \in \real^n$ mit $\nabla f(x)^{\top} d < 0$ ein endliches $\ell \in \nat$ mit
	\begin{align*}
		f(x+ \beta^\ell d) \leq f(x)+ \sigma \beta^\ell \nabla f(x)^{\top} d, 
	\end{align*}
	d.h. die Armijo-Regel ist wohldefinitionniert.
\end{satz}

\begin{proof}
	Angenommen für alle $\ell \in \nat$ gilt
	\begin{align*}
		f(x+ \beta^\ell d) > f(x)+ \sigma \beta^\ell \nabla f(x)^{\top} d
	\end{align*}
	und somit auch
	\begin{align*}
		\frac {f(x+ \beta^\ell d) - f(x)}{\beta^\ell} > \sigma \nabla f(x)^{\top} d.
	\end{align*}
	Dann folgt mit $\ell \to \infty$ wegen der Differenzierbarkeit von $f$
	\begin{align*}
		\nabla f(x)^{\top} d > \sigma \nabla f(x)^{\top} d
	\end{align*}
	und weil $\sigma \in (0,1)$ ergibt sich
	\begin{align*}
		\nabla f(x)^{\top} d \geq 0,
	\end{align*}
	was im Widerspruch zur Voraussetzung des Satzes steht.
\end{proof}


\section{Wolfe-Powell-Schrittweitenstrategie}

\begin{definition}
	Seien $f: \real^n \to \real$ stetig differenzierbar und $\sigma \in (0, \frac{1}{2}), \, \rho \in [\sigma, 1)$ fest vorgegeben. Für $x,d \in \real^n$ mit $\nabla f(x)^{\top} d <0$ bestimme man ein $t >0$ mit 
	\begin{align}
		f(x+td) \leq f(x) + \sigma t \nabla f(x)^{\top} d \label{w_p_1}
	\end{align}
	und
	\begin{align}
		\nabla f(x+td)^{\top} d \geq \rho \nabla f(x)^{\top} d. \label{w_p_2}
	\end{align}
	\labelcref{w_p_1} und \labelcref{w_p_2} heißen dann die \begriff{Wolfe-Powell-Bedingungen}. \\ Setzt man $\phi(t) := f(x+td)$, so lauten \labelcref{w_p_1} und \labelcref{w_p_2}
	\begin{align*}
		\phi (t) \leq \phi (0)+\sigma t \phi' (0)
	\end{align*}
	und
	\begin{align*}
		\phi'(t) \geq \rho \phi' (0).
	\end{align*}
\end{definition}

\begin{satz}
	Seien $f:\real^n \to \real$ stetig differenzierbar und $\sigma \in (0, \frac{1}{2})$, $\rho \in [\sigma, 1)$, $x^0 \in \real^n$ fest. Zu $x \in \mathcal{L}(x^0) := \{ z \in \real^n | f(z) \leq f(x^0) \}$ und $d \in \real^n$ mit $\nabla f(x)^{\top} d <0$ sei 
	\begin{align*}
		T_{WP} (x,d) := \{ t>0 | \text{\labelcref{w_p_1} und \labelcref{w_p_2} gelten} \}
	\end{align*}
	die Menge der Wolfe-Powell-Schrittweiten in $x$ in Richtung $d$. Dann gelten:
	\begin{enumerate}[label=(\alph*)]
	\item Ist f nach unten beschränkt, so ist $T_{WP}(x,d) \neq \emptyset$, d.h. die Wolfe-Powell-Strategie ist wohldefinitionniert.
	\item Ist außerdem der Gradient $\nabla f$ auf der Levelmenge $\mathcal{L}(x^0)$ Lipschitz-stetig, so existiert eine Konstante $\theta >0$ (unabhängig von x und d) mit
	\begin{align*}
		f(x+td) \leq f(x) - \theta \left( \frac{\nabla f(x)^{\top} d}{\| d \|}     \right)^2
	\end{align*}	 
	für alle $t \in T_{WP}(x,d)$, d.h. die Wolfe-Powell-Schrittweitenstrategie ist effizient.
	\end{enumerate}
\end{satz}

\begin{proof}
	Zu (a): Setze 
	\begin{align*}
		\phi(t) &:= f(x+td), \\
		\psi (t) &:= f(x+td)-f(x)- \sigma t \nabla f(x)^{\top} d= \phi(t) - \phi(0) - \sigma t \phi'(0) 
	\end{align*}
	und somit
	\begin{align*}
		\phi'(t) = \nabla f(x+td)^{\top} d, \\
		\psi'(t) = \phi'(t) - \sigma \phi'(0).
	\end{align*}
	So lauten \labelcref{w_p_1} und \labelcref{w_p_2}
	\begin{align}
		\psi (t) &\leq 0 \label{w_p_3} \\
		\phi'(t) &\geq \rho \phi'(0) \label{w_p_4}.
	\end{align}
	Offenbar gilt $\psi(0)=0, \lim_{t \to \infty} \psi(t) = \infty$ (da $f$ beschränkt), $\psi'(0) <0$, damit $\psi'(t) <0$ für $t \in [0, t_0), t_0 >0$. Daher existiert $t^\ast>0$ minimal mit $\psi(t^\ast)=0, \psi(t)<0$ für $t \in (0,t^\ast)$, somit erfüllt $t^\ast$ \labelcref{w_p_3}. Da $\psi'(t^\ast) \geq 0$ folgt $\phi'(t^\ast) \geq \sigma \phi'(0)$, womit wegen $\rho \geq \sigma >0$ und $\phi'(0)<0$ \labelcref{w_p_4} für $t^\ast$ folgt, also $t^\ast \in T_{WP}(x,d).$ 
	\\ 
	Zu (b): Sei $t \in T_{WP}(x,d)$ gegeben. Dann ist $f(x+td) \leq f(x)$ und somit insbesondere $x+td \in \mathcal{L}(x^0)$. Aus der Wolfe-Powell-Regel folgt zunächst
	\begin{align*}
		\rho \nabla f(x)^{\top} d - \nabla f(x)^{\top} d &\leq \nabla f(x+td)^{\top} d - \nabla f(x)^{\top} d \\
		\Longleftrightarrow \qquad (\rho -1) \nabla f(x)^{\top}d &\leq ( \nabla f(x+td)- \nabla f(x) )^{\top} d.
	\end{align*}
	Mit der Cauchy-Schwarzschen Ungleichung und vorausgesetzten Lipschitz-Stetigkeit von $\nabla f(x)$ auf $\mathcal{L}(x^0)$ folgt mit einer geeigneten Konstanten $\ell > 0$:
	\begin{align*}
		(\rho -1) \nabla f(x)^{\top}d \leq \| \nabla f(x+td)- \nabla f(x) \| \; \|d\| \leq Lt \|d\|^2
	\end{align*}
	Hieraus folgt
	\begin{align*}
		t \geq \frac{(\rho -1) \nabla f(x)^{\top} d}{\ell \|d\| ^2}
	\end{align*}
	und daher 
	\begin{align*}
		f(x+td) &\mathop{\leq}^{\labelcref{w_p_1}} f(x) + \sigma  t \nabla f(x)^{\top} d \\
		&\leq f(x)+ \sigma \frac{(\rho -1) \nabla f(x)^{\top} d}{\ell \|d\| ^2} \nabla f(x)^{\top} d \\
		&\leq f(x) - \theta \left(\frac{\nabla f(x)^{\top} d}{\|d\|}\right)^2
	\end{align*}
	mit 
	\begin{align*}
		\theta := \frac{(1-\rho) \sigma}{\ell}
	\end{align*}
	Damit ist die Behauptung bewiesen.
\end{proof}

Abschließend zwei hinreichende Bedingungen dafür, dass der Gradient $\nabla f$ auf der Levelmenge $\mathcal{L}(x^0)$ Lipschitz-stetig ist.
\begin{lemma}
	Seien $f: \real^n \to \real$ zweimal stetig differenzierbar und $x^0 \in \real^n$. Ist eine der folgenden Bedingungen erfüllt:
	\begin{enumerate}
	\item[(a)] $\| \nabla^2 f(x) \|$ ist beschränkt auf einer konvexen Obermenge $X$ der Levelmenge $\mathcal{L}(x^0)$,
	\item[(b)] die Levelmenge $\mathcal{L}(x^0)$ ist kompakt,
	\end{enumerate}
	so ist der Gradient $\nabla f$ Lipschitz-stetig auf $\mathcal{L}(x^0)$. 
\end{lemma}

\begin{proof}
	Sei Bedingung (b) erfüllt. Definiere 
	\begin{align*}
		[\mathcal{L}(x^0)]:= \{ ax+(1-a)y \mid a \in [0,1], x,y \in \mathcal{L}(x^0) \},
	\end{align*} offenbar ist diese Menge eine konvexe Obermenge von $\mathcal{L}(x^0)$. Betrachte die Funktion  
	\begin{align*}
	f \colon \left\lbrace%	
	\begin{array}{ccl}%
		[0,1] \times \mathcal{L}(x^0) \times \mathcal{L}(x^0) & \to & [\mathcal{L}(x^0)] \\%
		(a,x,y) & \mapsto & ax+(1-a)y%
	\end{array}%
\right.
	\end{align*}
	Offenbar ist f stetig und surjektiv mit kompaktem Definitionsbereich. Da das Bild von kompakten Mengen unter stetigen Funktionen wieder kompakt ist, folgt $[\mathcal{L}(x^0)]$ kompakt, d.h. es es existiert eine konvexe und kompakte Obermenge $X \subseteq \real^n$ mit $\mathcal{L}(x^0) \subseteq X$.
	Aus Stetigkeitsgründen existiert dann eine Konstante $L$ mit 
	\begin{align}
		\| \nabla^2 f(x)\| \leq L \qquad \text{ für alle } x \in X, \label{wa}
	\end{align}
	d.h. die Bedingung (a) ist erfüllt. \\
	Sei nun die Bedingung (a) erfüllt. Dann existiert eine Zahl $L>0$ mit \labelcref{wa} . Es gilt
	\begin{align*}
		 \int_{0}^{1} \nabla^2 f(y+ \tau(x-y))(x-y) \, d \tau = \left[\nabla f(y+ \tau(x-y))\right]_{\tau =0}^1 = \nabla f(x) - \nabla f(y)
	\end{align*}
	für alle $x,y\in X$. Wegen $y+\tau (x-y) \in X$ ist daher
	\begin{align*}
		\| \nabla f(x) - \nabla f(y) \| &\leq \int_{0}^{1} \| \nabla^2 f(y+ \tau(x-y)) \|  \, d \tau \, \|x-y\| \\
		&\leq \int_0^1 L \, d\tau \, \| x-y \| \\
		&= L \| x-y \|
	\end{align*}
	für alle $x,y \in X$, was zu zeigen war. 
\end{proof}

\undef\nat
\undef\integer
\undef\ratio
\undef\real
\undef\complex
\undef\korper