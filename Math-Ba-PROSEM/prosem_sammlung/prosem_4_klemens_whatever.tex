%~ \documentclass[a4paper,titlepage,reqno]{amsart}
\documentclass[a4paper,reqno]{amsart}
\usepackage[foot]{amsaddr}
\usepackage[T1]{fontenc}
\usepackage[utf8]{inputenc}
\usepackage{lmodern}
\usepackage{textcomp}
\usepackage[ngerman]{babel}
\usepackage{blindtext}
\usepackage{amsmath,amssymb,mathtools,paralist,enumitem,color,wasysym}
\usepackage[mathscr]{eucal}


\newtheorem{theorem}{Theorem}[section]
\newtheorem{lemma}[theorem]{Lemma}
\newtheorem{satz}[theorem]{Satz}
\newtheorem{bemerkung}[theorem]{Bemerkung}
\newtheorem{beispiel}[theorem]{Beispiel}
%~ \newtheorem{beweis}[proof]{Beweis}

\theoremstyle{definition}
\newtheorem{definition}[theorem]{Definition}
\newtheorem{example}[theorem]{Example}
\newtheorem{xca}[theorem]{Exercise}

\theoremstyle{remark}
\newtheorem{remark}[theorem]{Bemerkung}

%~ \newcommand{\norm}[1]{\setcounter{section}{3}\setcounter{theorem}{}}
\newcommand\norm[1]{\left\lVert#1\right\rVert}
\newcommand{\dd}{\textnormal{d}}
\newcommand\red[1]{\textcolor{red}{#1}}

\title[Fortsetzung zu konvexen Funktionen]{Proseminar:\\4. Fortsetzung zu konvexen Funktionen}

%~ \author{Klemens Fritzsche}
%~ \address{}

\begin{document}
\maketitle

%~ \counterwithin{theorem}{section}
% neu: satz 3.8 + lemma 3.9

% im Buch ab S. 18 Lemma 3.9 bis S. 21, einschließlich Aufgaben 3.11 und 3.12
%~ \section{Fortsetzung zu konvexen Funktionen}%

\section*{Wiederholung}
\setcounter{section}{3}\setcounter{theorem}{1}
%~ \setcounter{theorem}{2}
\begin{definition}\label{def3.2}%[Def. 3.2]
    Sei $X\subseteq\mathbb{R}^n$ eine konvexe Menge. Eine Funktion
    $f\colon X\to\mathbb{R}$ heißt
    \begin{compactenum}[(a)]
        \item \emph{konvex} (auf $X$), wenn für alle $x,y\in X$ und alle
            $\lambda\in(0,1)$ gilt:
            \begin{align*}
                f(\lambda x+(1-\lambda)y)\leq\lambda f(x) + (1-\lambda)f(y);
            \end{align*}
        \item \emph{strikt konvex} (auf $X$), wenn für alle $x,y\in X$ mit
            $x\neq y$ und alle $\lambda\in(0,1)$ gilt:
            \begin{align*}
                f(\lambda x+(1-\lambda)y)<\lambda f(x) + (1-\lambda)f(y);
            \end{align*}
        \item \emph{gleichmäßig konvex} (auf $X$), wenn es ein $\mu>0$
            gibt mit
            \begin{align*}
                f(\lambda x+(1-\lambda)y) + \mu\lambda(1-\lambda)\norm{x-y}^2\leq\lambda f(x) + (1-\lambda)f(y).
            \end{align*}
            für alle $x,y\in X$ und alle $\lambda\in(0,1)$.
            (Man bezeichnet $f$ dann auch als \emph{gleichmäßig konvex mit
            Modulus $\mu$}.)
        \end{compactenum}
\end{definition}

\setcounter{theorem}{4}
\begin{satz}\label{satz3.5}%[Satz 3.5]
    Seien $X\subseteq\mathbb{R}^n$ eine offene und konvexe Menge und
    $f\colon X\to\mathbb{R}$ stetig differenzierbar. Dann gelten:
    \begin{compactenum}[(a)]
        \item $f$ konvex (auf $X$) $\Longleftrightarrow$ $\forall x,y\in X: f(x)-f(y)\geq \nabla f(y)^\top(x-y)$.
        \item $f$ strikt konvex (auf $X$) $\Longleftrightarrow$ $\forall x,y\in X,x\neq y: f(x)-f(y)>\nabla f(y)^\top(x-y)$.
        \item $f$ gleichmäßig konvex (auf $X$) $\Longleftrightarrow$ $\exists \mu>0:f(x)-f(y)\geq \nabla f(y)^\top(x-y) + \mu\norm{x-y}^2$ $\forall x,y\in X$
    \end{compactenum}
\end{satz}

\begin{definition}\label{def3.6}%[Def. 3.6]
    Sei $X\subseteq\mathbb{R}^n$ eine gegebene Menge. Eine Funktion
    $F\colon X\to\mathbb{R}^n$ heißt
    \begin{compactenum}[(a)]
        \item \emph{monoton} (auf $X$), wenn
            \begin{align*}
                (x-y)^\top(F(x)-F(y))\geq 0
            \end{align*}
            für alle $x,y\in X$ gilt;
        \item \emph{strikt monoton} (auf $X$), wenn
            \begin{align*}
                (x-y)^\top(F(x)-F(y))> 0
            \end{align*}
            für alle $x,y\in X$ mit $x\neq y$ gilt;
        \item \emph{gleichmäßig monoton} (auf $X$), wenn es ein $\mu>0$ gibt
            mit
            \begin{align*}
                (x-y)^\top(F(x)-F(y))\geq \mu\norm{x-y}^2
            \end{align*}
            für alle $x,y\in X$. (Man bezeichnet $F$ dann auch als
            \emph{gleichmäßig monoton mit Modulus $\mu$}.)
    \end{compactenum}
\end{definition}

\begin{satz}\label{3.7}%[Satz 3.7]
    Sei $X\subseteq\mathbb{R}^n$ eine offene und konvexe Menge und
    $f\colon X\to\mathbb{R}$ stetig differenzierbar. Dann gelten:
    \begin{compactenum}[(a)]
        \item $f$ konvex $\Longleftrightarrow$ $\nabla f$ monoton
        \item $f$ strikt konvex $\Longleftrightarrow$ $\nabla f$ strikt monoton
        \item $f$ gleichmäßig konvex $\Longleftrightarrow$ $\nabla f$
        gleichmäßig monoton.
    \end{compactenum}
\end{satz}

\def\thesection{\Alph{section}}
\setcounter{section}{1}\setcounter{theorem}{2}
\begin{satz}[Mittelwertsatz in Integralform]\label{MWS_Intform}%, Satz A.3]
    Sei $F\colon\mathbb{R}^n\to\mathbb{R}^m$ stetig differenzierbar,
    sowie $x,y\in\mathbb{R}^n$ gegeben. Dann gilt
    \begin{align*}
        F(x) = F(y) + \int_0^1 F'(y+\tau(x-y))(x-y)\dd\tau.
    \end{align*}
\end{satz}

Hier nutzen wir $F(x)=\nabla f(x)$, d.h. es gilt
    \begin{align}\label{MWS_intform}
        \nabla f(x) = \nabla f(y) + \int_0^1 \nabla^2 f(y+\tau(x-y))(x-y)\dd\tau.
    \end{align}

\def\thesection{\arabic{section}}
\section*{Fortsetzung}

Wir zeigen zuerst ein Resultat über zweimal stetig differenzierbare
(strikt, gleichmäßig) konvexe Funktionen unter Zuhilfenahme von Satz
\ref{3.7}:
\setcounter{section}{3}\setcounter{theorem}{7}
\begin{satz}\label{satz3.8}%[Satz 3.8]
    Sei $X\subseteq\mathbb{R}^n$ eine offene und konvexe Menge und
    $f\colon X\to\mathbb{R}^n$ zweimal stetig differenzierbar. Dann gilt
    \begin{compactenum}[(a)]
        \item $f$ konvex (auf $X$) $\Longleftrightarrow$ $\nabla^2 f(x)$
            positiv semidefinit für alle $x\in X$
        \item $\nabla^2 f(x)$ positiv definit für alle $x\in X$
            $\Longrightarrow$ $f$ strikt konvex (auf $X$)
        \item $f$ gleichmäßig konvex (auf $X$) $\Longleftrightarrow$
            $\nabla^2 f(x)$ gleichmäßig positiv definit auf $X$, d.h.,
            wenn es ein $\mu>0$ gibt mit
            \begin{align}\label{eq:3.11}
                d^\top\nabla^2 f(x)d\geq\mu\norm{d}^2
            \end{align}
            für alle $x\in X$ und für alle $d\in\mathbb{R}^n$.
    \end{compactenum}
\end{satz}

\begin{proof}$ $\newline
    \begin{itemize}%[leftmargin=-.1cm]
\item[\emph{(c)}] \underline{"$\Rightarrow$\!":} Sei $f$ gleichmäßig konvex. Wegen
    Satz \ref{3.7} \emph{(c)} ist
    $\nabla f$ gleichmäßig monoton, d.h. es existiert ein $\mu>0$ so,
    dass für alle $x,y\in X$ gilt
    \begin{align}\label{nablaf_glm_mon}
        (x-y)^\top(\nabla f(x)-\nabla f(y))\geq\mu\norm{x-y}^2.
    \end{align}
    Da $\nabla f$ stetig differenzierbar ist folgt mit einer geeigneten
    Konstanten $\mu>0$ also
    \begin{align*}
        d^\top\nabla^2 f(x) d
        &= d^\top \lim_{t\to 0} \frac{\nabla f(x+td) - \nabla f(x)}{t}\\
        &= \lim_{t\to 0} \frac{td^\top(\nabla f(x+td) - \nabla f(x))}{t^2}\\
        &\stackrel{\eqref{nablaf_glm_mon}}{\geq} \lim_{t\to 0} \frac{1}{t^2} \mu\norm{td}^2
        %~ =\lim_{t\to 0} \frac{2t\mu\norm{d}^2}{2t}
        = \lim_{t\to 0} \mu\norm{d}^2\\
        &= \mu\norm{d}^2\\
    \end{align*}
    für alle $x\in X$ und alle $d\in\mathbb{R}^n$, d.h., $\nabla^2 f(x)$
    ist gleichmäßig positiv definit (auf $X$).\\
    %
    \underline{"$\Leftarrow$\!":}
    Angenommen es gilt \eqref{eq:3.11},  Aus dem Mittelwertsatz in der
    Integralform \ref{MWS_Intform} und der Monotonie des Integrals
    ergibt sich
    \begin{align}    
        \qquad\quad(x-y)^\top(\nabla f(x)-\nabla f(y))
        &\stackrel{\eqref{MWS_intform}}{=}
            \int_0^1 (x-y)^\top \nabla^2f(y+\tau(x-y))(x-y)\dd \tau\nonumber\\
        &\stackrel{\eqref{eq:3.11}}{\geq} \mu\int_0^1 \norm{x-y}^2\dd\tau\nonumber\\
        &= \mu\norm{x-y}^2,\label{eq3.12}
    \end{align}
    d.h., nach Definition \ref{def3.6} (c) $\nabla f$ ist gleichmäßig
    monoton auf $X$. Wegen Satz
    \ref{3.7} \emph{(c)} ist $f$ selbst daher gleichmäßig konvex auf $X$.
\item[\emph{(a)}] Folgt aus (c) wenn man $\mu=0$ setzt.
    
\item[\emph{(b)}] Sei $\nabla^2 f(z)$ positiv definit für alle $z\in X$.
    Dann ist $\theta(\tau)\coloneqq (x-y)^\top\nabla^2 f(y+\tau(x-y))(x-y)>0$
    für alle $\tau\in[0,1]$ und alle $x,y\in X$ mit $x\neq y$. Folglich
    ist
    \begin{align}
        (x-y)^\top(\nabla f(x)-\nabla f(y))
        \stackrel{\eqref{MWS_intform}}{=} \int_0^1 \theta(\tau)\dd \tau>0
    \end{align}
    für alle $x,y\in X$ mit $x\neq y$, vgl. \eqref{eq3.12}. Also ist
    $\nabla f$ nach Definition \ref{def3.6} (b) strikt monoton und
    somit $f$ selbst strikt konvex aufgrund
    des Satzes \ref{3.7} \emph{(b)}.
    \end{itemize}
\end{proof}

Die Aussage des Satzes \ref{satz3.8} \emph{(b)} ist i.A. nur von
hinreichendem Charakter, denn die Funktion$f(x)\coloneqq x^4$ ist strikt
konvex, aber $\nabla^2 f(0)=0$ ist nur positiv semidefinit auf
$X=[0,1]\subseteq\mathbb{R}$:
\begin{itemize}
    \item Aus $f(x)=x^4$ folgt $\nabla f(x)=4x^3$ und $\nabla^2 f(x)=12x^2$.
    Also gilt $\nabla^2 f(0)=0$ und $f$ kann nicht positiv definit auf
    $X$ sein.
    \item $f(x)=x^4$ ist strikt konvex genau dann, wenn $\nabla f(x)$
    strikt monoton ist, d.h., wenn für alle $x,y\in X$ mit $x\neq y$ gilt
    $$
    (x-y)(\nabla f(x)-\nabla f(y))>0.
    $$
    O.B.d.A. können wir $x>y$ voraussetzen. Eingesetzt liefert dies
    $$
        \underbrace{(x-y)}_{>0}(4x^3-4y^3)>0
    $$
    was wegen $x^3>y^3$ erfüllt ist.
\end{itemize}

Dass Levelmengen von gleichmäßig konvexen Funktionen stets kompakt sind,
zeigen wir im folgenden

%~ \setcounter{section}{3}\setcounter{theorem}{8}
\begin{lemma}\label{lemma3.9}%[Lemma 3.9]
    Sei $f\colon \mathbb{R}^n\to\mathbb{R}$ stetig differenzierbar, $x^0\in\mathbb{R}^n$
    beliebig, die Levelmenge
    \begin{align}
        \mathscr{L}(x^0)
        \coloneqq
        \{ x\in\mathbb{R}^n \mid f(x)\leq f(x^0) \}
    \end{align}
    konvex und $f$ gleichmäßig konvex auf $\mathscr{L}(x^0)$. Dann ist
    die Menge $\mathscr{L}(x^0)$ kompakt.
\end{lemma}

\begin{proof}
    Die Levelmenge $\mathscr{L}(x^0)$ ist nichtleer, da
    $x^0\in\mathscr{L}(x^0)$.
    Die Funktion $f$ ist gleichmäßig konvex auf $\mathscr{L}(x^0)$,
    d.h., es existiert ein geeignetes $\mu>0$ mit $\lambda\coloneqq\frac{1}{2}$
    sodass für alle $x\in\mathscr{L}(x^0)$ und $y=x^0$ gilt (nach
    Einsetzen in die Definition \ref{def3.2} (c))
    \begin{align*}
        & f(\tfrac{1}{2} x+ \tfrac{1}{2}x^0) + \mu\tfrac{1}{2}\cdot\tfrac{1}{2}\norm{x-x^0}^2
        \leq \tfrac{1}{2} f(x) + \tfrac{1}{2}f(x^0)\\
        \Leftrightarrow\qquad
        & f(\tfrac{1}{2} (x+ x^0)) + \mu\tfrac{1}{4}\norm{x-x^0}^2
        \leq \tfrac{1}{2}\left( f(x) + f(x^0)\right)
    \end{align*}
    bzw. umgestellt
    \begin{align*}
        \tfrac{1}{4}\mu\norm{x-x^0}^2
        &\leq \tfrac{1}{2}( f(x) + f(x^0)) - f(\tfrac{1}{2} (x+ x^0))\\
        %~ \Leftrightarrow\qquad
        %~ & \tfrac{1}{4}\mu\norm{x-x^0}^2
        &= \tfrac{1}{2}\underbrace{( f(x) - f(x^0))}_{\leq0, \text{ da } x\in\mathscr{L}(x^0)}
        - \left(f(\tfrac{1}{2} (x+ x^0)) - f(x^0)\right)\\
        %~ \Leftrightarrow\qquad
        %~ & \tfrac{1}{4}\mu\norm{x-x^0}^2
        &\leq - \left(f(\tfrac{1}{2} (x+ x^0)) - f(x^0)\right)\\
        %~ & \tfrac{1}{4}\mu\norm{x-x^0}^2 \leq 
        &\!\!\!\!\stackrel{\ref{satz3.5} (a)}{\leq}
        - \tfrac{1}{2}\nabla f(x^0)^\top(x-x^0)\\
        %~ \Leftrightarrow\qquad
        %~ & \tfrac{1}{4}\mu\norm{x-x^0}^2
        &\leq 
        \tfrac{1}{2}\norm{\nabla f(x^0)^\top(x-x^0)}\\
        %~ \Leftrightarrow\qquad
        %~ & \tfrac{1}{4}\mu\norm{x-x^0}^2
        &\leq 
        \tfrac{1}{2}\norm{\nabla f(x^0)}\norm{(x-x^0)}
    \end{align*}
    und daraus folgt
    \begin{align*}
        \norm{x-x^0}\leq c\coloneqq \frac{2\norm{\nabla f(x^0)}}{\mu}
    \end{align*}
    für alle $x\in\mathscr{L}(x^0)$. Also ist $\mathscr{L}(x^0)$
    beschränkt. Aus Stetigkeitsgründen ist die Levelmenge
    $\mathscr{L}(x^0)$ aber auch abgeschlossen. Nach dem nach dem Satz
    von Heine-Borel ist $\mathscr{L}(x^0)$ also kompakt.
\end{proof}
    
    Es ist hierbei egal ob man $f$ als gleichmäßig konvex auf dem
    gesamten $\mathbb{R}^n$ oder auch nur auf einer konvexen Menge
    $X\subseteq\mathbb{R}^n$ voraussetzt: Es folgt automatisch, dass
    die Levelmenge $\mathscr{L}(x^0)$ konvex ist. Man könnte in diesem
    Fall also auf die explizit geforderte Voraussetzung der Konvexität
    von $\mathscr{L}(x^0)$ im Lemma \ref{lemma3.9} verzichten.


%~ \begin{satz}\label{satz310}% Satz 3.10
    %~ Sei $f\colon \mathbb{R}^n\to\mathbb{R}$ stetig differenzierbar und
    %~ $X\subseteq\mathbb{R}^n$ konvex. Man betrachte das restringierte
    %~ Optimierungsproblem
    %~ \begin{align}\label{eq:satz310}
        %~ \min{f(x)}\quad\text{u.d.N.}\quad x\in X.
    %~ \end{align}
    %~ Dann gelten die folgenden Aussagen:
    %~ \begin{compactenum}[(a)]
        %~ \item Ist $f$ konvex auf $X$, so ist die Lösungsmenge von \eqref{eq:satz310}
            %~ konvex (evtl. leer).
        %~ \item Ist $f$ strikt konvex auf $X$, so besitzt \eqref{eq:satz310}
            %~ höchstens eine Lösung.
        %~ \item Ist $f$ gleichmäßig konvex auf $X$ sowie $X$ nichtleer und
            %~ abgeschlossen, so besitzt \eqref{eq:satz310} genau eine
            %~ Lösung.
    %~ \end{compactenum}
%~ \end{satz}

%~ \begin{proof}
    %~ \red{asd}
%~ \end{proof}


%~ \begin{bemerkung}
    %~ \red{asd}
%~ \end{bemerkung}

%~ \begin{lemma}% Lemma 3.11
    %~ Sei $f\colon \mathbb{R}^n\to\mathbb{R}$ stetif differenzierbar,
    %~ $x^0\in\mathbb{R}^n$, die Levelmenge $\mathscr{L}(x^0)$ konvex, $f$
    %~ gleichmäßig konvex auf $\mathscr{L}(x^0)$ und $x^*\in\mathbb{R}^n$
    %~ das gemäß Satz \ref{satz310} (c) eindeutige globale Minimum von $f$.
    %~ Dann existiert ein $\mu>0$ mit
    %~ \begin{align}
        %~ \mu\norm{x-x^*}^2\leq f(x)-f(x^*)
    %~ \end{align}
    %~ für alle $x\in\mathscr{L}(x^0)$.
%~ \end{lemma}

%~ \begin{proof}
    %~ Da $x^*$ als globales Minimum von $f$ notwendigerweise ein stationärer
    %~ Punkt von $f$ ist, ergibt sich die behauptete Ungleichung aus dem
    %~ Satz \ref{satz310} (c).
%~ \end{proof}

%~ Die für die Optimierung vermutlich wichtigste Eigenschaft konvexer
%~ Funktionen formulieren wir folgenden

%~ \begin{satz}% Lemma 3.12
    %~ Sei $f\colon\mathbb{R}^n\to\mathbb{R}$ stetig differenzierbar und
    %~ konvex, sowie $x^*\in\mathbb{R}^n$ ein stationärer Punkt von $f$.
    %~ Dann ist $x^*$ ein globales Minimium von $f$ auf dem $\mathbb{R}^n$.
%~ \end{satz}

%~ \begin{proof}
    %~ Aus dem Satz \ref{satz310} (a) folgt
    %~ \begin{align}
        %~ f(x)-f(x^*)\geq\nabla f(x^*)^\top(x-x^*)=0
    %~ \end{align}
    %~ und damit
    %~ \begin{align}
        %~ f(x)\geq f(x^*)
    %~ \end{align}
    %~ für alle $x^*\in\mathbb{R}^n$. Also ist $x^*$ ein globales Minimum
    %~ von $f$.
%~ \end{proof}

%~ \begin{definition}
    %~ Seie $X\subseteq\mathbb{R}^n$ eine offene Menge und $f\colon X\to\mathbb{R}$
    %~ stetig differenzierbar. Dann heißt $f$ \emph{pseudokonvex} (auf $X$),
    %~ falls die Implikation
    %~ \begin{align}
        %~ \nabla f(y)^\top(x-y)\geq 0 \quad\Rightarrow\quad f(x)\geq f(y)
    %~ \end{align}
    %~ für alle $x,y\in X$ gilt.
%~ \end{definition}

%~ Zur Veranschaulichung dieser Definition sei zunächst $n=1$:
%~ \red{asd}

%~ \begin{beispiel}
    %~ Seien $a,b,\in\mathbb{R}^n, \alpha,\beta\in\mathbb{R},X\subseteq\in\mathbb{R}^n$
    %~ konvex mit der Eigenschaft
    %~ \begin{align}
        %~ b^\top x + \beta\neq 0\quad\text{für alle }x\in X
    %~ \end{align}
    %~ sowie $f\colon X\to\mathbb{R}$ definiert durch
    %~ \begin{align}
        %~ f(x)\coloneqq \frac{a^\top x+\alpha}{b^\top x+\beta}.
    %~ \end{align}
    %~ Die Funktion $f$ ist pseudokonvex, jedoch i.A. nicht konvex:
%~ \end{beispiel}



\end{document}
