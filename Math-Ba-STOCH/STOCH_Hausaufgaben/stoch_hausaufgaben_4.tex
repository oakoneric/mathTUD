\begin{exercisePage}[Gemeinsame Verteilungen][17/20]
	
%%%% AUFGABE 4.1 %%%%
	\begin{lemma} \label{lemma: 4_1_vandermonde}
		Für $m,n,k \in \N$ gilt \vspace{-\baselineskip}
		\begin{align} 
			\sum_{\ell = 0}^k \binom{n}{\ell} * \binom{m}{k - \ell} = \binom{n+m}{k}
			\label{eq: 4_1_vandermode}
		\end{align}
	\end{lemma}
	\begin{proof}
		Seien $N$ und $M$ disjunkte Mengen mit $\card{N} = n$ und $\card{M} = m$. Für alle $j \in \menge{0,1,\dots,k}$ sei $\mathcal{A}_\ell \defeq \menge{A \in \pows{N \cup M} : \card{A} = k , \card{A \cap N} = \ell , \card{A \cap M} = k- \ell}$. Wie bereits schon in Hausaufgabe 2.3 erläutert gibt es genau $\binom{n}{\ell}$ viele $\ell$-elementige Teilmengen von $N$ und $\binom{m}{k-\ell}$ viele $(k-\ell)$-elementige Teilmengen von $M$. Für die Kardinalität von $\mathcal{A}_\ell$ gilt dann $\card{\mathcal{A}_\ell} = \binom{n}{\ell} * \binom{m}{k-\ell}$. Da alle $\mathcal{A}_\ell$ paarweise disjunkt sind, gilt
		\begin{equation*}
			\binom{n+m}{k} = \sum_{\ell = 0}^k \card{\mathcal{A}_\ell} = \sum_{\ell = 0}^k \binom{n}{\ell} * \binom{m}{k-\ell}
		\end{equation*}
		für die Anzahl der $k$-elementigen Teilmengen von $N \cup M$.
	\end{proof}
	\begin{homework}
		\begin{enumerate}[leftmargin=*]
			\item Es seien $X \sim \Bin(n,p)$ und $Y \sim \Bin(m,p)$ zwei unabhängige, binomialverteilte Zufallsvariablen. Zeigen Sie, dass dann $X+Y \sim \Bin(n+m , p)$ ebenfalls Binomialverteilt ist, d.h. die Binomialverteilung ist stabil unter Faltungsoperationen.
			\item Es seien $X,Y \sim \Poisson(\lambda)$ unabhängige, identisch verteilte Zufallsvariablen. Bestimmen Sie $\P(X=k \mid X+Y = n)$ für $0 \leq k \leq n$.
		\end{enumerate}
	\end{homework}

	\begin{enumerate}[leftmargin=*, label=(zu \alph*)]
		\item Die unabhängigen Variablen $X$ und $Y$ haben die Zähldichten $\rho_X(k) = \binom{n}{k} p^k (1-p)^{n-k}$ und $\rho_Y(k) = \binom{m}{k} p^k (1-p)^{m-k}$. Da die Faltung gerade die Addition zweier Zufallsvariablen beschreibt, gilt nach Satz 3.25 $\P_{X+Y} = \P_X \star \P_Y$ und damit nach Proposition 3.26
		$\rho_X \star \rho_Y (k) = \sum_{\ell \in \Z} \rho_X(\ell) * \rho_Y(k - \ell)$ für $k \in \Z$. 
		Somit ist dann 
		\begin{equation*}
			\begin{aligned}
				\P(X+Y = k) 
				&= \sum_{\ell \in \Z} \binom{n}{\ell} p^\ell (1-p)^{n-\ell} * \binom{m}{k - \ell} p^{k-\ell} (1-p)^{m - k + \ell} \\
				&= \sum_{\ell \in \Z} \binom{n}{\ell} \binom{m}{k - \ell} p^k (1-p)^{n+m-k} \\
				\overset{\eqref{eq: 4_1_vandermode}}&{=} \binom{n+m}{k} p^k (1-p)^{(n+m)-k}
			\end{aligned}
		\end{equation*}
		%%%
		\item Seien $X,Y \sim \Poisson(\lambda)$ unabhängig und identisch verteilt, d.h. $\P(X=k) = \P(Y=k) = \frac{\lambda^k}{k!} e^{_\lambda}$.
		Aus der Vorlesung in Beispiel 3.27 ist bekannt, dass $\P(X+Y = k) = \frac{(2\lambda)^k}{k!} e^{-2\lambda}$ gilt. Somit ist dann 
		\begin{equation*}
			\begin{aligned}
				\P( X = k \mid X+Y = n) 
				&= \frac{\P( \menge{X=k} \cap \menge{X+Y=n})}{\P(\menge{X+Y=n})}
				= \frac{\P(X = k) * \P(Y = n-k)}{\P(X+Y=n)} \\
				&= \frac{\frac{\lambda^k}{k!} e^{-\lambda} * \frac{\lambda^{n-k}}{(n-k)!} e^{-\lambda}}{\frac{(2\lambda)^n}{n!} e^{-2 \lambda}} 
				= \frac{\lambda^n}{2^n \lambda^n} * \frac{n!}{k! (n-k)!} \\
				&= \binom{n}{k} * 2^{-n}
			\end{aligned}
		\end{equation*}
	\end{enumerate}


%%%% AUFGABE 4.2 %%%%
	\begin{homework}
		Es sei $X \sim \Geom(p)$ eine geometrisch verteilte Zufallsvariable. Zeigen Sie, dass $X$ gedächtnislos ist, d.h. $\P(X \geq i + j \mid X \geq i)$ für alle $i,j \geq 0$.
	\end{homework}
	
	Da $X$ geometrische verteilt ist, hat es die Zähldichte $\P(X=k) = p * (1-p)^k$. Dann gilt
	\begin{equation*}
		\P(X \geq j) = 1 - \P(X \leq j-1) = 1 - \sum_{n=0}^{j-1} p(1-p)^n  = 1 - p * \sum_{n=0}^{j-1} (1-p)^n
	\end{equation*}
	Da $1-p \leq 1$, gilt für die $(j-q)$-te Partialsumme der geometrischen Reihe
	\begin{equation*}
		\frac{1-(1-p)^{j-1}}{1 - (1-p)} = \frac{1 - (1-p)^{j-1}}{p}
	\end{equation*}
	Somit gilt dann
	\begin{equation*}
		1 - p * \sum_{n=0}^{j-1} (1-p)^n = 1 - (1 - (1-p)^{j-1}) = (1-p)^{j-1}
	\end{equation*}
	Analog gilt auch $\P(X \geq i) = (1-p)^{i-1}$ und $\P(X \geq i+j) = (1-p)^{i+j-1}$. Für die bedingte Wahrscheinlichkeit gilt schließlich
	\begin{equation*}
		\begin{aligned}
			\P(X \geq i+j \mid X \geq i) = \frac{\P( \menge{X \geq i+j} \cap \menge{X \geq i})}{\P(X \geq i)} &= \frac{\P(X \geq i+j)}{\P(X \geq i)} \\
			&= \frac{(1-p)^{i+j-1}}{(1-p)^i} = (1-p)^{j-1} = \P(X \geq j)
		\end{aligned}
	\end{equation*}
	und $X$ ist somit gedächtnislos.
	

%%%% AUFGABE 4.3 %%%%
	\begin{homework}
		$X,Y$ seien zwei Zufallsvariablen, deren gemeinsame Zähldichte aus \cref{tab: 4_3_zaehldichte} entnommen werden kann.
		
		{\centering
		\begin{tabular}{c||c|c|c|c} \label{tab: 4_3_zaehldichte}
			\diagbox[dir=NW, height=18pt, trim=l]{$X$}{$Y$} & -1 & 1 & 2 & 5 \\ \hline \hline
			-1 & $\frac{1}{27}$ & $\frac{1}{9}$ & $\frac{1}{27}$ & $\frac{1}{9}$ \\
			1  & $\frac{1}{ 9}$ & $\frac{2}{9}$ & $\frac{1}{ 9}$ & $0$ \\
			5  & $\frac{4}{27}$ & $\frac{1}{9}$ & $0$            & $0$ \\
		\end{tabular}
		\captionof{table}{Gemeinsame Zähldichte von $X$ und $Y$}}
		
		\begin{enumerate}[leftmargin=*, nolistsep]
			\item Bestimmen Sie die Wahrscheinlichkeit der Ereignisse ``$Y$ gerade'' und ``$X * Y$ ungerade''.
			\item Finden Sie die Verteilung $X+Y$. 
			\item Sind $X$ und $Y$ unabhängig.
		\end{enumerate}
	\end{homework}
	
	\begin{enumerate}[leftmargin=*, label=(zu \alph*)]
		\item~\vspace{-\baselineskip}
		\begin{equation*}
			\begin{aligned}
				\P(Y \text{ gerade}) 
				&= \P(Y = 2) = \frac{1}{27} + \frac{1}{9} = \frac{4}{27} \\
				\P( X*Y \text{ ungerade}) 
				&= \frac{1}{27} + \frac{3}{27} + \frac{3}{27} + \frac{3}{27} + \frac{6}{27} + 0 + \frac{4}{27} + \frac{3}{27} + 0
				= \frac{23}{27}
			\end{aligned}
		\end{equation*}
		\item%~\vspace{-\baselinskip}
		Sei $\ell = (-1,1,5)$. Dann ist $\P(X+Y = k) = \sum_{i=1}^3 \P(X = \ell_i , Y = k - \ell_i)$.
		
		{\centering
			\begin{tabular}{l||c|c|c|c|c|c|c|c|c} \label{tab: 4_3_zaehldichte_X+Y}
				$k$ & $-2$ & $0$ & $1$ & $2$ & $3$ & $4$ & $6$ & $7$ & $10$ \\ \hline 
				$\P(X+Y=k)$ & $\frac{1}{27}$ & $\frac{2}{9}$ & $\frac{1}{27}$ & $\frac{2}{9}$ & $\frac{1}{9}$ & $\frac{7}{27}$ & $\frac{1}{9}$ & $0$ & $0$ \\
			\end{tabular}
			\captionof{table}{Zähldichte von $X + Y$}}
		
		\item Wir betrachten $X=1$ und $Y=5$. Dann gilt
		\begin{equation*}
			\P(X=1,Y=5 = \frac{1}{9} \quad \und \quad \P(X=1) * \P(X=5) = \frac{4}{9} * \frac{1}{9}
		\end{equation*}
		und somit $\P(X=1,Y=5) \neq \P(X=1) * \P(X=5)$. Damit sind $X$ und $Y$ nicht stochastisch unabhängig.
	\end{enumerate}


%%%% AUFGABE 4.4 %%%%
	\begin{homework}
		Geben Sie zwei Zufallsvariablen $X$ und $Y$ an, sodass zwar $X$ und $Y$ abhängig sind, aber $X^2$ und $Y^2$ unabhängig.
	\end{homework}

	\bigskip

%%%% AUFGABE 4.5 %%%%
	\begin{homework}
		Auf der $(x,y)$-Ebene werden parallele Geraden $y = 2a*k$ ($k \in \Z$) mit Abstand $2a$ eingezeichnet. Eine Strecke (``Nadel'') der Länge $2 \ell$ mit $\ell < a$ wird zufällig auf die Ebene geworfen. Wie groß ist die Wahrscheinlichkeit dafür, dass die Strecke eine der Geraden schneidet?
	\end{homework}

	Wir betrachten zwei Variablen: $x$ gebe den Abstand des Nadelmittelpunktes von der nächstliegenden Geraden an und $\phi$ beschreibe den Winkel zwischen der durch die Nadel beschriebenen Geraden und der nächstliegenden Geraden. Damit die Nadel eine der Geraden kreuzt (oder berührt) muss $x \leq 2\ell * \sin(\phi)$ gelten. Somit lassen sich alle möglichen Lagepositionen einschränken zur Ergebnismenge $\Omega = \menge{(x,\phi) \in [0,a] \times [0,\pi)}$. Diese hat offensichtlich Lebesgue-Maß $\lambda(\Omega) = a*\pi$.
	Betrachten wir nun das Ereignis $S = \menge{(x,\phi) \in \Omega : x \leq \ell * \sin(\phi)}$, was genau die Positionen einer Nadel beschreibt, bei denen es zum Schnitt mit einer der Geraden kommt. Diese Menge repräsentiert also die Fläche unter dem Graphen der Funktion $y(\phi) = \ell * \sin(\phi)$. Damit ergibt sich ein Lebesgue-Maß
	\begin{equation*}
		\lambda(S) = \int_S \diff{\lambda} 
		= \int \one_{[0,\ell*\sin(\phi)] \times [0,\pi)} \diff{\lambda}
		= \int_0^\pi \ell * \sin(\phi) \diff{\phi}
		= \ell * \left[ - \cos(\phi) \right]_0^\pi = 2\ell
	\end{equation*}
	Nun können wir den Nadelwurf als zufällig insofern annehmen, dass die Lage des Mittelpunktes vom Winkel unabhängig ist und alle möglichen Positionen gleich wahrscheinlich sind.
	Dann ergibt sich mit der stetigen Gleichverteilung $\P(S) = \frac{\lambda(S)}{\lambda(\Omega)} = \frac{2\ell}{a\pi}$ als Wahrscheinlichkeit für den Schnitt der Nadel mit einer der Geraden.

\end{exercisePage}