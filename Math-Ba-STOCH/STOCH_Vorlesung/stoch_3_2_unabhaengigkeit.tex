\section{Stochastische Unabhängigkeit} \label{sec: unabhaengigkeit}
In vielen Fällen besagt die Intuition über verschiedene Zufallsexperimente / Ereignisse, dass diese sich \textit{nicht} gegenseitig beeinflussen. Für solche $A,B \in \F$ mit $\P(A) > 0$ und $\P(B) > 0$ sollte gelten
\begin{equation*}
	\P(A\mid B) = \P(A) \quad \und \quad \P(B\mid A) = \P(B).
\end{equation*}

\begin{definition} \label{def: 3_2_11}
	Sei $(\Omega, \F, \P)$ Wahrscheinlichkeitsraum. Zwei Ereignisse $A,B \in \F$ heißt \begriff{(stochastisch) unabhängig} bezüglich $\P$, falls
	\begin{equation*}
		\P(A\cap B) = \P(A) * \P(B)
	\end{equation*}
	Wir schreiben auch $A \upmodels B$.
\end{definition}

\begin{beispiel}
	Wir betrachten wieder das Würfeln mit 2 fairen, sechsseitigen Würfeln. 
	\begin{equation*}
		\Omega = \menge{(i,j) \mid i,j \in\menge{1,\dots,n}} \qquad 
		\F = \pows\Omega \qquad 
		\P = \Uni(\Omega)
	\end{equation*}
	Betrachte
	\begin{equation*}
	\begin{aligned}
		A &\defeq \menge{(i,j) \in \Omega \colon i \text{ gerade}}\\
		B &\defeq \menge{(i,j) \in \Omega \colon j \leq 2}
	\end{aligned}
	\end{equation*}
	In diesem Fall, erwarten wir intuitiv die Unabhängigkeit von $A$ und $B$.
	In der Tat ist $\P(A) = \lfrac{1}{2}$, $\P(B) = \lfrac{1}{3}$ und $\P(A\cap B) = \lfrac{1}{6}$, womit $\P(A \cap B) = \P(A) * \P(B)$ erfüllt ist.
	Betrachte nun
	\begin{equation*}
	\begin{aligned}
		C &\defeq \menge{(i,j) \in \Omega \colon i + j = 7}\\
		D &\defeq \menge{(i,j) \in \Omega \colon i = 6}
	\end{aligned}
	\end{equation*}
	Dann gilt $\P(C) = \lfrac{1}{6}$ und $\P(D) = \lfrac{1}{6}$. Wegen $C \cap D = \menge{(6,1)}$ folgt
	\begin{equation*}
		\P(C\cap D) = \frac{1}{36} = \frac{1}{6} * \frac{1}{6} = \P(C) * \P(D)
	\end{equation*}
	$C$ und $D$ sind also \textit{stochastisch} unabhängig, obwohl eine kausale Abhängigkeit vorliegt.
\end{beispiel}

\begin{definition}
	Sei $(\Omega, \F, \P)$ ein Wahrscheinlichkeitsraum und $I \neq \emptyset$ eine endliche Indexmenge. Dann heißt die Familie $(A_i)_{i \in I}$ von Ereignissen in $\F$ \begriff{unabhängig} bezüglich $\P$, falls für alle $\emptyset \neq J \subseteq I$ gilt
	\begin{equation*}
		\P\brackets{\bigcap_{i \in J} A_i} = \prod_{i \in J} \P(A_i)
	\end{equation*}
	Offensichtlich impliziert die Unabhängigkeit einer Familie die paarweise Unabhängigkeit je zweier Familienmitglieder nach \cref{def: 3_2_11}. Umgekehrt gilt dies nicht.
\end{definition}

\begin{beispiel}[Abhängigkeit trotz paarweiser Unabhängigkeit]
	Wir betrachten ein zweifaches Bernoulliexperiment mit Erfolgswahrscheinlichkeit $\lfrac{1}{2}$, d.h.
	\begin{equation*}
		\Omega = \menge{0,1}^2 \qquad \F = \pows\Omega \qquad \P = \Uni(\Omega)
	\end{equation*}
	sowie
	\begin{equation*}
	\begin{aligned}
		A &= \menge{1} \times \menge{0,1} \qquad &&\text{(Münzwurf: erster Wurf ist Zahl)}\\
		B &= \menge{0,1} \times \menge{1} &&\text{(Münzwurf: zweiter Wurf ist Zahl)}\\
		C &= \menge{(0,0), (1,1)} &&\text{(beide Würfe haben selbes Ergebnis)}
	\end{aligned}
	\end{equation*}
	Dann gelten $\P(A) = \lfrac{1}{2} = \P(B) = \P(C)$ und
	\begin{equation*}
	\begin{aligned}
		\P(A\cap B) &= \P(\menge{(1,1)}) = \lfrac{1}{4} = \P(A) * \P(B)\\
		\P(A\cap C) &= \P(\menge{(1,1)}) = \lfrac{1}{4} = \P(A) * \P(C)\\
		\P(B\cap C) &= \P(\menge{(1,1)}) = \lfrac{1}{4} = \P(B) * \P(C)
	\end{aligned}
	\end{equation*}
	Daraus folgt also paarweise Unabhängigkeit. Jedoch ist
	\begin{equation*}
		\P(A \cap B \cap C) = \P(\menge{(1,1)}) = \frac{1}{4} \neq \P(A) * \P(B) * \P(C)
	\end{equation*}
	und $A,B,C$ sind \textit{nicht} stochastisch unabhängig.
\end{beispiel}

\begin{definition}[Unabhängige $\sigma$-Algebren] \label{3_15_def}
	Seien $(\Omega, \F,\P)$ ein Wahrscheinlichkeitsraum, $I \neq \emptyset$ eine Indexmenge und $(E_i, \Ecal_i)$ Messräume.
	\begin{enumerate}[leftmargin=*]
		\item Die Familie $\F_i \subset \F (i \in I)$, heißen \begriff{unabhängig}, wenn für die $\emptyset \neq J \subseteq I$ mit $\abs{J} < \infty$ gilt
		\begin{equation*}
			\P\brackets{\bigcap_{i \in J} A_i} = \prod_{i \in J} \P(A_i) \qquad \text{ für beliebige } A_i \in \F_i (i \in J)
		\end{equation*}
		\item Die Zufallsvariable $\abb{X_i}{(\Omega, \F)}{(E_i, \Ecal_i)} (i \in I)$, heißen \begriff{unabhängig}, wenn die $\sigma$-Algebren
		\begin{equation*}
			\sigma(X_i) = X^{-1}(\Ecal_i) = \menge{\menge{X_i \in F} \colon F \in \Ecal_i} \quad (i \in I)
		\end{equation*}
		unabhängig sind.
	\end{enumerate}
\end{definition}

\begin{lemma}[Zusammenhang der Definitionen]
	\label{3_16_lemma}
	Sei $(\Omega,\F,\P)$ ein Wahrscheinlichkeitsraum, $I \neq \emptyset$, $A_i \in \F_i$ für alle $i \in I$. 
	Die folgenden Aussagen sind äquivalent:
	\begin{enumerate}[nolistsep]
		\item Die Ereignisse $A_i$ ($i \in I$) sind unabhängig. 
		\item Die $\sigma$-Algebren $\sigma(A_i)$ ($i \in I$) sind unabhängig.
		\item Die Zufallsvariablen $\one_{A_i}$ ($i \in I$) sind unabhängig.
	\end{enumerate}
\end{lemma}
\begin{proof}
	Da die Unabhängigkeit über endliche Teilemengen definiert ist, können wir oBdA $I = \menge{1, \dots, n}$ annehmen. 
	\begin{itemize}[leftmargin=*, nolistsep]
		\item Da $\sigma(\one_{A_i}) = \sigma(A_i)$ folgt die Äquivalenz von (2) und (3) direkt aus \cref{3_15_def}.
		\item Zudem ist (2) $\to$ (1) klar.
		\item Für (1) $\to$ (2) genügt es zu zeigen, dass
		\begin{equation*}
			A_1, \dots, A_n \text{ unabhängig } \Rightarrow B_1, \dots, B_n \text{ unabhängig mit } B_i \in \menge{\emptyset, A_i, A_i^\complement, \Omega}
		\end{equation*}
		Rekursiv folgt dies bereits aus
		\begin{equation*}
			A_1,\dots, A_n \text{ unabhängig } \Rightarrow B_1, A_2, \dots, A_n \text{ unabhängig mit } B_1 \in \menge{\emptyset, A_1, A_1^\complement, \Omega}.
		\end{equation*}
		Für $B_1 \in \menge{\emptyset, A_1, \Omega}$ ist dies klar.\\
		Sei also $B_1 = A_1^\complement$ und $J \subseteq I$, $J \neq \emptyset$. Falls $1 \notin J$, ist nichts zu zeigen. Sei $1 \in J$, dann gilt mit $A = \bigcap_{i\in J, i \neq 1} A_i$ sicherlich
		\begin{equation*}
			\begin{aligned}
				\P\brackets{A_1^\complement \cap A} 
				&= \P(A \setminus (A_1 \cap A))\\
				&= \P(A) - \P(A_1 \cap A)\\
				&= \prod_{i\in J \setminus \menge{1}} \P(A_i) - \prod_{i\in J}(A_i) \\
				&= (1- \P(A_1))\prod_{i\in J\setminus \menge{1}} \P(A_i)\\
				&= \P\brackets{A_1^\complement} \prod_{i\in J \setminus \menge{1}} \P(A_i)
			\end{aligned}
		\end{equation*}
	\end{itemize}
\end{proof}

Insbesondere zeigt \cref{3_16_lemma}, dass wir in einer Familie unabhängiger Ereignisse beliebig viele Ereignisse durch ihr Komplement, $\emptyset$ oder $\Omega$ ersetzen können ohne die Unabhängigkeit zu verlieren.

\begin{satz}
	\label{3_17_satz}
	Sei $(\Omega, \F, \P)$ Wahrscheinlichkeitsraum und $\F_i \subseteq \F, i \in I$, seien $\cap$-stabile Familien von Ereignissen. Dann gilt
	\begin{equation*}
		\F_i (i \in I) \text{ unabhängig } \equivalent \sigma(\F_i) (i \in I) \text{ unabhängig}. 
	\end{equation*}
\end{satz}

\begin{proof}
	oBdA sei $I = \menge{1, \dots, n}$ und $\Omega \in \F_i, i \in I$.
	\begin{proof-equivalence}[leftmargin=*, nolistsep]
		\rueckrichtung trivial, da $\F_i \subseteq \sigma(\F_i)$ und das Weglassen von Mengen erlaubt ist.
		\hinrichtung zeigen wir rekursiv
		\begin{enumerate}[label=(\roman*), nolistsep]
			\item Wähle $F_i \in \F_i, i = 2, \dots,n$ und definiere für $F \in \sigma(\F_i)$ die endlichen Maße
			\begin{equation*}
				\mu(F) = \P\brackets{ F \cap F_2 \cap \cdots \cap F_n} \und \nu(F) = \P(F) \, \P(F_2) \, \dots \, \P(F_n)
			\end{equation*}
			\item Da die Familien $\F_i$ unabhängig sind, gilt
			$\mu\mid_{\F_1} = \nu\mid_{\F_1}$.
			Nach dem Eindeutigkeitssatz für Maße (\cref{satz: 1.9_eindeutigkeitssatz}) folgt $\mu\mid_{\sigma(\F_1)} = \nu\mid_{\sigma(\F_1)}$ also
			\begin{equation*}
				\P\brackets{ F \cap F_2 \cap \cdots \cap F_n} = \P(F) \P(F_2) \dots \P(F_n)
			\end{equation*}
			für alle $F \in \sigma(\F_i)$ und $F_i \in \F_i, i = 1, \dots, n$. Da $\Omega \in \F_i$ für alle $i$ gilt die erhaltene Produktformel auf für alle Teilemengen $J \subseteq I$.\\
			Also sind $\sigma(\F_1), \F_2, \dots, \F_n$ unabhängig.
			\item Wiederholtes Anwenden von (i) und (ii) liefert den Satz.
		\end{enumerate}
	\end{proof-equivalence}
\end{proof}
%
Mit \cref{3_17_satz} folgen:

\begin{korollar}	\label{3_18_kor}
	Sei $(\Omega,\F,\P)$ ein Wahrscheinlichkeitsraum und $\F_{i,j} \subseteq \F$ mit $1 \le i \le n, 1 \le j \le m(i)$ unabhängige, $\cap$-stabile Familien.
	Dann sind auch
	\begin{equation*}
		\G_i = \sigma(\F_{i,1}, \dots , \F_{i,m(i)}), \quad 1 \le i \le n
	\end{equation*}
	unabhängig.
\end{korollar}
\begin{proof}
	oBdA sei $\Omega \in \F_{i,j} \enskip \forall i,j$, da $\Omega$ die Unabhängigkeit nach \cref{3_16_lemma} nicht zerstört.
	Dann sind die Familien 
	\begin{equation*}
		\F_i^\cap \defeq \menge{F_{i,1} \cap \cdots \cap F_{i,m(i)} : F_{i,j} \in \F_{i,j}, 1 \le j \le m(i)} \enskip (1 \le i \le n)
	\end{equation*}
	$\cap$-stabil, unabhängig und es gilt $\F_{i,1}, \dots , \F_{i,m(i)} \subseteq \F_i^\cap$ (vgl. Hausaufgabe). Nach \cref{3_17_satz} sind auch $\sigma(\F_i^\cap)$ unabhängig. Damit folgt die Behauptung, da aus 
	\begin{equation*}
		\F_{i,1}, \dots , \F_{i,m(i)} \subseteq \F_i^\cap \subseteq \sigma(\F_{i,1}, \dots , \F_{i,m(i)}) = \G_i
	\end{equation*} unter Anwendung des Erzeugenden-Operators $\G_i = \sigma(\F_{i,1}, \dots , \F_{i,m(i)}) \subseteq \sigma(\F_i^\cap) \subseteq \G_i$ folgt, d.h. $\sigma(\F_i^\cap) = \G_i$.
\end{proof}

\begin{korollar} \label{3_19_korollar}
	Sei $(\Omega,\F,\P)$ ein Wahrscheinlichkeitsraum und $\abb{X_{i,j}}{\Omega}{E}$ ($1 \le i \le n, 1 \le j \le m(i)$)
	unabhängige Zufallsvariablen. Zudem seinen $\abb{f_i}{E^{m(i)}}{\R}$ messbar. Dann sind auch die Zufallsvariablen $f_i(X_{i,1}, \dots, X_{i,m(i)})$ mit $1 \le i \le n$ unabhängig.
\end{korollar}
\begin{proof}
	Setze $\F_{i,j} = \sigma(X_{i,j})$ und $\G_i = \sigma(\F_{i,1} , \dots , \F_{i,m(i)})$. Dann sind nach \cref{3_18_kor} die $\G_i$ ($i = 1, \dots , n)$ unabhängig. Zudem ist $Y_i \defeq f_i(X_{i,1}, \dots , X_{i,m(i)})$ $\G_i$-messbar, also $\sigma(Y_i) \subseteq \G_i$. Damit erben die $Y_i$ die Unabhängigkeit der $\G_i$.
\end{proof}

\begin{beispiel}
	Seien $X_1, \dots, X_n$ unabhängige, reelle Zufallsvariablen. Dann sind auch
	\begin{equation*}
	Y_1 = X_1, Y_2 = X_2 + \cdots + X_n
	\end{equation*}
	unabhängig.
\end{beispiel}

\begin{satz}[Unabhängigkeit von Zufallsvariablen]
	\label{3_21_satz}
	Seien $(\Omega, \F, \P)$ ein Wahrscheinlichkeitsraum und $\abb{X_1, \dots, X_n}{(\Omega,\F)}{(E, \Ecal)}$ Zufallsvariablen. Dann sind äquivalent:
	\begin{enumerate}[label=(\arabic*), nolistsep]
		\item $X_1 , \dots , X_n$ sind unabhängig.
		\item $\P(X_1 \in A_1 , \dots , X_n \in A_n) = \prod_{i=1}^n \P(X_i \in A_i) \quad \forall A_1, \dots , A_n \in \Ecal$
		\item Die gemeinsame Verteilung der $X_i$ entspricht dem Produktmaß der einzelnen Verteilungen
		\begin{equation*}
			\P_{X_1 , \dots , X_n} = \P_{X_1} \otimes \dots \otimes \P_{X_n}
		\end{equation*}
	\end{enumerate}
\end{satz}
\begin{proof}
	per Ringschluss 
	\begin{description}[nolistsep]
		\item [(1 $\boldsymbol{\Rightarrow}$ 2)] Seien $A_1, \dots, A_n \in \Ecal$ beliebig. Dann gilt per Definition
		\begin{equation*}
			\begin{aligned}
				\P_{X_1 , \dots , X_n}(A_1 \times \dots A_n) 
				&= \P(X_1 in A_1, \dots , X_n \in A_n) \\
				&= \P\brackets{\bigcap_{i=1}^n \menge{X_i \in A_i}} \\
				&= \prod_{i=1}^n \P(X_i \in A_i) \\
				&= \prod_{i=1}^n \P_{X_i}(A_i) \\
				&= \brackets{\bigotimes_{i=1}^n \P_{X_i}} \brackets{A_1 \times \dots \times A_n}
			\end{aligned}
		\end{equation*}
		\item [(2 $\boldsymbol{\Rightarrow}$ 3)] Aus der obigen Gleichung sehen wir, dass (2) bereits (3) impliziert für alle Rechtecke $A_1 \times \dots \times A_n$. Da die Familie der Rechtecke $\cap$-stabil ist und $\Ecal^{\otimes n}$ erzeugt, folgt die Aussage aus dem Eindeutigkeitssatz für Maße.
		\item [(3 $\boldsymbol{\Rightarrow}$ 1)] Es sei $J \subseteq \menge{1, \dots, n}$ und 
		\begin{equation*}
			A_i \defeq \begin{cases}
			\text{beliebig in } \Ecal & \text{falls } i \in J \\
			E & \text{falls } i \notin J
			\end{cases}
		\end{equation*}
		Dann ist
		\begin{equation*}
			\begin{aligned}
			\P(X_i \in A_i : i \in  J) 
			&= \P(X_i \in A_i : i = 1 , \dots , n) \\
			&= \prod_{i=1}^n \P(X_i \in A_i) \\
			&= \prod_{i \in J} \P(X_i \in A_i)
			\end{aligned}
		\end{equation*}
	\end{description}
\end{proof}

\begin{beispiel}
	Im Urnenmodell mit Zurücklegen hat der Vektor $X = (X_1, \dots , X_n)$ mit $X_i$ = Farbe im $i$-ten Zug als Zähldichte die Produktdichte der $X_i$. $X_1 , \dots , X_n$ sind also unabhängig.
\end{beispiel}