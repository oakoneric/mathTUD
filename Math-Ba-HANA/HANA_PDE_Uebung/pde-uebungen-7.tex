\begin{exercisePage}
	\setcounter{taskcount}{17}
	
	\begin{task}
		Für $x\in\mathbb R^n$, $t\in \mathbb R$ und $r>0$ ist
		\begin{equation*}
			E(x,t;r) = \left\{(y,s)\in \R^{n+1}\mid s\leq t, \Phi(x-y,t-s)\geq \frac{1}{r^n}\right\}
		\end{equation*}
		wobei $\Phi$ die Fundamentallösung der Wärmeleitungsgleichung ist.
		Zeigen Sie
		\begin{equation*}
			E(0,0;r) =\left\{ (y,s)\in \R^{n+1}\mid s\leq 0, |y|^2\leq 2n s\ {\ln}\left(-\frac{4\pi s}{r^2}\right)\right\}
		\end{equation*}
		und bestimmen Sie für jedes $s\in \R$ den Schnitt $B_s := \{y\in \R^n \mid (y,s)\in E(0,0;r)\}$
		
		\textit{Hinweis}: $E(x,t;r)$ ist der aus der Vorlesung bekannte \textit{Wärmeball}. Dabei ist $\Phi(x,0)$ als Grenzwert $\lim_{ s\to 0+} \Phi(x,s)$ zu interpretieren. Analoges gilt in der Darstellung von $E(0,0;r)$.
	\end{task}

	Es ist $\Phi(x,t) = (4 \pi t)^{-\frac{n}{2}} e^{- \frac{\abs{x}^2}{4t}}$ für $t>0$ die Fundamentallösung der Wärmeleitungsgleichungen. Betrachten wir $E(0,0;r)$. Die Bedingung $s \le t$ wird damit automatisch zu $s \le 0$ und wir können umstellen:
	\begin{align*}
		\Phi(-y,-s) = (-4 \pi s)^{-\frac{n}{2}} e^{- \frac{\abs{y}^2}{-4s}} &\ge \frac{1}{r^n} 
		&\equivalent && e^{- \frac{\abs{y}^2}{-4s}} &\ge \frac{(-4 \pi s)^{\frac{n}{2}}}{r^n} \\
		&&\equivalent && - \frac{\abs{y}^2}{-4s} &\ge \ln \brackets{ \frac{-4 \pi s}{r^2} }^{\frac{n}{2}} \\
		&&\equivalent && \frac{\abs{y}^2}{-4s} &\le \frac{n}{2} * \ln \brackets{ \frac{r^2}{-4 \pi s}} \\
		&&\equivalent && \abs{y}^2 &\le 2ns * \ln \brackets{- \frac{4 \pi s}{r^2}}
	\end{align*}
	Betrachten wir nun $B_s \defeq \menge{y \in \Rn : (y,s) \in E(0,0;r)}$. Wir unterscheiden fünf Fälle:
	\begin{enumerate}[label=(\roman*), leftmargin=*]
		\item Ist $s > 0$, so ist $B_s = \emptyset$.
		\item Ist $s = 0$, so ist
		\begin{equation*}
			\Phi(-y,0) = \lim_{t \to 0+} \Phi(-y,t) = \begin{cases}
			0 & \text{wenn } y \neq 0 \\
			\infty & \text{wenn } y = 0
			\end{cases}
		\end{equation*} 
		Damit kann $\Phi(-y,0) \ge \frac{1}{r^2} > 0$ nur für $y = 0$ erfüllt sein, also $B_s = B_0 = \menge{(0,0)}$.
		\item Sein nun $s < 0$. Dann müssen wir in obiger Umstellung 
		\begin{equation*}
			\abs{y}^2 \le 2ns * \ln \brackets{- \frac{4 \pi s}{r^2}}
		\end{equation*}
		beachten, wann die rechte Seite positiv bzw. Null wird. Da $s < 0$, sind alle Ausdrücke definiert. Es ist
		\begin{equation*}
			0 = 2ns * \ln \brackets{- \frac{4 \pi s}{r^2}} \equivalent - \frac{4 \pi s}{r^2} = 0 \equivalent s = -\frac{r^2}{4\pi}
		\end{equation*}
		und somit $B_s = B_{-\frac{r^2}{4\pi}} = \menge{(0,0)}$. 
		Abschließend ist 
		\begin{equation*}
			0 > 2ns * \ln \brackets{- \frac{4 \pi s}{r^2}} \equivalent - \frac{4 \pi s}{r^2} < 0 \equivalent s > -\frac{r^2}{4\pi}
		\end{equation*}
		d.h. $B_s = \emptyset$ für $s < -\frac{r^2}{4\pi}$ und
		\begin{equation*}
			B_s = B_r(0) \mit r = \sqrt{2ns * \ln\brackets{-\frac{4\pi s}{r^2}}} \qquad \text{für } -\frac{r^2}{4\pi} < s < 0
		\end{equation*}
		Zusammengefasst gilt also
		\begin{equation*}
			B_s = \begin{cases}
				\emptyset & \text{für } s > 0 \\
				\menge{(0,0)} & \text{für } s = 0 \\
				\menge{y \in \Rn : \abs{y} \le \sqrt{2ns * \ln\brackets{-\frac{4\pi s}{r^2}}}} & \text{für } -\frac{r^2}{4\pi} < s < 0 \\
				\menge{(0,0)} & \text{für } s = -\frac{r^2}{4\pi} \\
				\emptyset & \text{für } s < -\frac{r^2}{4\pi}
			\end{cases}
		\end{equation*}
	\end{enumerate}
	
	\begin{task}
		Lösen Sie die \textit{Wellengleichung} für $n=1$, also das Problem
		\begin{align*}
			u_{tt} - u_{xx} = 0\quad&\text{in } \R \times(0,\infty),\\
			u = g,\quad u_t = h\quad&\text{auf } \R \times \{0\}
		\end{align*}
		Gehen Sie dazu wie folgt vor:
		\begin{enumerate}
			\item Zeigen Sie, dass die allgemeine Lösung der partiellen Differentialgleichung $\quer{u}_{\eta\xi} = 0$ auf $\R^2$.
			gegeben ist durch $\overline{u}(\eta,\xi) = F(\eta) + G(\xi)$, wobei $F,G \in C^1(\R)$ beliebig sind.
			\item 
			Betrachten Sie im $\mathbb{R}^2$ die Koordinatentransormation
			\begin{equation*}
				\phi(t,x) = (t + x,t - x)\quad \text{für } x,t\in \mathbb{R}.
			\end{equation*}
			und zeigen Sie, dass die Gleichung $u_{tt} - u_{xx} = 0$ genau dann erfüllt ist, wenn $\overline{u}_{\eta\xi} = 0$ für
			$\quer{u}(\eta, \xi) := (u \circ \phi^{-1})(\eta,\xi)$ gilt.
			\item 
			Nutzen Sie (a) und (b), um die \textit{Formel von d'Alembert} herzuleiten, also die Lösungs\-darstellung
			\begin{equation*}
				u(x,t) = \tfrac12\left(g(x+t)+g(x-t)\right) + \tfrac1{2} \int_{x-t}^{x+t} h(y)\,\mathrm{d}y
			\end{equation*}
		\end{enumerate}
		\textit{Hinweis}: Die Wellengleichung wird demnächst Gegenstand der Vorlesung sein. Dort wird die Formel von d'Alembert geschickt mittels der Methode der Charakteristiken hergeleitet.
	\end{task}

	\begin{enumerate}[label=(zu \alph*), leftmargin=*]
		\item Sei $\quer{u}(\eta, \xi) = F(\eta) + G(\xi)$ für $F,G \in C^1(\R)$. Dann ist $\quer{u}_\eta(\eta, \xi) = F'(\eta)$ und $\quer{u}_{\eta \xi} (\eta, \xi) = 0$ für alle $\eta, \xi \in \R$.
		\item Es ist $\phi^{-1}(\eta, \xi) = \frac{1}{2}(\eta + \xi, \eta - \xi)$. Damit gilt unter Nutzung des Satz von Schwarz
		\begin{equation*}
			\begin{aligned}
				&\partdiff{\xi} \partdiff{\eta} u\brackets{\frac{\eta + \xi}{2}, \frac{\eta - \xi}{2}} \\ 
				= \quad &\partdiff{\xi} \brackets{\tfrac{1}{2} \partial_1 u\brackets{\frac{\eta + \xi}{2}, \frac{\eta - \xi}{2}} + \tfrac{1}{2} \partial_2 u\brackets{\frac{\eta + \xi}{2}, \frac{\eta - \xi}{2}}} \\
				= \quad & \tfrac{1}{4} \partial_{11} u\brackets{\frac{\eta + \xi}{2}, \frac{\eta - \xi}{2}} 
				- \tfrac{1}{4} \partial_{12} u\brackets{\frac{\eta + \xi}{2}, \frac{\eta - \xi}{2}}  \\
				+ &\tfrac{1}{4} \partial_{21} u\brackets{\frac{\eta + \xi}{2}, \frac{\eta - \xi}{2}} 
				- \tfrac{1}{4} \partial_{22} u\brackets{\frac{\eta + \xi}{2}, \frac{\eta - \xi}{2}} \\
				= \quad &\tfrac{1}{4} \brackets{u_{tt}(t,x) - u_{xx}(t,x)} 
				\enskip = \enskip 0 \\
				\equivalent 0 \quad = \quad &u_{tt}(t,x) - u_{xx}(t,x)
			\end{aligned}
		\end{equation*}
		\item Mit (a) und (b) ist $u\brackets{\frac{\eta + \xi}{2}, \frac{\eta - \xi}{2}} = F(\eta) + G(\xi)$ bzw. $u(x,t) = F(x+t) + G(x-t)$. Mit den Randwerten ist
		\begin{align*}
			u(x,0) &= F(x) + G(x) = g(x) & \text{für alle } x \in \R \tag{I} \\
			u_t(x,0) &= F'(x) - G'(x) = h(x) & \text{für alle } x \in \R
		\end{align*}
		Aus der zweiten Randwertbedingung erhalten wir 
		\begin{equation}
			F(x) - G(x) = \int F'(x) - G'(x) \dx = \int_{x_0}^x h(y) \dy \tag{II}
		\end{equation}
		Addieren bzw. Subtrahieren wir Gleichungen~(I)~und~(II) so erhalten wir
		\begin{equation*}
			\begin{aligned}
				F(x) &= \frac{1}{2} \brackets{g(x) + \int_{x_0}^x h(y) \dy} \\
				G(x) &= \frac{1}{2} \brackets{g(x) - \int_{x_0}^x h(y) \dy}
			\end{aligned}
		\end{equation*}
		Setzen wir dies nun in die allgemeine Lösung von oben ein, dann erhält man mit
		\begin{align*}
			u(t,x) &= \frac{1}{2} \brackets{ g(x+t) - g(x-t) + \int_{x_0}^{x+t} h(y) \dy - \int_{x_0}^{x-t} h(y) \dy } \\
			&= \frac{1}{2} \brackets{ g(x+t) - g(x-t)} + \frac{1}{2} \int_{x-t}^{x+t} h(y) \dy
		\end{align*}
		die Formel von d'Alembert
	\end{enumerate}

\end{exercisePage}