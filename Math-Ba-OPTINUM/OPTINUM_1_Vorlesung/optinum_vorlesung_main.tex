% This work is licensed under the Creative Commons
% Attribution-NonCommercial-ShareAlike 4.0 International License. To view a copy
% of this license, visit http://creativecommons.org/licenses/by-nc-sa/4.0/ or
% send a letter to Creative Commons, PO Box 1866, Mountain View, CA 94042, USA.

% (c) Eric Kunze, 2019

%%%%%%%%%%%%%%%%%%%%%%%%%%%%%%%%%%%%%%%%%%%%%%%%%%%%%%%%%%%%%%%%%%%%%%%%%%%%
% Template for lecture notes and exercises at TU Dresden.
%%%%%%%%%%%%%%%%%%%%%%%%%%%%%%%%%%%%%%%%%%%%%%%%%%%%%%%%%%%%%%%%%%%%%%%%%%%%

\documentclass[ngerman, a4paper, 11pt]{report}

\usepackage[ngerman]{babel}
\usepackage{../../texmf/tex/latex/layoutMathTUD}
\usepackage[smallequationskip]{../../texmf/tex/latex/mathworkMathTUD}

\usepackage{../../texmf/tex/latex/mathoperatorsMathTUD}
\usepackage{optinum_theorem}

\usepackage{../../texmf/tex/latex/titlepageMathTUD}
\usepackage{../../texmf/tex/latex/graphicsMathTUD}

%%%%%%%%%%%%%%%%%%%%%%%%%%%%%%%%%%%%%%%%%%%%%%%%%%%%%%%%%%%%%%%%%%%
%                        TITLE STYLES                             %
%%%%%%%%%%%%%%%%%%%%%%%%%%%%%%%%%%%%%%%%%%%%%%%%%%%%%%%%%%%%%%%%%%%

\usepackage{titlesec}   % change title headings look
\usepackage{chngcntr}   % modify counters
\usepackage{relsize}    % relative font size (smaller[i], larger[i], ...)

\makeatletter
\@ifpackageloaded{opensans}{}{\usepackage[scale=1]{opensans}}
\ifx\textos\undefined
    \DeclareTextFontCommand{\textos}{\fosfamily}
\fi
\makeatother

\newcommand{\titlefont}{\fosfamily}
\newcommand{\chaptersize}{\huge}
\newcommand{\sectionsize}{\LARGE}

\renewcommand{\thepart}{\Alph{part}}

% \titleformat{<command>}[<shape>]{<format>}{<label>}{<sep>}{<before-code>}[<after-code>]
% \titlespacing*{<command>}{<left>}{<before-sep>}{<after-sep>}[<right-sep>]

%%%%%%%%% Kapitel * \\ Titel
%\titleformat{\chapter}[display]{\bfseries\titlefont\color{cddarkblue}}{\chaptersize\smaller \chaptername\;\thechapter}{-10pt}{\chaptersize\MakeUppercase}%
%\titlespacing{\chapter}{10pt}{0pt}{10pt}%

%%%%%%%%% like break but additionally framed
\titleformat{\chapter}[frame]{\bfseries\titlefont\color{cddarkblue}}{\enskip \chaptersize \smaller \chaptername\;\thechapter \enskip}{10pt}{\chaptersize\centering\MakeUppercase}%
\titlespacing{\chapter}{0pt}{0pt}{10pt}%


%%%%%%%%% chapter.section
\counterwithin{section}{chapter}%
\titleformat*{\section}{\bfseries\titlefont\sectionsize}%{\thesection}{8pt}{}%
\titlespacing{\section}{0pt}{10pt}{5pt}
\titleformat*{\subsection}{\bfseries\titlefont\sectionsize\smaller}
\titleformat*{\subsubsection}{\bfseries\titlefont\sectionsize\smaller\smaller}
%%%%%%%%% section.
%\renewcommand{\thechapter}{\Roman{chapter}}
%\titlelabel{\thetitle.\quad} % "." behind section/sub... (3. instead of 3)
%\counterwithout{section}{chapter}%
%\titleformat*{\section}{\bfseries\titlefont\sectionsize}%{\thesection}{8pt}{}%
%\titleformat*{\subsection}{\bfseries\titlefont\sectionsize\smaller}

%%%%%%%%% section
%\titlelabel{\thetitle \quad} % no "." behind section/sub... (3 instead of 3.)
%\titleformat{\section}[hang]{\bfseries\titlefont\sectionsize}{\thesection}{8pt}{}%
%\titleformat*{\section}{\bfseries\titlefont\sectionsize}
%\titleformat*{\subsection}{\bfseries\titlefont\sectionsize\smaller}

\input{./optinum_local_changes}

%\usepackage{../../texmf/tex/latex/referencesMathTUD}

%%%%%%%%%%%%%%%%%%%%%%%%%%%%%%%%%%%%%%%%%%%%%%%%%%%%%%%%%%%%%%%%%%%%%%%%%%%%

%---------------------------------------
% additional packages
%---------------------------------------

\usepackage{csquotes}

%---------------------------------------
% general settings
%---------------------------------------

\name{Eric Kunze}
\matnr{Nummer}
\email{\href{mailto:eric.kunze@mailbox.tu-dresden.de}{\ttfamily eric.kunze@mailbox.tu-dresden.de}}

\modul{Optimierung und Numerik}
\period{Wintersemester 2019/20}

%\renewcommand{\tutor}{Dr. Legrand}
%\renewcommand{\group}{Tag x. DS, (un)gerade Woche}

\lecturer{Dr. John Martinovic}
\faculty{Mathematik}
\institute{Numerik}
\professorship{Numerik der Optimalen Steuerung}

%%%%%%%%%%%%%%%%%%%%%%%%%%%%%%%%%%%%%%%%%%%%%%%%%%%%%%%%%%%%%%%%%%%%%%%%%%%%

\usepackage{pdfpages}

\undef\folge
\NewDocumentCommand{\folge}{m m}{\left\{ #1 \right\}_{#2}}
\renewcommand{\complement}{\mathsf{C}}

\newcommand{\widesim}[1][2.5]{
	\mathrel{\scalebox{#1}[1]{\ensuremath{\sim}}}
}

\newenvironment{indentpar}{\vspace{\parskip} \par \setlength{\parindent}{1cm}}{\vspace{\parskip} \par}

\newcolumntype{R}[1]{>{\raggedleft\arraybackslash}p{#1}}

\makeatletter
\newcommand{\leqnomode}{\tagsleft@true\let\veqno\@@leqno}
\newcommand{\reqnomode}{\tagsleft@false\let\veqno\@@eqno}
\makeatother

\DeclareMathOperator{\TP}{TP}
\DeclareMathOperator{\AZ}{AZ}
\DeclareMathOperator{\FAZ}{FAZ}
\DeclareMathOperator{\SAZ}{SAZ}
\DeclareMathOperator{\PZ}{PZ}

\usepackage{stmaryrd}

\renewcommand{\makeTUtitle}{%
	\begin{titlepage}
		\pagecolor{cddarkblue!90}%
		\color{white}%
		
		\raggedright%
		\fosfamily%
		\setlength{\parindent}{0pt}%

		
		\hspace{-18.6mm}
		\includegraphics[scale=0.6]{TUD-white.pdf}%
%		\includegraphics[scale=0.6]{TUD-blue.pdf} 
		\vspace{3mm} 
		\begin{tabular}{m{\textwidth}}
			\hline
			\hspace{-4pt}\small{\textbf{Fakultät Mathematik} Institut für Numerik, Professur für Numerik der Optimalen Steuerung} \\
			\hline
		\end{tabular} \\
	
		\vspace{5cm}
		{\Huge\bfseries \MakeUppercase Optimierung und Numerik \par}
		%
		% Untertitel
		\vspace{0.5cm}%
		{\Large \itshape Teil 1 --- Optimierung} \\%
		
		%
		\vspace{1.5cm}
		\textbf{{\Large Dr. John Martinovic}} \par
		\vspace{0.5cm}
		{\large Wintersemester 2019/20}
		
		
		
		\vfill%
		\begin{tabular}{lll}
			Autor  & : & Eric Kunze \\
			E-Mail & : & \url{eric.kunze@mailbox.tu-dresden.de} \\
		\end{tabular}%
	\end{titlepage}
	\nopagecolor
}
%%%%%%%%%%%%%%%%%%%%%%%%%%%%%%%%%%%%%%%%%%%%%%%%%%%%%%%%%%%%%%%%%%%%%%%%%%%%


\begin{document}

\makeTUtitle
    
\tableofcontents

\chapter{Einführung}
\label{chapter_1_einfuehrung}
\input{optinum_1_1_aufgabenstellungGrundbegriffe}
\section{Beispiele zur kontinuierlichen Optimierung}

\subsection{Transportoptimierung}

Die Transportoptimierung ist ein Beispiel einer linearen Optimierung.

Es gebe Erzeuger $i \in I = \menge{1, \dots , m}$ und Verbraucher $j \in J = \menge{1, \dots , n}$. Weiterhin seien die Kosten $c_{ij}$ für den Transport einer Einheit von $i$ nach $j$ sowie der Vorrat $a_i > 0$ und der Bedarf $b_j > 0$ für alle $i$ und $j$ gegeben. Wie muss der Transport organisiert werden, damit die Gesamtkosten minimal sind?

Für jedes mathematische Modell einer OA braucht man
\begin{itemize}[nolistsep, topsep=-\parskip]
	\item geeignete Variablen ($\to x$)
	\item Zielfuntkion ($\to f$)
	\item Nebenbedingungen ($\to G$)
\end{itemize}

\begin{description}
	\item[Variablen] $x_{ij} \ge 0$ für alle $i \in I$ und $j \in J$ beschreibe die Einheiten, die von $i$ nach $j$ transportiert werden.
	\item[Zielfunktion] $f(x) = \sum\limits_{i \in I} \sum\limits_{j \in J} c_{ij} x_{ij} \to \min$
	\item[Nebenbedingungen] \leavevmode
%	\begin{equation*}
%		\begin{alignedat}{2}
%		\text{Kapazitätsbeschränkung der Erzeuger: } \qquad  \sum\limits_{j \in J} x_{ij} &\le a_i & \qquad & (i \in I) \\
%		\text{Bedarfserfüllung von Verbrauchern: } \qquad \sum\limits_{i \in I} x_{ij} &\ge b_j & \qquad & (j \in J)
%		\end{alignedat}
%	\end{equation*}
	\begin{itemize}[nolistsep, topsep=-\parskip]
		\item Kapazitätsbeschränkung der Erzeuger $i \in I$: $\sum\limits_{j \in J} x_{ij} \le a_i \quad (i \in I)$
		\item Bedarfserfüllung von Verbrauchern $j \in J$: $\sum\limits_{i \in I} x_{ij} \ge b_j \quad (j \in J) $
	\end{itemize}
\end{description}

Somit können wir als Modell formulieren:
\begin{equation*}
	\begin{alignedat}{2}
	f(x) = \sum_{i \in I} \sum_{j \in J} c_{ij} x_{ij} \to \min \quad \bei \quad \sum_{j \in J} x_{ij} &\le a_i &\quad &(i \in I), \\
	\sum_{i \in I} x_{ij} &\ge b_j &\quad & (j \in J), \\
	 x_{ij} &\ge 0 & \quad & ((i,j) \in I \times J)
	\end{alignedat}
\end{equation*}

\subsection{Kürzeste euklidische Entfernung}

Die Optimierung der kürzesten euklidischen Entfernung ist nichtlinear.

Gegeben seien ein Punkt $\schlange{x} \in \Rn$ und eine Menge $G \subseteq \Rn$ mit $\schlange{x} \notin G$. Wir betrachten die folgende OA:
\begin{equation*}
	f(x) = \norm{x - \schlange{x}}_2^2 \to \min \quad \bei x \in G
\end{equation*}
Ist $G \neq \emptyset$ und abgeschlossen, so existiert eine Lösung. Ist $G$ zusätzlich konvex, so ist die Lösung sogar eindeutig.

Weitere Beispiele und Theorie sind in der Vorlesung \enquote{Kontinuierliche Optimierung} im Master Mathematik zu erfahren.
\input{optinum_1_3_beispieleDiskreteOptimierung}

\chapter{Grundlagen}
\label{chapter_2_grundlagen}
\input{optinum_2_1_existenzLoesungen}
\input{optinum_2_2_optimalitaetsbedingungen}
\input{optinum_2_3_lemmaVonFarkas}

\chapter{Lineare Optimierung}
\label{chapter_3_lineareOptimierung}
\input{optinum_3_0_intro}
\section{Basislösungen und Ecken}

Sei $I \defeq \menge{1, \dots , n}$. Da $\rg(A) = m \le n$, existiert eine Indexmenge $I_B \subseteq I$ mit $\card{I_B} = m$ derart, dass alle Spalten $A^i$ ($i \in I_B$) linear unabhängig sind. $I_B$ wird \begriff{Basis-Indexmenge} genannt. Mit $I_N \defeq I \setminus I_B$ (Nichtbasis) definieren wir
\begin{align*}
A_B &= (A^i)_{i \in I_B} & A_N &= (A^i)_{i \in I_N} \\
c_B &= (c_i)_{i \in I_B} & c_N &= (c_i)_{i \in I_N} \\
x_B &= (x_i)_{i \in I_B} & x_N &= (x_i)_{i \in I_N}
\end{align*}
Dann lässt sich \eqref{eq: 3.1} schreiben als
\begin{equation}
	z = \trans{c_B} x_B + \trans{c_N} x_N \to \min \bei A_B x_B + A_N x_N = b, x_b \ge 0, x_N \ge 0 \label{eq: 3.2}
\end{equation}
bzw. durch Auflösen der Gleichung nach $x_B$ (beachte: $A_B$ hat Vollrang) als
\begin{equation}
	z = \brackets{\trans{c_N} - \trans{c_B} A_B^{-1}A_N} x_N + \trans{c_B} A_B^{-1} b \to \min \bei x_B = -A_B^{-1} A_N x_N + A_B^{-1}b, x_B, x_N \ge 0 \label{eq: 3.3}
\end{equation}

\begin{definition} %3.1
	Der Punkt 
	\begin{equation*}
		x = \left(\begin{matrix} x_B \\ x_N \end{matrix} \right) = \left( \begin{matrix} A_B^{-1} b \\ 0 \end{matrix} \right)
	\end{equation*}
	heißt \begriff{Basislösung} zu  $I_B$. Gilt zusätzlich $A_B^{-1}b \ge 0$, dann heißt $x=(x_B,x_N)$ \begriff{zulässige Basislösung}.
\end{definition}

\begin{definition} %3.2
	Der Punkt $x \in G$ heißt \begriff{Ecke} (von $G$), falls aus $x = \frac{1}{2} \brackets{x^1 + x^2}$ mit $x^1,x^2 \in G$ stets $x = x^1 = x^2$ folgt.
\end{definition}

Ecken des zulässigen Bereichs können also nicht durch andere zulässige Punkte linear kombiniert werden.

Zur Wiederholung benennen wir im Folgenden (ohne Beweis) einige Eigenschaften von Ecken und zulässigen Basislösungen.

\begin{satz} %3.1
	Sei $\rg(A) = m$. Dann ist jede zulässige Basislösung auch Ecke von $G$. Umgekehrt gibt es zu jeder Ecke mindestens eine zulässige Basislösung.
\end{satz}

Häufig unterscheidet man zwischen 
\begin{itemize}[nolistsep, topsep=-\parskip]
	\item \begriff{degenerierten} (oder entarteten) Ecken, die mehrere zulässige Basislösungen besitzen
	\item \begriff{nicht-degenerierten} (oder nicht-entarteten) Ecken, die genau eine zulässige Basislösung besitzen.
\end{itemize}
Dabei gilt: Eine Ecke $x \in G$ ist genau dann degeneriert, wenn ein $i \in I_B$ mit $x_i=0$ existiert.

\begin{beispiel} %3.1
	Sei
	\begin{equation*}
		G \defeq \menge{x \in \R^4 \colon x_1+x_2+x_3=1, \enskip 2x_1+x_2+x_4=2, \enskip x_1,\dots , x_4 \ge 0}
	\end{equation*}
	Hierbei ist die Ecke $E_1 = \transpose{0,1,0,1}$ nicht degeneriert, da sie nur die Zerlegung $I_B = \menge{2,4}$ und $I_N = \menge{1,3}$ gestattet.
	Die Ecke $E_2 = \transpose{1,0,0,0}$ ist degeneriert, weil ein $i \in I_B$ zwangsläufig $x_i = 0$ erfüllen muss.
\end{beispiel}

\begin{satz} %3.2
	Sei $G \neq \emptyset$. Dann besitzt $G$
	\begin{enumerate}[nolistsep, topsep=-\parskip]
		\item mindestens eine Ecke
		\item höchstens endlich viele Ecken.
	\end{enumerate}
\end{satz}
\begin{proof}
	siehe Übung
\end{proof}

\begin{satz} %3.3
	Ist \eqref{eq: 3.1} lösbar, dann gibt es eine Ecke von $G$, die \eqref{eq: 3.1} löst.
\end{satz}

Bei linearen Optimierungsaufgaben genügt es daher die Ecken von $G$ zu betrachten. Ist die Aufgabe lösbar, so findet man durch systematisches Abschreiten der Ecken eine Lösung.
Um dabei zu erkennen, ob Optimalität vorliegt, hilft folgendes Resultat:

\begin{aussage}[Optimalitätskriterium] %3.4
	\label{aussage: 3.4}
	Gilt für die zuässige Basislösung $x = (x_B, x_N) = (A_B^{-1}b, 0)$ die Bedingung 
	\begin{equation}
		\trans{c_N} - \trans{c_B} A_B^{-1} A_N \ge 0 \label{eq: 3.4}
	\end{equation}
	dann ist $x$ Lösung von \eqref{eq: 3.1}.
\end{aussage}
\begin{proof}
	Sei $x = (x_B, x_N)$ eine zulässige Basislösung. Wir zeigen zunächst:
	\begin{equation*}
		Z(x) \subseteq \menge{d \in \Rn \colon Ad = 0, d_N \ge 0}
	\end{equation*}
	Sei $d \in Z(x)$. Dann existiert $t > 0$ mit $A(x+td) \overset{!}{=} b$ (beachte die Definition von $G$ mit Gleichheitsrestriktionen). Es gilt 
	\begin{equation*}
		Ax + t Ad = b  \equivalent b + t Ad = b \overset{t > 0}{\equivalent} Ad = 0
	\end{equation*}
	Wegen $x_N = 0$ ergibt sich aus $x + td \overset{!}{\ge} 0$ (nach Definition von $G$) sofort $d_N \ge 0$. Insbesondere gilt
	\begin{equation*}
		Ad = 0 \equivalent A_B d_B + A_N d_N = 0 \equivalent d_B = -A_B^{-1} A_N d_N \qquad \forall d \in Z(x)
	\end{equation*}
	Damit folgt unter Berücksichtigung von \eqref{eq: 3.4}
	\begin{align*}
		\nabla \trans{f(x)} d 
		&= \trans{c} d \\
		&= \trans{c_B} d_B + \trans{c_N} d_N \\
		&= -\trans{c_B} A_B^{-1} A_N d_N + \trans{c_N} d_N \\
		&= \underbrace{\brackets{\trans{c_N} - \trans{c_B} A_B^{-1} A_N}}_{\ge 0} \underbrace{d_N}_{\ge 0} \ge 0 \qquad \forall d \in Z(x)
	\end{align*}
	d.h. $x$ genügt der notwendigen Optimalitätsbedingung \eqref{eq: 2.2}, die hier (im konvexen Fall) auch hinreichend ist.
\end{proof}

Eine entsprechende Systematik zum Abschreiten der Ecken wird im Folgenden Abschnitt behandelt.

\input{optinum_3_2_primalesSimplexverfahren}
\input{optinum_3_3_dualesSimplexverfahren}
\input{optinum_3_4_dualitaet}
\input{optinum_3_5_transportoptimierung}

\chapter{Diskrete Optimierung}
\label{chapter_4_diskreteOptimierung}
\input{optinum_4_0_intro}
\input{optinum_4_1_spaltengenerierung}
\input{optinum_4_2_methodeBranchBound}
\input{optinum_4_3_dynamischeOptimierung}
\input{optinum_4_4_schnittebenenverfahren}
%
\chapter{Optimierung auf Graphen}
\label{chapter_5_optimierungGraphen}
\input{optinum_5_0_intro}
\input{optinum_5_1_definitionen}
\input{optinum_5_2_minimumSpanningTree}
\input{optinum_5_3_optimaleWege}
\input{optinum_5_4_maximaleFluesse}

\end{document}