\section{Dualität}

Wir betrachten nun die Optimierungsaufgabe
\begin{equation*}
	\begin{aligned}
		\qquad &\trans{c} x \to \min \bei Ax \le b \und x \in \Rn_+ \\
		&I \defeq \menge{1, \dots, m} \und J \defeq \menge{1, \dots, n}
	\end{aligned} \tag{P}
	% \label{eq: 3.11}
	\label{eq: P}
\end{equation*}

\begin{satz}[Charakterisierungssatz] %3.9
	Ein Punkt $x \in \Rn$ ist genau dann Lösung von (P), wenn ein $\quer{x} \in \Rm$ existiert, sodass insgesamt das folgende System gelöst wird:
	\begin{equation*}
		\begin{alignedat}{3}
			A \quer{x} - b &\le 0 \quad &\quer{x} &\ge 0 \qquad \qquad &(1) \\
			\trans{A} \quer{u} + c &\ge 0 \quad &\quer{u} &\ge 0 &(2) \\
			\trans{\quer{u}} \brackets{A \quer{x} - b} &= 0 \qquad \qquad &\trans{\quer{x}} \brackets{\trans{A}\quer{u} + c} &= 0 &(3)
		\end{alignedat}
	\end{equation*}
\end{satz}
\begin{proof}
	Die vorliegende Optimierungsaufgabe ist äquivalent zu
	\begin{equation*}
		f(x) = \trans{c} x \to \min \bei \underbrace{\begin{pmatrix} A \\ - \one_n \end{pmatrix}}_{= \schlange{A} \in \R^{(m + n) \times n}} x \le \underbrace{\begin{pmatrix} b \\ 0 \end{pmatrix}}_{= \schlange{b} \in \R^{m+n}}
		\tag{P'}
		\label{eq: P'}
	\end{equation*}
	Gemäß \cref{lemma: 2.8} ist $x$ genau dann Lösung von \eqref{eq: P'} (und \eqref{eq: P}), wenn ein Vektor $w = \left( \begin{smallmatrix} u \\ v \end{smallmatrix} \right) \in \R^{m + n}$ existiert mit 
	\begin{equation*}
		\begin{alignedat}{2}
			\nabla f(x) + \sum_{i \in I \cup J} w_i \schlange{a}_i &= 0 \\
			w_i &\ge 0 \qquad \qquad &&(i \in I \cup J) \\
			\trans{\schlange{a}_i} x - \schlange{b}_i &\le 0 &&(i \in I \cup J) \\
			w_i \brackets{\trans{\schlange{a}_i} x - b_i} &= 0 &&(i \in I \cup J)
		\end{alignedat}
		\tag{KKT}
	\end{equation*}
	Trennung von $I$ und $J$ führt zu
	\begin{equation*}
		c + \sum_{i \in I} u_i a_i + \sum_{j \in J} v_j (-e^j) = 0
		\tag{KKT}
	\end{equation*}
	\begin{equation*}
		\begin{alignedat}{4}
			u_i &\ge 0 \qquad &&(i \in I) 											& v_j &\ge 0 \qquad &&(j \in J) \\
			\trans{a_i} x - b_i &\le 0 &&(i \in I) 									& \transpose{-e^j} x - 0 &\le 0 &&(j \in J) \\
			u_i \brackets{\trans{a_i} x - b_i} &= 0 &&(i \in I) \qquad\quad	& v_j \brackets{\transpose{-e^j} x - 0} &= 0 &&(j \in J)  \\
		\end{alignedat}
	\end{equation*}
	Überführt man dieses System in eine Matrix-Vektor-Schreibweise, so ergibt sich
	\begin{equation*}
		\begin{aligned}
			c + \trans{A} u - v &= 0 \\
			u,v,x &\ge 0 \\
			Ax - b &\le 0 \\
			\trans{u} \brackets{Ax - b} &= 0 \\
			\trans{x} v &= 0
		\end{aligned}
	\end{equation*}
	Durch Umstellen der ersten Gleichung nach $v$ lässt sich diese Variable im System ''eliminieren`` und wir erhalten die Behauptung.
\end{proof}