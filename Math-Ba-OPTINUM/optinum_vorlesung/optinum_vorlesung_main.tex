% This work is licensed under the Creative Commons
% Attribution-NonCommercial-ShareAlike 4.0 International License. To view a copy
% of this license, visit http://creativecommons.org/licenses/by-nc-sa/4.0/ or
% send a letter to Creative Commons, PO Box 1866, Mountain View, CA 94042, USA.

% (c) Eric Kunze, 2019

%%%%%%%%%%%%%%%%%%%%%%%%%%%%%%%%%%%%%%%%%%%%%%%%%%%%%%%%%%%%%%%%%%%%%%%%%%%%
% Template for lecture notes and exercises at TU Dresden.
%%%%%%%%%%%%%%%%%%%%%%%%%%%%%%%%%%%%%%%%%%%%%%%%%%%%%%%%%%%%%%%%%%%%%%%%%%%%

\documentclass[ngerman, a4paper, 11pt]{report}

\usepackage[ngerman]{babel}
\usepackage{../../texmf/tex/latex/layoutMathTUD}
\usepackage[smallequationskip]{../../texmf/tex/latex/mathworkMathTUD}

\usepackage{../../texmf/tex/latex/mathoperatorsMathTUD}
\usepackage[includechapter]{../../texmf/tex/latex/maththeorems3MathTUD}

\usepackage{../../texmf/tex/latex/titlepageMathTUD}
\usepackage{../../texmf/tex/latex/graphicsMathTUD}

\input{./optinum_local_headings}
\input{./optinum_local_changes}

\usepackage{../../texmf/tex/latex/referencesMathTUD}

%%%%%%%%%%%%%%%%%%%%%%%%%%%%%%%%%%%%%%%%%%%%%%%%%%%%%%%%%%%%%%%%%%%%%%%%%%%%

%---------------------------------------
% additional packages
%---------------------------------------

% none

%---------------------------------------
% general settings
%---------------------------------------

\name{Eric Kunze}
\matnr{Nummer}
\email{\href{mailto:eric.kunze@mailbox.tu-dresden.de}{\ttfamily eric.kunze@mailbox.tu-dresden.de}}

\modul{Optimierung und Numerik}
\period{Wintersemester 2019/20}

%\renewcommand{\tutor}{Dr. Legrand}
%\renewcommand{\group}{Tag x. DS, (un)gerade Woche}

\lecturer{Dr. John Martinovic}
\faculty{Mathematik}
\institute{Numerik}
\professorship{Numerik der Optimalen Steuerung}

%%%%%%%%%%%%%%%%%%%%%%%%%%%%%%%%%%%%%%%%%%%%%%%%%%%%%%%%%%%%%%%%%%%%%%%%%%%%



\undef\folge
\NewDocumentCommand{\folge}{m O{n \in \N}}{\left( #1 \right)_{#2}}
\renewcommand{\complement}{\mathsf{C}}

\newcommand{\widesim}[1][2.5]{
	\mathrel{\scalebox{#1}[1]{\ensuremath{\sim}}}
}


%%%%%%%%%%%%%%%%%%%%%%%%%%%%%%%%%%%%%%%%%%%%%%%%%%%%%%%%%%%%%%%%%%%%%%%%%%%%


\begin{document}

\makeTUtitle
    
\tableofcontents

\chapter{Einführung}
\label{chapter_1_einfuehrung}
\section{Aufgabenstellung und Grundbegriffe}

Es seien $G \subseteq \Rn$ und $\abb{f}{G}{\R}$ gegeben. In dieser Vorlesung betrachten wir Optimierungsaufgaben (OA) der Form
\begin{equation}\label{eq: oa}
	f(x) \to \min \quad \bei x \in G
\end{equation}
Man nennt
\begin{itemize}[nolistsep, topsep=-\parskip]
	\item $f$ die \begriff{Zielfunktion},
	\item $G$ den \begriff{zulässigen Bereich} und
	\item ein $x \in  G$ \begriff{zulässigen Punkt} (oder zulässige Lösung).
\end{itemize}
Ein zulässiger Punkt $x^\ast \in G$ heißt \begriff{optimal} (oder Lösung oder optimale Lösung), wenn für alle $x \in G$ die Ungleichung
\begin{equation} \label{eq: 1_2_optimal}
	f(x^\ast) \le f(x)
\end{equation}
gilt. Falls das Problem \eqref{eq: oa} lösbar ist, so wird mit $f^\ast = f(x^\ast)$ der \begriff{Optimalwert} bezeichnet. Das Problem \eqref{eq: oa} ist ein
\begin{itemize}[nolistsep, topsep=-\parskip]
	\item \begriff{unrestringiertes} (oder freies) Optimierungsproblem, wenn $G = \Rn$ gilt,
	\item andernfalls (d.h. für $G \neq \Rn$) ein \begriff{restringiertes} Problem
\end{itemize}
und außerdem eine
\begin{itemize}[nolistsep, topsep=-\parskip]
	\item \begriff{diskrete} (oder ganzzahlige) OA (engl. integer program), falls jede Variable eine diskreten Menge angehört
	\item \begriff{kontinuierliche} (oder stetige) OA, falls alle Variablen stetige Werte annehmen
	\item gemischt ganzzahlige OA, wenn sowohl stetige als auch diskrete Variablen vorkommen.
\end{itemize}

Gilt in \eqref{eq: oa} $f(x) = \trans{c} x$ für ein $c \in \Rn$ und ist $G$ durch lineare Bedingungen beschreibbar, so heißt \eqref{eq: oa} \begriff{linear}. In diesem Fall lässt sich \eqref{eq: oa} schreiben als
\begin{equation} \label{eq: 1_3_lineareOA}
	\trans{c} x \to \min \quad \bei Ax = a, Bx \le b
\end{equation}
mit geeigneten Matrizen $A$ und $B$ sowie Vektoren $a$ und $b$.

Gerade für (gemischt) ganzzahlige OA kann die Lösung der Originalaufgabe schwierig sein. Eine verwandte, jedoch im Allgemeinen leichter zu lösende Aufgabe kann  in diesen Fällen wie folgt erhalten werden:

\begin{definition}
	Wir betrachten die Optimierungsaufgaben
	\begin{itemize}[nolistsep, topsep=-\parskip]
		\item[(P)] $f(x) \to \min \quad \bei x \in D \cap E$
		\item[(Q)] $g(x) \to \min \quad \bei x \in E$
	\end{itemize}
	(Q) heißt \begriff{Relaxation} zu (P) falls $g(x) \le f(x)$ für alle $x \in D \cap E$ gilt. In vielen Fällen wird dabei $g = f$ gewählt.
\end{definition}

Der Optimalwert der Relaxation kann als Näherung (bzw. untere Schranke) für den tatsächlichen Optimalwert von (P) genutzt werden. Meistens liefert die Lösung von (Q) jedoch keinen zulässigen Puntk für (P).

\begin{satz}
	Ist $\quer{x}$ eine Lösung von (Q) und gilt $\quer{x} \in D$ sowie $f(\quer{x}) = g(\quer{x})$, dann löst $\quer{x}$ auch (P).
\end{satz}
\begin{proof}
	siehe Übung
\end{proof}

\begin{definition}
	Seien (Q1) und (Q2) Relaxationen zu (P). (Q1) heißt \begriff{stärker} (oder strenger) als (Q2), wenn die Schranke (d.h. der Optimalwert) von (Q1) größer oder gleich der Schranke (Optimalwert) von (Q2) für jede Instanz von (P) ist.
\end{definition}

\begin{*anmerkung}
	Zur Erklärung des Begriffes ''Instanz`` betrachte das folgende Beispiel.
	\begin{itemize}[nolistsep, topsep=-\parskip]
		\item Problemklasse: $\trans{c} x \to \min$
		\item Instanz der Problemklasse: $x_1 + 2x_2 - 3x_3 \to \min$
	\end{itemize}
	Eine Instanz ist also eine konkrete Belegung.
\end{*anmerkung}
\section{Beispiele zur kontinuierlichen Optimierung}

\subsection{Transportoptimierung}

Die Transportoptimierung ist ein Beispiel einer linearen Optimierung.

Es gebe Erzeuger $i \in I = \menge{1, \dots , n}$ und Verbraucher $j \in J = \menge{1, \dots , n}$. Weiterhin seien die Kosten $c_{ij}$ für den Transport einer Einheit von $i$ nach $j$ sowie der Vorrat $a_i > 0$ und der Bedarf $b_j > 0$ für alle $i$ und $j$ gegeben. Wie muss der Transport organisiert werden, damit die Gesamtkosten minimal sind?

Für jedes mathematische Modell einer OA braucht man
\begin{itemize}[nolistsep, topsep=-\parskip]
	\item geeignete Variablen ($\to x$)
	\item Zielfuntkion ($\to f$)
	\item Nebenbedingungen ($\to G$)
\end{itemize}

\begin{description}
	\item[Variablen] $x_{ij} \ge 0$ für alle $i \in I$ und $j \in J$ beschreibe die Einheiten, die von $i$ nach $j$ transportiert werden.
	\item[Zielfunktion] $f(x) = \sum\limits_{i \in I} \sum\limits_{j \in J} c_{ij} x_{ij} \to \min$
	\item[Nebenbedingungen] \leavevmode
%	\begin{equation*}
%		\begin{alignedat}{2}
%		\text{Kapazitätsbeschränkung der Erzeuger: } \qquad  \sum\limits_{j \in J} x_{ij} &\le a_i & \qquad & (i \in I) \\
%		\text{Bedarfserfüllung von Verbrauchern: } \qquad \sum\limits_{i \in I} x_{ij} &\ge b_j & \qquad & (j \in J)
%		\end{alignedat}
%	\end{equation*}
	\begin{itemize}[nolistsep, topsep=-\parskip]
		\item Kapazitätsbeschränkung der Erzeuger $i \in I$: $\sum\limits_{j \in J} x_{ij} \le a_i \quad (i \in I)$
		\item Bedarfserfüllung von Verbrauchern $j \in J$: $\sum\limits_{i \in I} x_{ij} \ge b_j \quad (j \in J) $
	\end{itemize}
\end{description}

Somit können wir als Modell formulieren:
\begin{equation*}
	\begin{alignedat}{2}
	f(x) = \sum_{i \in I} \sum_{j \in J} c_{ij} x_{ij} \to \min \quad \bei \quad \sum_{j \in J} x_{ij} &\le a_i &\quad &(i \in I), \\
	\sum_{i \in I} x_{ij} &\ge b_j &\quad & (j \in J), \\
	 x_{ij} &\ge 0 & \quad & ((i,j) \in I \times J)
	\end{alignedat}
\end{equation*}

\subsection{Kürzeste euklidische Entfernung}

Die Optimierung der kürzesten euklidischen Entfernung ist nichtlinear.

Gegeben seien ein Punkt $\schlange{x} \in \Rn$ und eine Menge $G \subseteq \Rn$ mit $\schlange{x} \notin G$. Wir betrachten die folgende OA:
\begin{equation*}
	f(x) = \norm{x - \schlange{x}}_2^2 \to \min \quad \bei x \in G
\end{equation*}
Ist $G \neq \emptyset$ und abgeschlossen, so existiert eine Lösung. Ist $G$ zusätzlich konvex, so ist die Lösung sogar eindeutig.

Weitere Beispiele und Theorie sind in der Vorlesung \enquote{Kontinuierliche Optimierung} im Master Mathematik zu erfahren.


\end{document}