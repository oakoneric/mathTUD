\documentclass[german]{scrreprt}

\usepackage[top=2.5cm,bottom=2.5cm,left=2.5cm,right=2.5cm]{geometry}

\usepackage[utf8]{inputenc}
\usepackage[T1]{fontenc}
%\setlength{\parindent}{0pt}
\usepackage{parskip}

\usepackage{lmodern}
\usepackage{color}
\usepackage[ngerman]{babel}
\usepackage{csquotes}

\usepackage[scale=0.95]{opensans}	% new font OpenSans
\newcommand*{\osfamily}{\fontfamily{fos}\selectfont}
\DeclareTextFontCommand{\textos}{\fosfamily}

\let\emph\textbf
\newcommand{\begriff}[1]{\textbf{#1}}

\addtokomafont{disposition}{\fosfamily}
\addtokomafont{chapter}{\Huge}
\addtokomafont{section}{\LARGE}
\addtokomafont{subsection}{\Large}

\usepackage{scrlayer-scrpage}
\usepackage[linesnumbered,inoutnumbered]{algorithm2e}
%\usepackage[defernumbers=true,
%            maxbibnames=99,  % Never write 'et al.' in a bibliography
%            giveninits=true, % Abbreviate first names
%            sortcites=true,  % Sort citations when there are more than one
%            natbib=true,     % Allow commands like \citet and \citep
%            backend=biber]{biblatex}
%\addbibresource{skript-numerik-sander.bib}

\usepackage[makeroom]{cancel}


\usepackage{enumerate}
\usepackage[inline]{enumitem} 		%customize label

\renewcommand{\labelitemi}{\raisebox{1pt}{\scalebox{.5}{$\blacksquare$}}}
\renewcommand{\labelitemii}{$\vartriangleright$}
\renewcommand{\labelitemiii}{--}
% Variantionen des Dreiecks als Aufzählungszeichen $\blacktriangleright$ / $\vartriangleright$ / $\triangleright$

%\renewcommand{\labelenumi}{(\alph{enumi})}
%\renewcommand{\labelenumii}{\roman{enumii}.}
%\renewcommand{\labelenumiii}{\roman{enumiii}.}



\usepackage{microtype}
\usepackage{todonotes}
\usepackage{pdfcomment}  % todonotes without tikz -- the two don't play well together
\newcommand{\todoannot}[2]{
  \pdffreetextcomment[width=\textwidth,height=#1,color=orange]{#2}
}

\usepackage{xcolor}
\usepackage[table,dvipsnames]{tudscrcolor}

\usepackage{esdiff}
\usepackage{url}
\usepackage{overpic}
\usepackage{hyperref}
\usepackage{array}
\usepackage{tikz}
\usetikzlibrary{arrows,calc,cd,decorations.pathmorphing,matrix,patterns,positioning,shapes}
\usepackage{graphicx}
\usepackage{float}


\usepackage{pgfplots}
\usepgfplotslibrary{fillbetween,colormaps}
\pgfplotsset{compat=newest}


\usepackage[ntheorem,framemethod=TikZ]{mdframed}
\usepackage{amsmath,amssymb,amsfonts,mathtools}
\usepackage[amsmath,thmmarks,framed]{ntheorem}
\usepackage{colonequals}
\usepackage{wasysym}

% allow to smash underbraces:
% use \smashedUnderbrace instead of \underbrace
% surround formula with \reboxSmashedUnderbraces
% https://web.archive.org/web/20160820203342/http://tex.stackexchange.com/questions/38930/ignoring-vertical-space-due-to-underbraces-in-cases-environment
\newcommand\realSmashC[1]{\smash{\mathclap{#1}}}
\newcommand\realSmashR[1]{\smash{\mathrlap{#1}}}
\newcommand\smashedUnderbrace[2]{\underbrace{#1}_{#2}}
\newcommand\actualSmashUnderbrace[2]{%
	\realSmashR{\underbrace{#1}_{\realSmashC{#2}}} \phantom{#1}}
\newcommand\reboxSmashedUnderbraces[1]{%
	{\renewcommand{\smashedUnderbrace}{\actualSmashUnderbrace}#1}%
	\vphantom{#1}}

\allowdisplaybreaks[0]


\newcommand{\R}{\mathbb R}
\newcommand{\N}{\mathbb N}
\newcommand{\K}{\mathbb K}
\renewcommand{\C}{\mathbb C}
\newcommand{\D}{\mathcal{D}}
\newcommand{\spann}{\operatorname{span}}
\newcommand{\extendadd}{\;\; \updownarrow \hspace{-12pt} \longleftrightarrow}
\newcommand{\qq}[1]{\glqq #1\grqq}

\renewcommand{\Re}{\operatorname{Re}}
\renewcommand{\Im}{\operatorname{Im}}

\newcommand{\abs}[1]{\lvert #1 \rvert}
\newcommand{\norm}[1]{\lVert #1 \rVert}

%\newcommand{\diff}{\mathrm{d}}
\newcommand{\diffskip}[1]{\enskip \mathrm{d}#1}
\newcommand{\dx}{\diffskip{x}}
\newcommand{\dt}{\diffskip{t}}


\newcommand{\skiparound}{10pt}

\theoremstyle{plain}
\theoremheaderfont{\osfamily\normalsize\bfseries\upshape}
\theorembodyfont{\normalsize}
\theoremseparator{.}
\theoremsymbol{}


\newmdtheoremenv[%
	%style=boxedtheorem,%
	backgroundcolor=cdblue!10,%
	linecolor=cddarkblue!90,%
	%innertopmargin=\boxskip,%
	%innerbottommargin=\boxskip,%
	innerleftmargin=4pt,%
	outerlinewidth=5pt,%
	topline=false,%
	rightline=true,%
	leftline=false,%
	bottomline=false,%
	%innertopmargin=3pt,%
	%innerbottommargin=3pt,%
	leftmargin=-4pt,%
	%rightmargin=-10pt,%
	skipabove=\skiparound,%
	skipbelow=\skiparound,%
	nobreak,%
]{satz}{Satz}

\newmdtheoremenv[%
	%style=boxedtheorem,%
	backgroundcolor=cdindigo!5,%
	linecolor=cdindigo!50,%
	%innertopmargin=\boxskip,%
	%innerbottommargin=\boxskip,%
	innerleftmargin=4pt,%
	outerlinewidth=5pt,%
	topline=false,%
	rightline=true,%
	leftline=false,%
	bottomline=false,%
	%innertopmargin=3pt,%
	%innerbottommargin=3pt,%
	leftmargin=-4pt,%
	%rightmargin=-10pt,%
	%frametitlefont=\osfamily\bfseries,%
	%skipabove=5pt,%
	%skipbelow=5pt,%
	skipabove=\skiparound,%
	skipbelow=\skiparound,%
	nobreak,%
]{lemma}{Lemma}


\theoremstyle{nonumberplain}

\newmdtheoremenv[%
	%style=boxedtheorem,%
	backgroundcolor=cdorange!10,%
	linecolor=cdorange,%
	%innertopmargin=\boxskip,%
	%innerbottommargin=\boxskip,%
	innerleftmargin=4pt,%
	outerlinewidth=5pt,%
	topline=false,%
	rightline=true,%
	leftline=false,%
	bottomline=false,%
	%innertopmargin=3pt,%
	%innerbottommargin=3pt,%
	leftmargin=-4pt,%
	%rightmargin=-10pt,%
	%frametitlefont=\osfamily\bfseries,%
	%skipabove=5pt,%
	%skipbelow=5pt,%
	skipabove=\skiparound,%
	skipbelow=\skiparound,%
	nobreak,%
]{defi}{Definition}

\newmdtheoremenv[%
	%style=boxedtheorem,%
	backgroundcolor=cdorange!10,%
	linecolor=cdorange,%
	%innertopmargin=\boxskip,%
	%innerbottommargin=\boxskip,%
	innerleftmargin=4pt,%
	outerlinewidth=5pt,%
	topline=false,%
	rightline=true,%
	leftline=false,%
	bottomline=false,%
	%innertopmargin=3pt,%
	%innerbottommargin=3pt,%
	leftmargin=-4pt,%
	%rightmargin=-10pt,%
	%frametitlefont=\osfamily\bfseries,%
	%skipabove=5pt,%
	%skipbelow=5pt,%
	skipabove=\skiparound,%
	skipbelow=\skiparound,%
	nobreak,%
]{definition}{Definition}


\theoremstyle{nonumberplain}
\theorempreskip{\skiparound}
\theorempostskip{\skiparound}

\newtheorem{bsp}{Beispiel}
\newtheorem{algo}{Algorithmus}
\newtheorem{folg}{Folgerung}
\newtheorem{kor}{Korollar}

\theoremheaderfont{\normalfont \bfseries}
\theorembodyfont{\itshape}
\newtheorem{bem}{Bemerkung}
\newtheorem{hinw}{Hinweis}

\theoremheaderfont{\normalfont \bfseries}
\newtheorem{idea}{Idee}
\newtheorem{aim}{Ziel}
\newtheorem{question}{Frage}

\theorembodyfont{}
\newtheorem{aufw}{Aufwand}

\newtheorem{vSfS}{vSfS}

%\newtheoremstyle{proofstyle}%
%{\item[\hskip\labelsep {\theorem@headerfont ##1}\theorem@separator]}%
%{\item[\hskip\labelsep {\theorem@headerfont ##1}\ (##3)\theorem@separator]}

\theoremstyle{nonumberplain}
\theoremheaderfont{\normalsize\slshape}
\theorembodyfont{}
\theoremseparator{.}
\theorempreskip{0pt}
\theorempostskip{5pt}
\theoremsymbol{$\square$}
\newtheorem{proof}{Beweis}

\usepackage{chngcntr}
\counterwithin{satz}{chapter}
\counterwithin{lemma}{chapter}

%\pagestyle{scrheadings}
%\clearscrheadings
%\ihead{Mathematik}
%\chead{\textbf{Numerik}}
%\ohead{Prof.~Dr.~O.\,Sander}
%\setheadsepline{0.4pt}
\cfoot{\pagemark}

%%  All graphics files may be in this subdirectory
\graphicspath{{gfx/}}
%\includeonly{numerische-quadratur}

\usepackage{accents}
\renewcommand*{\dot}[1]{\accentset{\mbox{\large\bfseries .}}{#1}}
\renewcommand*{\ddot}[1]{\accentset{\mbox{\large\bfseries .\hspace{-0.25ex}.}}{#1}}

\begin{document}

%\frontmatter

%\begin{titlepage}
%	\pagecolor{cddarkblue!90} \color{white}%
%	\raggedright \fosfamily%
%	\setlength{\parindent}{0pt}% 
%	% Logo / Kopf
%	\hspace{-18.6mm} %
%	\includegraphics[scale=0.6]{TUD-white.pdf} \\
%	\vspace{3mm} 
%	\begin{tabular}{m{\textwidth}}
%		\hline
%		\hspace{-4pt}\small{\textbf{Fakultät Mathematik} Institut für Numerik, Professur für Numerik partieller Differentialgleichungen} \\
%		\hline
%	\end{tabular} \\
%	% Titel
%	\vspace{5cm}
%	{\Huge\bfseries \MakeUppercase Optimierung \& Numerik --- Teil 2 \par}
%		\vspace{0.5cm}%
%		{\Large \itshape Numerik gewöhnlicher Differentialgleichungen} \\%
%	\vspace{1.5cm}
%	\textbf{{\Large Prof. Dr. Oliver Sander}} \par
%	\vspace{0.5cm}
%	{\large Sommersemester 2020} \par
%	
%	\vspace{1.5cm}
%	
%	
%	\vfill%
%	% Fußzeile
%	\begin{tabular}{lll}
%		Name  & : & Eric Kunze \\
%		E-Mail & : & \url{eric.kunze@mailbox.tu-dresden.de} \\
%		Datum & : & \today
%	\end{tabular}%
%\end{titlepage}
%\nopagecolor
%
%%%%%%%%%%%%%%%%%%%%%%%%%%%%%%%%%%%%%%%%%%%%%%%%%%%%%%%%%%%%%%%%%%%%%%%%%%%%%%%%%
%%%  Copyright notice
%%%%%%%%%%%%%%%%%%%%%%%%%%%%%%%%%%%%%%%%%%%%%%%%%%%%%%%%%%%%%%%%%%%%%%%%%%%%%%%%%
%
%\vspace*{\fill}
%
%\begin{center}
%	\slshape
%	Das vorliegende Skript basiert nahezu vollständig auf dem Vorlesungsskript von Prof. Dr. Oliver Sander für die Vorlesung \enquote{Optimierung und Numerik} an der Technischen Universität Dresden. Das aktuelle Original ist unter \\ \url{https://gitlab.mn.tu-dresden.de/osander/skript-numerik} \\
%	zu finden.
%\end{center}
%
%\vspace*{\fill}
%
%\pagebreak


%%%%%%%%%%%%%%%%%%%%%%%%%%%%%%%%%%%%%%%%%%%%%%%%%%%%%%%%%%%%%%%%%%%%%%%%%%%

\renewcommand*\contentsname{\huge \centering Optimierung \& Numerik --- Vorlesung 12}
\tableofcontents

%%%%%%%%%%%%%%%%%%%%%%%%%%%%%%%%%%%%%%%%%%%%%%%%%%%%%%%%%%%%%%%%%%%%%%%%%%%

\setcounter{chapter}{12}
\setcounter{section}{0}
\setcounter{satz}{0}
\setcounter{lemma}{0}
\setcounter{equation}{0}


\section{Hamilton-Systeme}

Extrem wichtige Klasse von Differentialgleichungen entstammen der klassischen Mechanik, Quantenmechanik und relativistische Mechanik. Dazu gehören auch spezielle numerische Verfahren --- eine \enquote{schöne Mathematik}.

\enquote{Vereinigendes Prinzip}: Bringt ganz unterschiedliche Gleichungen auf eine gemeinsame Form.


\begin{bsp}
	Mathematisches Pendel -- Fadenpendel
	\begin{itemize}
		\item Koordinate: Winkel $\alpha$
		\item Masse $m$, Fadenlänge $l$, Erdbeschleunigung $g$
	\end{itemize}
	Bewegungsgleichungen: $\displaystyle \ddot \alpha + \frac{g}{l} \sin \alpha = 0$
\end{bsp}

\begin{bsp}
	Teilchen in einem Kraftfeld $F(x)$
	\begin{equation}
		m \ddot x = F(x) \tag{Newtons Gesetz}
	\end{equation}
\end{bsp}

\begin{bsp}
	1d-Wellengleichung --- Longitudinale Auslenkung einer elastischen Schnur
	\begin{alignat*}{2}
		\frac{\partial^2 u}{\partial x^2} & = \frac{\partial^2 u}{\partial t^2} & \qquad & x \in [a, b], t \ge 0 \\
		u(a, t) = u(b, t) & = 0 && \forall t \ge 0
	\end{alignat*}
\end{bsp}


\subsection{Die Lagrange-Gleichungen}
\label{sec:lagrange_gleichung}

Wir betrachten ein mechanisches System mit $d$ Freiheitsgraden $q = (q_1, \dots, q_d)$.
\begin{itemize}
	\item Kinetische Energie: $T = T(q, \dot q)$
	(häufig: $T(q, \dot{q}) = \frac12 \dot{q}^T M(q) \dot{q}$ mit $M(q)$ s.p.d.)
	\item Potentielle Energie: $U = U(q)$
\end{itemize}

\begin{definition}
  Die Lagrange-Funktion eines mechanischen Systems ist $L = T - U$.
\end{definition}

Das mechanische System löst die Lagrange-Gleichungen
\begin{equation*}
  \frac{d}{dt} \left( \frac{\partial L}{\partial \dot q} \right) = \frac{\partial L}{\partial q}
\end{equation*}

\emph{Warum?} Es gilt das Prinzip der stationären Wirkung.

\begin{definition}[Prinzip der stationären Wirkung / Hamilton'sches Prinzip]
	Sei $q: [t_0, t_1] \to \R^d$ eine Trajektorie eines mechanischen Systems.
	Für die in der Natur vorkommenden Trajektorien ist die \textit{Wirkung}
	\begin{equation*}
		W \colonequals \int_{t_0}^{t_1} L(q(t) \ \dot q(t)) \dt
	\end{equation*}
	stationär.
\end{definition}

Sei $q$ eine Trajektorie, und $\delta q$ eine Variation davon, die die Endpunkte fest lässt, also $\delta q(t_0)=\delta q(t_1) = 0$.
Stationarität von $q$ heißt dann, dass für alle solche $\delta q$
\begin{equation*}
	\frac{d}{d\epsilon} S(q+\epsilon\delta q)\vert_{\epsilon=0} = 0.
\end{equation*}
gilt. Ausrechnen:
\begin{align*}
	\frac{d}{d\epsilon} \int_{t_0}^{t_1} L(q+ \epsilon\delta q,\dot q + \epsilon\delta \dot q)\dt\vert_{\epsilon=0} &= \int_{t_0}^{t_1} \frac{\partial L}{\partial q}\delta q + \frac{\partial L}{\partial\dot q} \delta\dot q\, dt \\
	&= \int_{t_0}^{t_1} \Bigg(\frac{\partial L}{\partial q} \delta q - \frac{d}{dt} \frac{\partial L}{\partial \dot q} \delta q\Bigg)\dt \tag{partielle Integration} \\
	&= \int_{t_0}^{t_1} \Bigg(\frac{\partial L}{\partial q} - \frac{d}{dt} \frac{\partial L}{\partial \dot q}  \Bigg)\delta q\dt.
\end{align*}
Da dieser Ausdruck für alle hinreichend glatten Funktionen $\delta q$ gleich Null sein muss, erhält man die Lagrange-Gleichung
\begin{equation*}
  \delta W = 0 = \frac{d}{dt} \left( \frac{\partial L}{\partial \dot q} \right) - \frac{\partial L}{\partial q}
\end{equation*}

\begin{bsp}[Pendel]
	\begin{itemize}
		\item Kinetische Energie
		\begin{equation*}
		T = \frac12 m (\dot x^2 + \dot y^2) = \frac12 m l^2 \dot\alpha^2
		\end{equation*}
		\item Potentielle Energie
		\begin{equation*}
		U = mgy = -mgl \cos \alpha
		\end{equation*}
		\item Lagrange-Funktion
		\begin{equation*}
		L(\alpha, \dot\alpha) = \frac12 m l^2 \dot\alpha^2 + mgl \cos \alpha
		\end{equation*}
		\item Lagrange-Gleichung
		\begin{equation*}
		0
		= \frac{d}{dt} \left( \frac{\partial L}{\partial \dot q} \right) - \frac{\partial L}{\partial q}
		= \frac{d}{dt}( ml^2\dot\alpha ) + mgl \sin \alpha
		= ml^2\ddot\alpha + mgl \sin \alpha.
		\end{equation*}
	\end{itemize}
\end{bsp}

\begin{bsp}[Teilchen in einem Kraftfeld]
	Angenommen das Kraftfeld ist \emph{konservativ}, d.h. es gibt ein $U: \R^3 \to \R$, so dass $F(x) = -\nabla U(x)$.
	\begin{itemize}
		\item Kinetische Energie
		\begin{equation*}
			T(x, \dot x) = \frac12 m \langle \dot x, \dot x \rangle
		\end{equation*}
		\item Potentielle Energie
		\begin{equation*}
			U
		\end{equation*}
		\item Lagrange-Gleichung
		\begin{equation*}
			0
			= \frac{d}{dt} \left( \frac{\partial L}{\partial \dot q} \right) - \frac{\partial L}{\partial q}
			= \frac{d}{dt}(m\dot x) + \nabla U(x) = m\ddot x - F(x)
		\end{equation*}
	\end{itemize}
\end{bsp}

\begin{bsp}[Eindimensionale Wellengleichung]
	Ein unendlich-dimensionales System wird nicht beschrieben durch $d$
	Freiheitsgrade $(q_1, \dots, q_d)$, sondern durch die Funktion $u: [a, b] \to \R$. Diese beschreibt die transversale Auslenkung einer Saite.
	\begin{itemize}
		\item Kinetische Energie
		\begin{equation*}
			T(u, \dot u) = \frac12 \int_a^b m \dot u(x)^2 dx
		\end{equation*}
		Dabei ist $m$ die Massendichte.
		\item Potentielle Energie
		\begin{equation*}
			U(u)
			=
			\int_a^b S \Big[ \sqrt{1+ u'(x)^2} -1 \Big] dx
			\approx
			\int_a^b S\frac{u'(x)^2}{2} dx
		\end{equation*}
		Dabei ist $S$ die Zugsteifigkeit.
		\item Lagrange-Funktion
		\begin{equation*}
			L(u, \dot u) = T(u, \dot u) - U(u)
		\end{equation*}
		\item Lagrange-Gleichung
		\begin{equation*}
			\frac{\partial L}{\partial u}
			=
			\frac{\partial}{\partial x} \frac{\partial L}{\partial u'}
			+ \frac{\partial}{\partial t} \frac{\partial L}{\partial \dot{u}}
		\end{equation*}
		Einsetzen:
		\begin{equation*}
			0 =
			\frac{\partial}{\partial x} \Big(- S u'(x)\Big)
			+ \frac{\partial}{\partial t}m\dot{u}
		\end{equation*}
		Umstellen:
		\begin{equation*}
			\frac{\partial^2 u}{\partial t^2} = \frac{S}{m} \frac{\partial^2 u}{\partial x^2}
		\end{equation*}
		Das ist die eindimensionale Wellengleichung.
	\end{itemize}
\end{bsp}



\bibliographystyle{alpha}
\bibliography{skript-numerik-sander}

\end{document}