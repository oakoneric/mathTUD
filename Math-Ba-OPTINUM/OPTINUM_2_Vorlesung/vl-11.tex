\documentclass[german]{scrreprt}

\usepackage[top=2.5cm,bottom=2.5cm,left=2.5cm,right=2.5cm]{geometry}

\usepackage[utf8]{inputenc}
\usepackage[T1]{fontenc}
%\setlength{\parindent}{0pt}
\usepackage{parskip}

\usepackage{lmodern}
\usepackage{color}
\usepackage[ngerman]{babel}
\usepackage{csquotes}

\usepackage[scale=0.95]{opensans}	% new font OpenSans
\newcommand*{\osfamily}{\fontfamily{fos}\selectfont}
\DeclareTextFontCommand{\textos}{\fosfamily}

\let\emph\textbf
\newcommand{\begriff}[1]{\textbf{#1}}

\addtokomafont{disposition}{\fosfamily}
\addtokomafont{chapter}{\Huge}
\addtokomafont{section}{\LARGE}
\addtokomafont{subsection}{\Large}

\usepackage{scrlayer-scrpage}
\usepackage[linesnumbered,inoutnumbered]{algorithm2e}
%\usepackage[defernumbers=true,
%            maxbibnames=99,  % Never write 'et al.' in a bibliography
%            giveninits=true, % Abbreviate first names
%            sortcites=true,  % Sort citations when there are more than one
%            natbib=true,     % Allow commands like \citet and \citep
%            backend=biber]{biblatex}
%\addbibresource{skript-numerik-sander.bib}

\usepackage[makeroom]{cancel}


\usepackage{enumerate}
\usepackage[inline]{enumitem} 		%customize label

\renewcommand{\labelitemi}{\raisebox{1pt}{\scalebox{.5}{$\blacksquare$}}}
\renewcommand{\labelitemii}{$\vartriangleright$}
\renewcommand{\labelitemiii}{--}
% Variantionen des Dreiecks als Aufzählungszeichen $\blacktriangleright$ / $\vartriangleright$ / $\triangleright$

%\renewcommand{\labelenumi}{(\alph{enumi})}
%\renewcommand{\labelenumii}{\roman{enumii}.}
%\renewcommand{\labelenumiii}{\roman{enumiii}.}



\usepackage{microtype}
\usepackage{todonotes}
\usepackage{pdfcomment}  % todonotes without tikz -- the two don't play well together
\newcommand{\todoannot}[2]{
  \pdffreetextcomment[width=\textwidth,height=#1,color=orange]{#2}
}

\usepackage{xcolor}
\usepackage[table,dvipsnames]{tudscrcolor}

\usepackage{esdiff}
\usepackage{url}
\usepackage{overpic}
\usepackage{hyperref}
\usepackage{array}
\usepackage{tikz}
\usetikzlibrary{arrows,calc,cd,decorations.pathmorphing,matrix,patterns,positioning,shapes}
\usepackage{graphicx}
\usepackage{float}


\usepackage{pgfplots}
\usepgfplotslibrary{fillbetween,colormaps}
\pgfplotsset{compat=newest}


\usepackage[ntheorem,framemethod=TikZ]{mdframed}
\usepackage{amsmath,amssymb,amsfonts,mathtools}
\usepackage[amsmath,thmmarks,framed]{ntheorem}
\usepackage{colonequals}
\usepackage{wasysym}

% allow to smash underbraces:
% use \smashedUnderbrace instead of \underbrace
% surround formula with \reboxSmashedUnderbraces
% https://web.archive.org/web/20160820203342/http://tex.stackexchange.com/questions/38930/ignoring-vertical-space-due-to-underbraces-in-cases-environment
\newcommand\realSmashC[1]{\smash{\mathclap{#1}}}
\newcommand\realSmashR[1]{\smash{\mathrlap{#1}}}
\newcommand\smashedUnderbrace[2]{\underbrace{#1}_{#2}}
\newcommand\actualSmashUnderbrace[2]{%
	\realSmashR{\underbrace{#1}_{\realSmashC{#2}}} \phantom{#1}}
\newcommand\reboxSmashedUnderbraces[1]{%
	{\renewcommand{\smashedUnderbrace}{\actualSmashUnderbrace}#1}%
	\vphantom{#1}}

\allowdisplaybreaks[0]


\newcommand{\R}{\mathbb R}
\newcommand{\N}{\mathbb N}
\newcommand{\K}{\mathbb K}
\renewcommand{\C}{\mathbb C}
\newcommand{\D}{\mathcal{D}}
\newcommand{\spann}{\operatorname{span}}
\newcommand{\extendadd}{\;\; \updownarrow \hspace{-12pt} \longleftrightarrow}
\newcommand{\qq}[1]{\glqq #1\grqq}

\renewcommand{\Re}{\operatorname{Re}}
\renewcommand{\Im}{\operatorname{Im}}

\newcommand{\abs}[1]{\lvert #1 \rvert}
\newcommand{\norm}[1]{\lVert #1 \rVert}


\newcommand{\skiparound}{10pt}

\theoremstyle{plain}
\theoremheaderfont{\osfamily\normalsize\bfseries\upshape}
\theorembodyfont{\normalsize}
\theoremseparator{.}
\theoremsymbol{}


\newmdtheoremenv[%
	%style=boxedtheorem,%
	backgroundcolor=cdblue!10,%
	linecolor=cddarkblue!90,%
	%innertopmargin=\boxskip,%
	%innerbottommargin=\boxskip,%
	innerleftmargin=4pt,%
	outerlinewidth=5pt,%
	topline=false,%
	rightline=true,%
	leftline=false,%
	bottomline=false,%
	%innertopmargin=3pt,%
	%innerbottommargin=3pt,%
	leftmargin=-4pt,%
	%rightmargin=-10pt,%
	skipabove=\skiparound,%
	skipbelow=\skiparound,%
	nobreak,%
]{satz}{Satz}

\newmdtheoremenv[%
	%style=boxedtheorem,%
	backgroundcolor=cdindigo!5,%
	linecolor=cdindigo!50,%
	%innertopmargin=\boxskip,%
	%innerbottommargin=\boxskip,%
	innerleftmargin=4pt,%
	outerlinewidth=5pt,%
	topline=false,%
	rightline=true,%
	leftline=false,%
	bottomline=false,%
	%innertopmargin=3pt,%
	%innerbottommargin=3pt,%
	leftmargin=-4pt,%
	%rightmargin=-10pt,%
	%frametitlefont=\osfamily\bfseries,%
	%skipabove=5pt,%
	%skipbelow=5pt,%
	skipabove=\skiparound,%
	skipbelow=\skiparound,%
	nobreak,%
]{lemma}{Lemma}


\theoremstyle{nonumberplain}

\newmdtheoremenv[%
	%style=boxedtheorem,%
	backgroundcolor=cdorange!10,%
	linecolor=cdorange,%
	%innertopmargin=\boxskip,%
	%innerbottommargin=\boxskip,%
	innerleftmargin=4pt,%
	outerlinewidth=5pt,%
	topline=false,%
	rightline=true,%
	leftline=false,%
	bottomline=false,%
	%innertopmargin=3pt,%
	%innerbottommargin=3pt,%
	leftmargin=-4pt,%
	%rightmargin=-10pt,%
	%frametitlefont=\osfamily\bfseries,%
	%skipabove=5pt,%
	%skipbelow=5pt,%
	skipabove=\skiparound,%
	skipbelow=\skiparound,%
	nobreak,%
]{defi}{Definition}

\newmdtheoremenv[%
	%style=boxedtheorem,%
	backgroundcolor=cdorange!10,%
	linecolor=cdorange,%
	%innertopmargin=\boxskip,%
	%innerbottommargin=\boxskip,%
	innerleftmargin=4pt,%
	outerlinewidth=5pt,%
	topline=false,%
	rightline=true,%
	leftline=false,%
	bottomline=false,%
	%innertopmargin=3pt,%
	%innerbottommargin=3pt,%
	leftmargin=-4pt,%
	%rightmargin=-10pt,%
	%frametitlefont=\osfamily\bfseries,%
	%skipabove=5pt,%
	%skipbelow=5pt,%
	skipabove=\skiparound,%
	skipbelow=\skiparound,%
	nobreak,%
]{definition}{Definition}


\theoremstyle{nonumberplain}
\theorempreskip{\skiparound}
\theorempostskip{\skiparound}

\newtheorem{bsp}{Beispiel}
\newtheorem{algo}{Algorithmus}
\newtheorem{folg}{Folgerung}
\newtheorem{kor}{Korollar}

\theoremheaderfont{\normalfont \bfseries}
\theorembodyfont{\itshape}
\newtheorem{bem}{Bemerkung}
\newtheorem{hinw}{Hinweis}

\theoremheaderfont{\normalfont \bfseries}
\newtheorem{idea}{Idee}
\newtheorem{aim}{Ziel}
\newtheorem{question}{Frage}

\theorembodyfont{}
\newtheorem{aufw}{Aufwand}

\newtheorem{vSfS}{vSfS}

%\newtheoremstyle{proofstyle}%
%{\item[\hskip\labelsep {\theorem@headerfont ##1}\theorem@separator]}%
%{\item[\hskip\labelsep {\theorem@headerfont ##1}\ (##3)\theorem@separator]}

\theoremstyle{nonumberplain}
\theoremheaderfont{\normalsize\slshape}
\theorembodyfont{}
\theoremseparator{.}
\theorempreskip{0pt}
\theorempostskip{5pt}
\theoremsymbol{$\square$}
\newtheorem{proof}{Beweis}

\usepackage{chngcntr}
\counterwithin{satz}{chapter}
\counterwithin{lemma}{chapter}

%\pagestyle{scrheadings}
%\clearscrheadings
%\ihead{Mathematik}
%\chead{\textbf{Numerik}}
%\ohead{Prof.~Dr.~O.\,Sander}
%\setheadsepline{0.4pt}
\cfoot{\pagemark}

%%  All graphics files may be in this subdirectory
\graphicspath{{gfx/}}
%\includeonly{numerische-quadratur}
\begin{document}

%\frontmatter

%\begin{titlepage}
%	\pagecolor{cddarkblue!90} \color{white}%
%	\raggedright \fosfamily%
%	\setlength{\parindent}{0pt}% 
%	% Logo / Kopf
%	\hspace{-18.6mm} %
%	\includegraphics[scale=0.6]{TUD-white.pdf} \\
%	\vspace{3mm} 
%	\begin{tabular}{m{\textwidth}}
%		\hline
%		\hspace{-4pt}\small{\textbf{Fakultät Mathematik} Institut für Numerik, Professur für Numerik partieller Differentialgleichungen} \\
%		\hline
%	\end{tabular} \\
%	% Titel
%	\vspace{5cm}
%	{\Huge\bfseries \MakeUppercase Optimierung \& Numerik --- Teil 2 \par}
%		\vspace{0.5cm}%
%		{\Large \itshape Numerik gewöhnlicher Differentialgleichungen} \\%
%	\vspace{1.5cm}
%	\textbf{{\Large Prof. Dr. Oliver Sander}} \par
%	\vspace{0.5cm}
%	{\large Sommersemester 2020} \par
%	
%	\vspace{1.5cm}
%	
%	
%	\vfill%
%	% Fußzeile
%	\begin{tabular}{lll}
%		Name  & : & Eric Kunze \\
%		E-Mail & : & \url{eric.kunze@mailbox.tu-dresden.de} \\
%		Datum & : & \today
%	\end{tabular}%
%\end{titlepage}
%\nopagecolor
%
%%%%%%%%%%%%%%%%%%%%%%%%%%%%%%%%%%%%%%%%%%%%%%%%%%%%%%%%%%%%%%%%%%%%%%%%%%%%%%%%%
%%%  Copyright notice
%%%%%%%%%%%%%%%%%%%%%%%%%%%%%%%%%%%%%%%%%%%%%%%%%%%%%%%%%%%%%%%%%%%%%%%%%%%%%%%%%
%
%\vspace*{\fill}
%
%\begin{center}
%	\slshape
%	Das vorliegende Skript basiert nahezu vollständig auf dem Vorlesungsskript von Prof. Dr. Oliver Sander für die Vorlesung \enquote{Optimierung und Numerik} an der Technischen Universität Dresden. Das aktuelle Original ist unter \\ \url{https://gitlab.mn.tu-dresden.de/osander/skript-numerik} \\
%	zu finden.
%\end{center}
%
%\vspace*{\fill}
%
%\pagebreak


%%%%%%%%%%%%%%%%%%%%%%%%%%%%%%%%%%%%%%%%%%%%%%%%%%%%%%%%%%%%%%%%%%%%%%%%%%%

\renewcommand*\contentsname{\huge \centering Optimierung \& Numerik --- Vorlesung 11}
\tableofcontents

%%%%%%%%%%%%%%%%%%%%%%%%%%%%%%%%%%%%%%%%%%%%%%%%%%%%%%%%%%%%%%%%%%%%%%%%%%%

\setcounter{chapter}{11}
\setcounter{section}{7}
\setcounter{satz}{17}
\setcounter{lemma}{14}
\setcounter{equation}{4}

\section{Erhalt erster Integrale}

Betrachte die autonome Differentialgleichung $x'=f(x)$ auf einem Phasenraum $\Omega_0$.

\begin{definition}
	Eine Funktion $\mathcal E \colon \Omega_0 \to \R$ heißt \begriff{erstes Integral}, wenn
	\begin{equation*}
	\mathcal E(\Phi^t x) = \mathcal E(x)
	\end{equation*}
	für alle $x\in\Omega_0$ und alle zulässigen $t$ gilt.
\end{definition}

Alternative Bezeichnungen: Invariante, Erhaltungsgröße, engl.: constant of motion

\begin{bsp}
	Mathematisches (Faden-)pendel
	\begin{center}
		\begin{tikzpicture}
		\draw (0,4) -- (5,4);
		\foreach \i in {0,...,10}
		\draw (\i / 2, 4) -- ( \i/2+ 0.5, 4.5);
		\draw[fill] (2.5,4) circle (2pt);
		\draw[fill] (4,2.15) circle (2pt) node[below right]{$m=1$};
		\draw (0.5,2.4) arc (-120:-60:4);
		\draw  (2.5,4) -- node[right]{$l=1$} (4,2.15) ;
		\draw (2.5,4) -- (2.5,1.85);
		\draw[dashed] (4,2.15) -- (2.5,2.15);
		\draw (2.5, 3) arc (-90:-50:1);
		\draw (2.8,3.3) node {$q$};
		\draw [decorate,decoration={brace,amplitude=4pt}]
		(2.5,2.15)--(2.5,4) node[midway, left] {$\cos(q)$};
		\end{tikzpicture}
	\end{center}
	
	Bewegungsgleichungen für Winkel $q$:
	\begin{equation*}
	\ddot{q} + \frac{g}{l} \sin q = 0
	\end{equation*}
	bzw. als System erster Ordnung:
	\begin{equation*}
	\begin{aligned}
	\dot{p} &= - mgl \sin q\\
	\dot{q} &= \frac{1}{ml^2} p
	\end{aligned}
	\end{equation*}
	Erhält $\mathcal E(p,q) = \frac{1}{2} \frac{1}{ml^2} p^2 - mgl \cos q $ (die totale Energie).
\end{bsp}

\begin{bsp}
	Betrachte ein System mit $N$ Partikeln
	\begin{itemize}
		\item $q_i\in\R^3,\ i=1,\hdots,N$ Positionen, $p_i\in\R,\ i=1,\hdots,N$ Impulse
		\item $m_i$: Massen
		\item Paarweise Interaktion über Kräfte, die vom Abstand abhängen.
	\end{itemize}
	Bewegungsgleichungen:
	\begin{equation*}
	q_i'=\frac{p_i}{m_i},\qquad p_i' = \sum_{j=1}^N \nu_{ij}(q_i-q_j)
	\end{equation*}
	mit
	\begin{align*}
	\nu_{ij}(y) = - \nu_{ji}(-y)
	\end{align*}
	daraus folgt insbesondere dass $\nu_{ii} = 0$.
	
	Die Bewegungsgleichungen erhalten den Gesamtimpuls $P=\sum_{i=1}^Np_i$, denn
	\begin{equation*}
	\frac{d}{dt}\sum_{i=1}^Np_i = \sum_{i=1}^N p_i' = \sum_{i=1}^N\sum_{j=1}^N\nu_{ij}(q_i-q_j) = 0
	\end{equation*}
	Ebenso: Der Gesamtdrehimpuls $L=\sum_{i=1}^N q_i\times p_i$
	\begin{align*}
	\frac{d}{dt}\sum_{i=1}^N q_i\times p_i
	& = \sum_{i=1}^N q_i'\times p_i+\sum_{i=1}^N q_i\times p_i' \\
	& = \sum_{i=1}^N \frac{1}{m_i}\underbrace{p_i\times p_i}_{=0} + \sum_{i=1}^N\sum_{j=1}^Nq_i\times\nu_{ij}(q_i-q_j) \\
	& = 0.
	\end{align*}
\end{bsp}

Klassifikation der Erhaltungsgrößen: (für diese Beispiele)
\begin{itemize}
	\item Impuls: linear
	\item Drehimpuls: quadratisch
	\item Energie beim Fadenpendel: nichtlinear
\end{itemize}

Man hätte nun gerne numerische Verfahren, die erste Integrale erhalten.

Zunächst eine einfache Charakterisierung mit Hilfe von $f$:

\begin{lemma}[{{\cite[6.56]{deuflhard_bornemann:2008}}}]
	Sei $f$ lokal Lipschitz-stetig. Eine Funktion $\mathcal E\in C^1(\Omega_0,\R)$ ist genau dann erstes Integral, wenn
	\begin{equation*}
	\nabla\mathcal E(x)\cdot f(x) = 0
	\end{equation*}
	für alle $x\in\Omega_0$.
\end{lemma}
\begin{proof}
	Die Kettenregel liefert $\displaystyle 0 = \frac{d}{dt} \mathcal E(\Phi^t x) = \nabla\mathcal E(\Phi^tx)\cdot \frac{d}{dt}\Phi^tx = \nabla\mathcal E(\Phi^tx)\cdot f(\Phi^tx)$.
\end{proof}

\begin{bsp}
	Wir zeigen Energieerhaltung des Fadenpendels.
	Für dieses Modell gilt:
	\begin{align*}
	f(x) & = f(p,q)
	=
	\begin{pmatrix}
	- mgl \sin q \\ \frac{p}{ml^2}
	\end{pmatrix} \\
	%
	\mathcal{E}(x) & = \mathcal{E}(p,q)
	=
	\frac{1}{2} \frac{p^2}{ml^2} - mgl \cos q
	\end{align*}
	Der Gradient der Energie ist
	\begin{equation*}
	\nabla \mathcal{E}(p,q) =
	\begin{pmatrix}
	\frac{p}{ml^2} \\ mgl \sin q
	\end{pmatrix}
	\end{equation*}
	Damit erhält man
	\begin{equation*}
	\nabla \mathcal{E}(p,q) \cdot f(p,q)
	= \frac{p}{ml^2}(- mgl \sin q) + mgl \sin q\frac{p}{ml^2}
	= 0
	\end{equation*}
\end{bsp}


\begin{satz}[{{\cite[Thm.\ IV.1.5]{hairer_lubich_wanner:2006}}}]
	Alle Runge-Kutta-Verfahren erhalten lineare Invarianten.
\end{satz}
\begin{proof}%\mbox{}
	Sei $\mathcal{E}$ lineare Invariante, also $\mathcal{E}(x) = d^Tx$ mit festem Vektor $d$.
	Nach dem vorigen Satz ist dann $d^T f(x) = 0$ für alle $x\in\Omega_0$.
	Für eine Stufe $k_i$ eines beliebigen RK-Verfahrens ist dann
	\begin{equation*}
	d^T k_i = d^T f\Big( x+\tau\sum_{j=1}^s a_{ij}k_j \Big) = 0.
	\end{equation*}
	Also ist
	\begin{equation*}
	\mathcal E(x_{k+1})
	= d^T x_{k+1}
	= d^T\Big(x_k+\tau\sum_{i=1}^s b_{i}k_i\Big)
	= d^T x_k
	= \mathcal{E}(x_k)
	\end{equation*}
\end{proof}

Für die quadratischen Invarianten betrachten wir zunächst einen wichtigen Spezialfall:

\textbf{Frage:} Für welche linearen autonomen Differentialgleichungen $x' = Ax$ erhält der Phasenfluss $\Phi^t$ die Euklidische Norm $\norm{\Phi^t x}_2 = \norm{x}_2$ für alle $t$?
\textbf{Antwort:} Genau dann, wenn $\Phi^t = \exp(tA)$ eine orthogonale Matrix ist.
\begin{satz}[{{\cite[6.18]{deuflhard_bornemann:2008}}}]
	\label{thm:normerhaltende_fluesse}
	Sei $A\in\R^{d \times d}$. Die Matrix $\exp(tA)$ ist genau dann orthogonal,
	wenn $A$ schiefsymmetrisch ist.
\end{satz}
\begin{proof}
	\begin{description}
		\item[($\boldsymbol{\Rightarrow}$)] Sei $\exp(tA)\in O(d)$ für alle $t$.
		Dann ist
		\begin{equation*}
		I = \exp(tA)^T \exp(tA) = \exp(tA^T) \exp(tA).
		\end{equation*}
		Differenziere nach $t$ und betrachte $t=0$
		\begin{equation*}
		0=\Big(A^T \exp(tA^T) \exp(tA) + \exp(tA^T)A \exp(tA) \Big)\Big|_{t=0} = A^T+A
		\end{equation*}
		\item[($\boldsymbol{\Leftarrow}$)] 
		\begin{align*}
		I
		&= \exp(tA-tA) = \exp(tA)\cdot \exp(-tA)\quad\text{(da $A$ mit $A$ kommutiert)}\\
		&= \exp(tA)\cdot \exp(tA^T) \\
		&= \exp(tA)\cdot \exp(tA)^T	 
		\end{align*}
	\end{description}
\end{proof}

Zentral ist anscheinend die Eigenschaft
\begin{equation*}
\exp(z)\cdot \exp(-z) = 1\qquad \forall z\in\C.
\end{equation*}
Das nennt man \begriff{Reversibilität}.

Man hätte diese Eigenschaft gerne auch für diskrete Verfahren.
\begin{definition}
	Eine diskrete Evolution $\Psi$ heißt reversibel, wenn
	\begin{equation*}
	\Psi^{t,t+\tau}\Psi^{t+\tau,t} x = x
	\end{equation*}
	für alle $(t,x)\in\Omega$ und hinreichend kleine $\tau$.
\end{definition}

\begin{bsp}
	Das explizite Euler-Verfahren ist nicht reversibel.
\end{bsp}

Reversible rationale Approximationen der Exponentialfunktion erzeugen normerhaltende diskrete Flüsse.

\begin{satz}[{{\cite[6.21]{deuflhard_bornemann:2008}}}]
	Sei $R$ eine rationale, konsistente, reversible Approximation der Exponentialfunktion. Dann gilt für eine Matrix $A\in\R^{d\times d}$
	\begin{equation*}
	R(\tau A)\in O(d)\qquad\forall \tau > 0
	\end{equation*}
	genau dann, wenn $A=-A^T$.
\end{satz}
\begin{proof}
	Weitestgehend wie bei Satz~\ref{thm:normerhaltende_fluesse}.
\end{proof}

\begin{bsp}
	\begin{equation*}
	R(z) = \frac{1+\frac{z}{2}}{1-\frac{z}{2}}=1+z+\frac{z^2}{2} + \frac{z^3}{4} + \mathcal{O}(z^4) = e^z + \mathcal O(z^3)
	\end{equation*}
	\begin{itemize}
		\item Die entsprechende Matrix-Abbildung heißt \begriff{Cayley-Transformation}.
		\item Stabilitätsfunktion insbesondere der impliziten Mittelpunktsregel $\to$ dem einfachsten Gauß-Verfahren.
	\end{itemize}
	Gauß-Verfahren erhalten sogar beliebige quadratische Invarianten!
\end{bsp}

\begin{satz}[{{\cite[6.58]{deuflhard_bornemann:2008}}}]
	Die Differentialgleichung $x'=f(x)$ mit lokal Lipschitz-stetigem $f$ besitze das quadratische erste Integral $\mathcal E$, d.h.
	\begin{equation*}
	\mathcal E(x) = x^T E x + e^T x + \eta
	\end{equation*}
	mit $E\in\R^{d\times d},\ e\in\R^d,\ \eta\in\R$. Jedes Gaus-Verfahren erzeugt einen Phasenfluss $\Psi$, der $\mathcal E$ erhält, d.h.
	\begin{equation*}
	\mathcal E(\Psi^\tau x) = \mathcal E(x)
	\end{equation*}
	für alle $x\in\Omega_0$ und zulässige $\tau$.
\end{satz}
\begin{proof}
	Ganz ähnlich wie der Beweis der $B$-Stabilität.
	Sei $x\in\Omega_0$, und $\tau$ so klein, dass das Kollokationspolynom
	\begin{equation*}
	u\in P_s,\quad u(0)=x,\quad u(\tau) = \Psi^\tau x
	\end{equation*}
	existiert. Da $\mathcal E$ quadratisch ist, ist $q(\theta) \colonequals \mathcal E(u(\theta\tau))$ ein Polynom in $P_{2s}$.
	Der Hauptsatz der Integralrechnung liefert
	\begin{equation*}
	\mathcal E(\Psi^\tau x) = q(1) = q(0) + \int_0^1q'(\theta)\,d\theta = \mathcal E(x) + \int_0^1q'(\theta)\,d\theta . 
	\end{equation*}
	Zu zeigen ist nun also $\int_0^1q'(\theta)\,d\theta =0$.
	Nutze die Quadraturformel des Gauß-Verfahrens. Diese ist für Polynome in $P_{2s-1}$ exakt:
	\begin{equation*}
	\int_0^1q'(\theta)\,d\theta = \sum_{j=1}^s b_j q'(c_j).
	\end{equation*}
	Es sind aber alle $q'(c_j)=0$, denn
	\begin{align*}
	q'(c_j) 
	&= \big(\mathcal E(u(c_j\tau))\big)' \\
	&= \tau\nabla\mathcal E(u(c_j\tau))\cdot u'(c_j\tau) \tag{Kettenregel} \\
	&= \tau\nabla\mathcal E(u(c_j\tau))\cdot f(u(c_j\tau)) \tag{Kollokationseigenschaft} \\
	&= 0 \tag{da $\mathcal E$ eine Invariante ist}
	\end{align*}
\end{proof}

Was ist mit der Energieerhaltung des Fadenpendels? $\to$ Das behandeln wir später mit der Theorie der Hamiltonschen Systeme.


\bibliographystyle{alpha}
\bibliography{skript-numerik-sander}

\end{document}