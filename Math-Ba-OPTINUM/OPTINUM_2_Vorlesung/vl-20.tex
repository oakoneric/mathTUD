\documentclass[german]{scrreprt}

\usepackage[top=2.5cm,bottom=2.5cm,left=2.5cm,right=2.5cm]{geometry}

\usepackage[utf8]{inputenc}
\usepackage[T1]{fontenc}
%\setlength{\parindent}{0pt}
\usepackage{parskip}

\usepackage{lmodern}
\usepackage{color}
\usepackage[ngerman]{babel}
\usepackage{csquotes}

\usepackage[scale=0.95]{opensans}	% new font OpenSans
\newcommand*{\osfamily}{\fontfamily{fos}\selectfont}
\DeclareTextFontCommand{\textos}{\fosfamily}

\let\emph\textbf
\newcommand{\begriff}[1]{\textbf{#1}}

\addtokomafont{disposition}{\fosfamily}
\addtokomafont{chapter}{\Huge}
\addtokomafont{section}{\LARGE}
\addtokomafont{subsection}{\Large}

\usepackage{scrlayer-scrpage}
\usepackage[linesnumbered,inoutnumbered]{algorithm2e}
%\usepackage[defernumbers=true,
%            maxbibnames=99,  % Never write 'et al.' in a bibliography
%            giveninits=true, % Abbreviate first names
%            sortcites=true,  % Sort citations when there are more than one
%            natbib=true,     % Allow commands like \citet and \citep
%            backend=biber]{biblatex}
%\addbibresource{skript-numerik-sander.bib}

\usepackage[makeroom]{cancel}


\usepackage{enumerate}
\usepackage[inline]{enumitem} 		%customize label

\renewcommand{\labelitemi}{\raisebox{1pt}{\scalebox{.5}{$\blacksquare$}}}
\renewcommand{\labelitemii}{$\vartriangleright$}
\renewcommand{\labelitemiii}{--}
% Variantionen des Dreiecks als Aufzählungszeichen $\blacktriangleright$ / $\vartriangleright$ / $\triangleright$

%\renewcommand{\labelenumi}{(\alph{enumi})}
%\renewcommand{\labelenumii}{\roman{enumii}.}
%\renewcommand{\labelenumiii}{\roman{enumiii}.}



\usepackage{microtype}
\usepackage{todonotes}
\usepackage{pdfcomment}  % todonotes without tikz -- the two don't play well together
\newcommand{\todoannot}[2]{
  \pdffreetextcomment[width=\textwidth,height=#1,color=orange]{#2}
}

\usepackage{xcolor}
\usepackage[table,dvipsnames]{tudscrcolor}

\usepackage{esdiff}
\usepackage{url}
\usepackage{overpic}
\usepackage{hyperref}
\usepackage{array}
\usepackage{tikz}
\usetikzlibrary{arrows,calc,cd,decorations.pathmorphing,matrix,patterns,positioning,shapes}
\usepackage{graphicx}
\usepackage{float}


\usepackage{pgfplots}
\usepgfplotslibrary{fillbetween,colormaps}
\pgfplotsset{compat=newest}


\usepackage[ntheorem,framemethod=TikZ]{mdframed}
\usepackage{amsmath,amssymb,amsfonts,mathtools}
\usepackage[amsmath,thmmarks,framed]{ntheorem}
\usepackage{colonequals}
\usepackage{wasysym}

% allow to smash underbraces:
% use \smashedUnderbrace instead of \underbrace
% surround formula with \reboxSmashedUnderbraces
% https://web.archive.org/web/20160820203342/http://tex.stackexchange.com/questions/38930/ignoring-vertical-space-due-to-underbraces-in-cases-environment
\newcommand\realSmashC[1]{\smash{\mathclap{#1}}}
\newcommand\realSmashR[1]{\smash{\mathrlap{#1}}}
\newcommand\smashedUnderbrace[2]{\underbrace{#1}_{#2}}
\newcommand\actualSmashUnderbrace[2]{%
	\realSmashR{\underbrace{#1}_{\realSmashC{#2}}} \phantom{#1}}
\newcommand\reboxSmashedUnderbraces[1]{%
	{\renewcommand{\smashedUnderbrace}{\actualSmashUnderbrace}#1}%
	\vphantom{#1}}

\allowdisplaybreaks[0]


\newcommand{\R}{\mathbb R}
\newcommand{\N}{\mathbb N}
\newcommand{\K}{\mathbb K}
\renewcommand{\C}{\mathbb C}
\newcommand{\D}{\mathcal{D}}
\newcommand{\spann}{\operatorname{span}}
\newcommand{\extendadd}{\;\; \updownarrow \hspace{-12pt} \longleftrightarrow}
\newcommand{\qq}[1]{\glqq #1\grqq}

\renewcommand{\Re}{\operatorname{Re}}
\renewcommand{\Im}{\operatorname{Im}}

\newcommand{\abs}[1]{\lvert #1 \rvert}
\newcommand{\norm}[1]{\lVert #1 \rVert}

%\newcommand{\diff}{\mathrm{d}}
\newcommand{\diffskip}[1]{\enskip \mathrm{d}#1}
\newcommand{\dx}{\diffskip{x}}
\newcommand{\dt}{\diffskip{t}}


\newcommand{\skiparound}{10pt}

\theoremstyle{plain}
\theoremheaderfont{\osfamily\normalsize\bfseries\upshape}
\theorembodyfont{\normalsize}
\theoremseparator{.}
\theoremsymbol{}


\newmdtheoremenv[%
	%style=boxedtheorem,%
	backgroundcolor=cdblue!10,%
	linecolor=cddarkblue!90,%
	%innertopmargin=\boxskip,%
	%innerbottommargin=\boxskip,%
	innerleftmargin=4pt,%
	outerlinewidth=5pt,%
	topline=false,%
	rightline=true,%
	leftline=false,%
	bottomline=false,%
	%innertopmargin=3pt,%
	%innerbottommargin=3pt,%
	leftmargin=-4pt,%
	%rightmargin=-10pt,%
	skipabove=\skiparound,%
	skipbelow=\skiparound,%
	nobreak,%
]{satz}{Satz}

\newmdtheoremenv[%
	%style=boxedtheorem,%
	backgroundcolor=cdindigo!5,%
	linecolor=cdindigo!50,%
	%innertopmargin=\boxskip,%
	%innerbottommargin=\boxskip,%
	innerleftmargin=4pt,%
	outerlinewidth=5pt,%
	topline=false,%
	rightline=true,%
	leftline=false,%
	bottomline=false,%
	%innertopmargin=3pt,%
	%innerbottommargin=3pt,%
	leftmargin=-4pt,%
	%rightmargin=-10pt,%
	%frametitlefont=\osfamily\bfseries,%
	%skipabove=5pt,%
	%skipbelow=5pt,%
	skipabove=\skiparound,%
	skipbelow=\skiparound,%
	nobreak,%
]{lemma}{Lemma}


\theoremstyle{nonumberplain}

\newmdtheoremenv[%
	%style=boxedtheorem,%
	backgroundcolor=cdorange!10,%
	linecolor=cdorange,%
	%innertopmargin=\boxskip,%
	%innerbottommargin=\boxskip,%
	innerleftmargin=4pt,%
	outerlinewidth=5pt,%
	topline=false,%
	rightline=true,%
	leftline=false,%
	bottomline=false,%
	%innertopmargin=3pt,%
	%innerbottommargin=3pt,%
	leftmargin=-4pt,%
	%rightmargin=-10pt,%
	%frametitlefont=\osfamily\bfseries,%
	%skipabove=5pt,%
	%skipbelow=5pt,%
	skipabove=\skiparound,%
	skipbelow=\skiparound,%
	nobreak,%
]{defi}{Definition}

\newmdtheoremenv[%
	%style=boxedtheorem,%
	backgroundcolor=cdorange!10,%
	linecolor=cdorange,%
	%innertopmargin=\boxskip,%
	%innerbottommargin=\boxskip,%
	innerleftmargin=4pt,%
	outerlinewidth=5pt,%
	topline=false,%
	rightline=true,%
	leftline=false,%
	bottomline=false,%
	%innertopmargin=3pt,%
	%innerbottommargin=3pt,%
	leftmargin=-4pt,%
	%rightmargin=-10pt,%
	%frametitlefont=\osfamily\bfseries,%
	%skipabove=5pt,%
	%skipbelow=5pt,%
	skipabove=\skiparound,%
	skipbelow=\skiparound,%
	nobreak,%
]{definition}{Definition}


\theoremstyle{nonumberplain}
\theorempreskip{\skiparound}
\theorempostskip{\skiparound}

\newtheorem{bsp}{Beispiel}
\newtheorem{algo}{Algorithmus}
\newtheorem{folg}{Folgerung}
\newtheorem{kor}{Korollar}

\theoremheaderfont{\normalfont \bfseries}
\theorembodyfont{\itshape}
\newtheorem{bem}{Bemerkung}
\newtheorem{hinw}{Hinweis}

\theoremheaderfont{\normalfont \bfseries}
\newtheorem{idea}{Idee}
\newtheorem{aim}{Ziel}
\newtheorem{question}{Frage}

\theorembodyfont{}
\newtheorem{aufw}{Aufwand}

\newtheorem{vSfS}{vSfS}

%\newtheoremstyle{proofstyle}%
%{\item[\hskip\labelsep {\theorem@headerfont ##1}\theorem@separator]}%
%{\item[\hskip\labelsep {\theorem@headerfont ##1}\ (##3)\theorem@separator]}

\theoremstyle{nonumberplain}
\theoremheaderfont{\normalsize\slshape}
\theorembodyfont{}
\theoremseparator{.}
\theorempreskip{0pt}
\theorempostskip{5pt}
\theoremsymbol{$\square$}
\newtheorem{proof}{Beweis}

\usepackage{chngcntr}
\counterwithin{satz}{chapter}
\counterwithin{lemma}{chapter}

%\pagestyle{scrheadings}
%\clearscrheadings
%\ihead{Mathematik}
%\chead{\textbf{Numerik}}
%\ohead{Prof.~Dr.~O.\,Sander}
%\setheadsepline{0.4pt}
\cfoot{\pagemark}

%%  All graphics files may be in this subdirectory
\graphicspath{{gfx/}}
%\includeonly{numerische-quadratur}

\usepackage{accents}
\renewcommand*{\dot}[1]{\accentset{\mbox{\large\bfseries .}}{#1}}
\renewcommand*{\ddot}[1]{\accentset{\mbox{\large\bfseries .\hspace{-0.25ex}.}}{#1}}


\begin{document}

\setlength{\marginparwidth}{2cm}
\setlength{\marginparsep}{3pt}


%\frontmatter

%\begin{titlepage}
%	\pagecolor{cddarkblue!90} \color{white}%
%	\raggedright \fosfamily%
%	\setlength{\parindent}{0pt}% 
%	% Logo / Kopf
%	\hspace{-18.6mm} %
%	\includegraphics[scale=0.6]{TUD-white.pdf} \\
%	\vspace{3mm} 
%	\begin{tabular}{m{\textwidth}}
%		\hline
%		\hspace{-4pt}\small{\textbf{Fakultät Mathematik} Institut für Numerik, Professur für Numerik partieller Differentialgleichungen} \\
%		\hline
%	\end{tabular} \\
%	% Titel
%	\vspace{5cm}
%	{\Huge\bfseries \MakeUppercase Optimierung \& Numerik --- Teil 2 \par}
%		\vspace{0.5cm}%
%		{\Large \itshape Numerik gewöhnlicher Differentialgleichungen} \\%
%	\vspace{1.5cm}
%	\textbf{{\Large Prof. Dr. Oliver Sander}} \par
%	\vspace{0.5cm}
%	{\large Sommersemester 2020} \par
%	
%	\vspace{1.5cm}
%	
%	
%	\vfill%
%	% Fußzeile
%	\begin{tabular}{lll}
%		Name  & : & Eric Kunze \\
%		E-Mail & : & \url{eric.kunze@mailbox.tu-dresden.de} \\
%		Datum & : & \today
%	\end{tabular}%
%\end{titlepage}
%\nopagecolor
%
%%%%%%%%%%%%%%%%%%%%%%%%%%%%%%%%%%%%%%%%%%%%%%%%%%%%%%%%%%%%%%%%%%%%%%%%%%%%%%%%%
%%%  Copyright notice
%%%%%%%%%%%%%%%%%%%%%%%%%%%%%%%%%%%%%%%%%%%%%%%%%%%%%%%%%%%%%%%%%%%%%%%%%%%%%%%%%
%
%\vspace*{\fill}
%
%\begin{center}
%	\slshape
%	Das vorliegende Skript basiert nahezu vollständig auf dem Vorlesungsskript von Prof. Dr. Oliver Sander für die Vorlesung \enquote{Optimierung und Numerik} an der Technischen Universität Dresden. Das aktuelle Original ist unter \\ \url{https://gitlab.mn.tu-dresden.de/osander/skript-numerik} \\
%	zu finden.
%\end{center}
%
%\vspace*{\fill}
%
%\pagebreak


%%%%%%%%%%%%%%%%%%%%%%%%%%%%%%%%%%%%%%%%%%%%%%%%%%%%%%%%%%%%%%%%%%%%%%%%%%%

\renewcommand*\contentsname{\huge \centering Optimierung \& Numerik --- Vorlesung 20}
\tableofcontents

%%%%%%%%%%%%%%%%%%%%%%%%%%%%%%%%%%%%%%%%%%%%%%%%%%%%%%%%%%%%%%%%%%%%%%%%%%%

\setcounter{chapter}{12}
\setcounter{section}{5}
\setcounter{subsection}{2}
\setcounter{satz}{8}
\setcounter{lemma}{1}
\setcounter{equation}{5}





\subsection{Variationelle Integratoren sind symplektisch}

Wir schreiben die diskrete Wirkung jetzt wieder als Funktion von Anfangs- und Endzustand
\begin{equation*}
S_h(q_0, q_N) = \sum_{k=0}^{N-1} L_h (q_k, q_{k+1}).
\end{equation*}
Dabei ist $\{q_k\}$ die dazugehörige Lösung des variationellen Integrators.

Wir rechnen wieder die partiellen Ableitungen aus. Es bezeichne $\frac{\partial L_h}{\partial x}$, $\frac{\partial L_h}{\partial y}$ die partiellen Ableitungen von $L_h$ nach dem ersten bzw.\ zweiten Argument:
\begin{align*}
\frac{\partial S_h}{\partial q_0}
& = \sum_{k=0}^{N-1} \Bigg[ \frac{\partial L_h}{\partial x} \cdot\frac{\partial q_k}{\partial q_0} +  \frac{\partial L_h}{\partial y} \cdot\frac{\partial q_{k+1}}{\partial q_0}\Bigg] \\
%
& = \frac{\partial L_h}{\partial x} (q_0,q_1)\cdot \underbrace{\frac{\partial q_0}{\partial q_0}}_{=1} \\
& \qquad  + \sum_{k=1}^{N-1} \Bigg[ \underbrace{\frac{\partial L_h}{\partial y}(q_{k-1},q_k) \cdot\frac{\partial q_k}{\partial q_0} +  \frac{\partial L_h}{\partial x} (q_k,q_{k+1})\cdot\frac{\partial q_k}{\partial q_0}}_{\text{$=0$, wg.\ diskreter Lagrange-Gleichung}}\Bigg] \\
& \qquad \qquad  + \frac{\partial L_h}{\partial y} (q_{N-1},q_N)\cdot \underbrace{\frac{\partial q_N}{\partial q_0}}_{=0} \\
%
& = \frac{\partial L_h}{\partial x} (q_0, q_1)
\end{align*}
Ebenso 
\begin{equation*}
\frac{\partial S_h}{\partial q_N}  = \frac{\partial L_h}{\partial y} (q_{N-1}, q_N)
\end{equation*}

Jetzt führen wir die \textbf{diskreten Impulse} durch eine \textbf{diskrete Legendre-Transformation} ein:
\begin{equation}
\label{eq:diskrete_lagrange_transformation}
p_k
\colonequals
- \frac{\partial L_h}{\partial x} (q_k, q_{k+1})
= \frac{\partial L_h}{\partial y} (q_{k-1}, q_k)
\end{equation}
Die Gleichheit ist gerade die diskrete Euler--Lagrange-Gleichung.

[Vergleiche: $p = \frac{\partial L}{\partial \dot{q}}(q,\dot{q})$]

Für $k=N$ erhalten wir
\begin{equation*}
p_N = \frac{\partial L_h}{\partial y} (q_{N-1}, q_N).
\end{equation*}

Zusammen also:
\begin{equation*}
\nabla S_h = \begin{pmatrix}
\frac{\partial S_h}{\partial q_0} \\
\frac{\partial S_h}{\partial q_N}
\end{pmatrix} 
=
\begin{pmatrix}
\frac{\partial L_h}{\partial x} (q_0, q_1) \\
\frac{\partial L_h}{\partial y} (q_{N-1},q_N)
\end{pmatrix}
=
\begin{pmatrix}
- p_0 \\ p_N
\end{pmatrix}
\end{equation*}

Nach Satz~\ref{thm:erzeugendenfunktion} ist $S_h$ also eine Erzeugendenfunktion für die symplektische Abbildung
\begin{equation*}
(p_0, q_0) \mapsto (p_N, q_N)
\end{equation*} 



\subsection{Variationelle Integratoren als klassische Einschrittverfahren}

Jetzt bauen wir uns ein klassisches Zeitschrittverfahren:

Angenommen die diskrete Legendre-Transformation \eqref{eq:diskrete_lagrange_transformation} sei eine Bijektion zwischen $p_k$ und $q_{k+1}$.

Einschrittverfahren:
\begin{center}
	\begin{tikzpicture}
	\draw (0,0)  node{$(p_k, q_k)$};
	\draw[->] (0,-1) -- node[right]{inv. disk. Legendre-Transformation} (0,-2);
	\draw (0,-3) node{$(q_k, q_{k+1})$};
	\draw[->] (0,-4)-- node[right]{diskrete Euler--Lagrange-Gleichung}(0,-5);
	\draw (0,-6) node{$(q_{k+1}, q_{k+2})$};
	\draw[->] (0,-7)-- node[right]{diskrete Legendre-Transformation}(0,-8);
	\draw (0,-9) node{$(p_{k+1}, q_{k+1})$};
	\end{tikzpicture}
\end{center}
Schritt 2 und 3 schreibt man als
\begin{equation*}
p_{k+1} = \frac{\partial L_h}{\partial y} (q_k, q_{k+1}).
\end{equation*}

\begin{satz}
	Das diskrete Hamilton Prinzip erzeugt das Zeitschrittverfahren
	\begin{equation*}
	(p_k, q_k) \mapsto (p_{k+1}, q_{k+1}),
	\qquad
	p_k = - \frac{\partial L_h}{\partial x} (q_k, q_{k+1}),
	\qquad
	p_{k+1}  = \frac{\partial L_h}{\partial y} (q_k, q_{k+1}).
	\end{equation*}
	Die erste Gleichung ist dabei als implizite Gleichung für $q_{k+1}$ zu verstehen.
	\begin{enumerate}[label=(\roman*)]
		\item Dieses Verfahren ist symplektisch.
		\item Jedes symplektische Verfahren lässt sich auf diese Art darstellen.
	\end{enumerate}
\end{satz}
\begin{proof}\mbox{} % Force line break
	\begin{enumerate}[label=(\roman*)]
		\item $L_h$ ist Erzeugendenfunktion für die symplektische Abbildung $(p_k, q_k) \mapsto (p_{k+1}, q_{k+1})$
		\item Jede symplektische Abbildung hat eine Erzeugendenfunktion. Wähle diese als diskrete Lagrange-Funktion.
	\end{enumerate}
\end{proof}

\begin{bsp}
	Das Verfahren von \cite{mackay:1992}:
	\begin{equation*}
	L_h (q_k, q_{k+1}) = \frac{\tau}{2} L(q_k, v_{k + \frac{1}{2}}) + \frac{\tau}{2} L(q_{k+1}, v_{k + \frac{1}{2}})
	\end{equation*}
	mit $v_{k + \frac{1}{2}} \colonequals \frac{1}{\tau} (q_{k+1} - q_k)$.
	\newline
	\newline
	Man erhält das Verfahren:
	\begin{flalign*}
	p_k &= - \frac{\partial L_h}{\partial x} (q_k, q_{k+1}) \\
	&= - \frac{\tau}{2} \Big[ \frac{\partial L}{\partial q}(q_k, v_{k + \frac{1}{2}}) + \frac{\partial L}{\partial \dot{q}}(q_k, v_{k + \frac{1}{2}}) \cdot \Big(-\frac{1}{\tau}\Big) \\
	& \quad + \frac{\partial L}{\partial q}(q_{k+1}, v_{k + \frac{1}{2}}) \cdot \underbrace{\frac{\partial q_{k+1}}{\partial q_n}}_{=0} + \frac{\partial L}{\partial \dot{q}}(q_{k+1}, v_{k + \frac{1}{2}}) \cdot \Big(-\frac{1}{\tau}\Big) \Big] \\
	&= - \frac{\tau}{2}\frac{\partial L}{\partial q}(q_k, v_{k + \frac{1}{2}}) + \frac{1}{2} \frac{\partial L}{\partial \dot{q}}(q_k, v_{k + \frac{1}{2}}) +  \frac{1}{2}\frac{\partial L}{\partial \dot{q}}(q_{k+1}, v_{k + \frac{1}{2}}),
	\end{flalign*}
	sowie
	\begin{flalign*}
	p_{k+1}
	& = \frac{\partial L_h}{\partial y} (q_k, q_{k+1}) \\
	%
	&= \frac{\tau}{2}\frac{\partial L}{\partial q}(q_{k+1}, v_{k + \frac{1}{2}}) + \frac{1}{2} \frac{\partial L}{\partial \dot{q}}(q_k, v_{k + \frac{1}{2}}) +  \frac{1}{2}\frac{\partial L}{\partial \dot{q}}(q_{k+1}, v_{k + \frac{1}{2}}) .
	\end{flalign*}
\end{bsp}


Wir betrachten das mechanische System
\begin{equation*}
L (q, \dot{q}) = \frac{1}{2} \dot{q}^T M \dot{q} - U(q) \quad\text{mit $M$ s.p.d}
\end{equation*}
Damit ist
\begin{align*}
\frac{\partial L}{\partial \dot{q}}(q_k, v_{k + \frac{1}{2}}) & = Mv_{k + \frac{1}{2}} = \frac{\partial L}{\partial \dot{q}} (q_{k+1}, v_{k + \frac{1}{2}}) \\
%
\frac{\partial L}{\partial q}(q_k, v_{k + \frac{1}{2}}) & = - \nabla U(q_k) = F(q_k) \quad\text{(Kraftfeld an der Stelle $q_k$)}
\end{align*}
Das Verfahren wird also zu:
\begin{align*}
p_k & = - \frac{\tau}{2} F(q_k) + \frac{1}{2} Mv_{k+\frac{1}{2}}  + \frac{1}{2} Mv_{k+\frac{1}{2}} \\
p_{k + 1} & =  \frac{\tau}{2} F(q_{k+1}) + \frac{1}{2} Mv_{k+\frac{1}{2}}  + \frac{1}{2} Mv_{k+\frac{1}{2}}
\end{align*}
Umschreiben:
\begin{align*}
Mv_{k+\frac{1}{2}} & = p_k + \frac{\tau}{2} F(q_k) \tag{erste Gleichung}\\
q_{k + 1} & = q_k + \tau v_{k+\frac{1}{2}} \tag{Definition von $v_{k+ \frac{1}{2}}$}\\
p_{k + 1} & = Mv_{k+\frac{1}{2}}  + \frac{\tau}{2} F(q_{k+1}) \tag{zweite Gleichung}\\
\end{align*}

Das ist gerade das \textbf{Störmer-Verlet-Verfahren}!

Diskrete Lagrange-Gleichung in diesem Fall:
\begin{align*}
0 & =  \frac{\partial L_h}{\partial y} (q_{k-1}, q_k) + \frac{\partial L_h}{\partial x} (q_k, q_{k+1}) \\
\Leftrightarrow M\underbrace{\frac{(q_{k+1} - 2q_k + q_{k-1})}{\tau^2}}_{\approx \ddot{q}_k} & = F(q_k)
\end{align*}

Kann also als direkte Diskretisierung der Bewegungsgleichung $M \ddot{q} = F(q)$ interpretiert werden!


\bibliographystyle{alpha}
\bibliography{skript-numerik-sander}

\end{document}