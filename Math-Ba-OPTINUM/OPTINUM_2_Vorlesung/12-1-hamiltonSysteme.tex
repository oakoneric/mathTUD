\section{Hamilton-Systeme}

Extrem wichtige Klasse von Differentialgleichungen entstammen der klassischen Mechanik, Quantenmechanik und relativistische Mechanik. Dazu gehören auch spezielle numerische Verfahren --- eine \enquote{schöne Mathematik}.

\enquote{Vereinigendes Prinzip}: Bringt ganz unterschiedliche Gleichungen auf eine gemeinsame Form.


\begin{bsp}
	Mathematisches Pendel -- Fadenpendel
	\begin{itemize}
		\item Koordinate: Winkel $\alpha$
		\item Masse $m$, Fadenlänge $l$, Erdbeschleunigung $g$
	\end{itemize}
	Bewegungsgleichungen: $\displaystyle \ddot \alpha + \frac{g}{l} \sin \alpha = 0$
\end{bsp}

\begin{bsp}
	Teilchen in einem Kraftfeld $F(x)$
	\begin{equation}
	m \ddot x = F(x) \tag{Newtons Gesetz}
	\end{equation}
\end{bsp}

\begin{bsp}
	1d-Wellengleichung --- Longitudinale Auslenkung einer elastischen Schnur
	\begin{alignat*}{2}
	\frac{\partial^2 u}{\partial x^2} & = \frac{\partial^2 u}{\partial t^2} & \qquad & x \in [a, b], t \ge 0 \\
	u(a, t) = u(b, t) & = 0 && \forall t \ge 0
	\end{alignat*}
\end{bsp}


\subsection{Die Lagrange-Gleichungen}
\label{sec:lagrange_gleichung}

Wir betrachten ein mechanisches System mit $d$ Freiheitsgraden $q = (q_1, \dots, q_d)$.
\begin{itemize}
	\item Kinetische Energie: $T = T(q, \dot q)$
	(häufig: $T(q, \dot{q}) = \frac12 \dot{q}^T M(q) \dot{q}$ mit $M(q)$ s.p.d.)
	\item Potentielle Energie: $U = U(q)$
\end{itemize}

\begin{definition}
	Die Lagrange-Funktion eines mechanischen Systems ist $L = T - U$.
\end{definition}

Das mechanische System löst die Lagrange-Gleichungen
\begin{equation*}
\frac{d}{dt} \left( \frac{\partial L}{\partial \dot q} \right) = \frac{\partial L}{\partial q}
\end{equation*}

\emph{Warum?} Es gilt das Prinzip der stationären Wirkung.

\begin{definition}[Prinzip der stationären Wirkung / Hamilton'sches Prinzip]
	Sei $q: [t_0, t_1] \to \R^d$ eine Trajektorie eines mechanischen Systems.
	Für die in der Natur vorkommenden Trajektorien ist die \textit{Wirkung}
	\begin{equation*}
	W \colonequals \int_{t_0}^{t_1} L(q(t) \ \dot q(t)) \dt
	\end{equation*}
	stationär.
\end{definition}

Sei $q$ eine Trajektorie, und $\delta q$ eine Variation davon, die die Endpunkte fest lässt, also $\delta q(t_0)=\delta q(t_1) = 0$.
Stationarität von $q$ heißt dann, dass für alle solche $\delta q$
\begin{equation*}
\frac{d}{d\epsilon} S(q+\epsilon\delta q)\vert_{\epsilon=0} = 0.
\end{equation*}
gilt. Ausrechnen:
\begin{align*}
\frac{d}{d\epsilon} \int_{t_0}^{t_1} L(q+ \epsilon\delta q,\dot q + \epsilon\delta \dot q)\dt\vert_{\epsilon=0} &= \int_{t_0}^{t_1} \frac{\partial L}{\partial q}\delta q + \frac{\partial L}{\partial\dot q} \delta\dot q\, dt \\
&= \int_{t_0}^{t_1} \Bigg(\frac{\partial L}{\partial q} \delta q - \frac{d}{dt} \frac{\partial L}{\partial \dot q} \delta q\Bigg)\dt \tag{partielle Integration} \\
&= \int_{t_0}^{t_1} \Bigg(\frac{\partial L}{\partial q} - \frac{d}{dt} \frac{\partial L}{\partial \dot q}  \Bigg)\delta q\dt
\end{align*}
Da dieser Ausdruck für alle hinreichend glatten Funktionen $\delta q$ gleich Null sein muss, erhält man die Lagrange-Gleichung
\begin{equation*}
\delta W = 0 = \frac{d}{dt} \left( \frac{\partial L}{\partial \dot q} \right) - \frac{\partial L}{\partial q}
\end{equation*}

\begin{bsp}[Pendel]
	\begin{itemize}
		\item Kinetische Energie
		\begin{equation*}
		T = \frac12 m (\dot x^2 + \dot y^2) = \frac12 m l^2 \dot\alpha^2
		\end{equation*}
		\item Potentielle Energie
		\begin{equation*}
		U = mgy = -mgl \cos \alpha
		\end{equation*}
		\item Lagrange-Funktion
		\begin{equation*}
		L(\alpha, \dot\alpha) = \frac12 m l^2 \dot\alpha^2 + mgl \cos \alpha
		\end{equation*}
		\item Lagrange-Gleichung
		\begin{equation*}
		0
		= \frac{d}{dt} \left( \frac{\partial L}{\partial \dot q} \right) - \frac{\partial L}{\partial q}
		= \frac{d}{dt}( ml^2\dot\alpha ) + mgl \sin \alpha
		= ml^2\ddot\alpha + mgl \sin \alpha.
		\end{equation*}
	\end{itemize}
\end{bsp}

\begin{bsp}[Teilchen in einem Kraftfeld]
	Angenommen das Kraftfeld ist \emph{konservativ}, d.h. es gibt ein $U: \R^3 \to \R$, so dass $F(x) = -\nabla U(x)$.
	\begin{itemize}
		\item Kinetische Energie
		\begin{equation*}
		T(x, \dot x) = \frac12 m \langle \dot x, \dot x \rangle
		\end{equation*}
		\item Potentielle Energie
		\begin{equation*}
		U
		\end{equation*}
		\item Lagrange-Gleichung
		\begin{equation*}
		0
		= \frac{d}{dt} \left( \frac{\partial L}{\partial \dot q} \right) - \frac{\partial L}{\partial q}
		= \frac{d}{dt}(m\dot x) + \nabla U(x) = m\ddot x - F(x)
		\end{equation*}
	\end{itemize}
\end{bsp}

\begin{bsp}[Eindimensionale Wellengleichung]
	Ein unendlich-dimensionales System wird nicht beschrieben durch $d$
	Freiheitsgrade $(q_1, \dots, q_d)$, sondern durch die Funktion $u: [a, b] \to \R$. Diese beschreibt die transversale Auslenkung einer Saite.
	\begin{itemize}
		\item Kinetische Energie
		\begin{equation*}
		T(u, \dot u) = \frac12 \int_a^b m \dot u(x)^2 dx
		\end{equation*}
		Dabei ist $m$ die Massendichte.
		\item Potentielle Energie
		\begin{equation*}
		U(u)
		=
		\int_a^b S \Big[ \sqrt{1+ u'(x)^2} -1 \Big] dx
		\approx
		\int_a^b S\frac{u'(x)^2}{2} dx
		\end{equation*}
		Dabei ist $S$ die Zugsteifigkeit.
		\item Lagrange-Funktion
		\begin{equation*}
		L(u, \dot u) = T(u, \dot u) - U(u)
		\end{equation*}
		\item Lagrange-Gleichung
		\begin{equation*}
		\frac{\partial L}{\partial u}
		=
		\frac{\partial}{\partial x} \frac{\partial L}{\partial u'}
		+ \frac{\partial}{\partial t} \frac{\partial L}{\partial \dot{u}}
		\end{equation*}
		Einsetzen:
		\begin{equation*}
		0 =
		\frac{\partial}{\partial x} \Big(- S u'(x)\Big)
		+ \frac{\partial}{\partial t}m\dot{u}
		\end{equation*}
		Umstellen:
		\begin{equation*}
		\frac{\partial^2 u}{\partial t^2} = \frac{S}{m} \frac{\partial^2 u}{\partial x^2}
		\end{equation*}
		Das ist die eindimensionale Wellengleichung.
	\end{itemize}
\end{bsp}


%\subsection{Die Hamiltonschen Gleichungen}
%
%Eine Transformation der Lagrange-Gleichung;
%quasi \glqq die andere Seite der Medaille\grqq
%
%\begin{itemize}
%\item Definiere die Impulse
%  \begin{equation*}
%    p_k \colonequals \frac{\partial L}{\partial \dot q_k}(q, \dot q) \qquad \text{für } k = 1, \dots, d
%  \end{equation*}
%  Diese Abbildung heißt \emph{Legendre-Transformation}.
%\end{itemize}
%
%\begin{definition}
%  Die Hamilton-Funktion ist
%  \begin{equation*}
%    H(p, q) \colonequals p^T \dot q - L(q, \dot q).
%  \end{equation*}
%\end{definition}
%
%Dabei geht man natürlich davon aus, dass die Legendre-Transformation eine $C^1$-Bijektion $\dot q \leftrightarrow p$ darstellt.
%
%\emph{Beispiel}: kinetische Energie ist quadratisch:
%
%\begin{equation*}
%  T = \frac 12 \dot q^T M \dot q \qquad \text{ mit $M$ s.p.d.}
%\end{equation*}
%
%\begin{itemize}
%\item Legendre-Transformation: Für festes $q$ hat man
%  \begin{equation*}
%    p = M \dot q.
%  \end{equation*}
%  Transformation ist also tatsächlich glatte Bijektion.
%\item Hamilton-Funktion
%  \begin{align*}
%    H(p, q)
%    & = p^T \dot q - L(q, \dot q) \\
%    & = p^T M^{-1} p - L(q, M^{-1} p) \\
%    & = p^T M^{-1} p - T(q, M^{-1} p) + U(q) \\
%    & = p^T M^{-1} p - \frac12 (M^{-1}p)^T M (M^{-1} p) + U(q) \\
%    & = \frac12 (M^{-1}p)^T M (M^{-1}p) + U(q) \\
%    & = T + U
%  \end{align*}
%  Die Hamilton-Funktion ist die Gesamtenergie!
%\end{itemize}
%
%Auch mit Hilfe der Hamilton-Funktion kann man das Verhalten des mechanischen Systems einfach ausdrücken.
%
%\begin{satz}[{\citet[Thm.\,VI.1.3]{hairer_lubich_wanner:2006}}]
%  Die Lagrange-Gleichung ist äquivalent zu den Hamilton-Gleichungen
%  \begin{equation*}
%    \dot p_k = -\frac{\partial H}{\partial q_k}(p, q),
%    \quad
%    \dot q_k = \frac{\partial H}{\partial p_k}(p, q),
%    \qquad
%    k = 1, \dots, d.
%  \end{equation*}
%\end{satz}
%
%\begin{proof}
%  Lagrange $\implies$ Hamilton (die andere Richtung ist ähnlich)
%  \begin{align*}
%    \frac{\partial H}{\partial q}
%    & = \frac\partial{\partial q}\left( p^T \dot q - L(q, \dot q) \right) & \text{(Def. von $H$)} \\
%    & = p^T \frac{\partial \dot q}{\partial q} - \frac{\partial L}{\partial q} - \underbrace{\frac{\partial L}{\partial \dot q}}_{= p^T} \frac{\partial \dot q}{\partial q} & \text{(Kettenregel)} \\
%    & = - \frac{\partial L}{\partial q} & \text{(Def. von $p = \frac{\partial L}{\partial \dot q}$)} \\
%    & = - \frac{d}{dt}\left( \frac{\partial L}{\partial \dot q} \right) & \text{(Lagrange-Gleichung)} \\
%    & = - \dot p & \text{(Def. von $p$)}
%  \end{align*}
%  Und:
%  \begin{align*}
%    \frac{\partial H}{\partial p}
%    & = \frac{\partial}{\partial p}\left( p^T \dot q - L(q, \dot q) \right) \\
%    & = \dot q + p^T \frac{\partial \dot q}{\partial p} - \underbrace{\frac{\partial L}{\partial \dot q}}_{=p^T} \frac{\partial \dot q}{\partial p} & \text{(Produktregel; $q$ hängt nicht von $p$ ab)} \\
%    & = \dot q \qedhere
%  \end{align*}
%\end{proof}
%
%Sowohl die Lagrangesche als auch die Hamiltonsche Formulierungen haben ihre Daseinsberechtigung.
%\begin{itemize}
%\item Die Lagrange-Formulierung ist besonders fundamental: sie beruht auf Variationsprinzipien
%\item Die Hamilton-Formulierung ist besonders fundamental: sie beruht auf der Gesamtenergie des Systems.
%\end{itemize}
%
%\emph{Beispiel}: Pendel (mit $q = \alpha$)
%\begin{itemize}
%\item Kinetische Energie
%  \begin{equation*}
%    T = \frac12 m l^2 \dot q^2
%  \end{equation*}
%\item Potentielle Energie
%  \begin{equation*}
%    U = -mgl \cos q
%  \end{equation*}
%
%\item Impuls
%  \begin{equation*}
%    p
%    \colonequals \frac{\partial L}{\partial\dot q}
%    = \frac{\partial}{\partial\dot q}\Big( \frac12 m l^2 \dot q^2 + mgl \cos q \Big)
%    = m l^2 \dot q
%  \end{equation*}
%\item Kinetische Energie ist quadratisch, also
%  \begin{align*}
%    H(p, q)
%    & = T(p, q) + U(q) \\
%    & = \frac12 m l^2 \dot q^2 - m g l \cos q \\
%    & = \frac12 \frac{1}{m l^2} p^2 - m g l \cos q
%  \end{align*}
%\item Bewegungsgleichungen:
%  \begin{eqnarray*}
%    \dot p = -\frac{\partial H}{\partial q} & \Leftrightarrow & \dot p = - m g l \sin q \\
%    \dot q = \frac{\partial H}{\partial p} & \Leftrightarrow & \dot q = \frac{1}{m l^2} p
%  \end{eqnarray*}
%\end{itemize}
%
%In der letzten Vorlesung hatten wir gesehen, dass das Pendel die Größe
%\begin{equation*}
%  \frac12 \frac{1}{ml^2} p^2 - mgl \cos q = T(p, q) + U(p)
%\end{equation*}
%erhält.
%Das ist kein Zufall.
%
%\begin{satz}
%  Die Hamilton-Funktion $H$ ist Invariante des Flusses der Hamiltonschen Gleichung.
%\end{satz}
%
%\begin{proof}
%  \begin{align*}
%    \frac{d}{dt} H(p, q)
%    & = \frac{\partial H}{\partial p} \dot p + \frac{\partial H}{\partial q} \dot q & \text{(Kettenregel)} \\
%    & = \frac{\partial H}{\partial p} \Big(-\frac{\partial H}{\partial q}\Big)
%      + \frac{\partial H}{\partial q} \Big(\frac{\partial H}{\partial p}\Big) & \text{(Hamiltonsche Gl.)} \\
%    & = 0 \qedhere
%  \end{align*}
%\end{proof}
%
%Diese sehr allgemeine Erhaltungseigenschaft wollen wir natürlich ins Diskrete übertragen!