\documentclass[german]{scrreprt}

\usepackage[top=2.5cm,bottom=2.5cm,left=2.5cm,right=2.5cm]{geometry}

\usepackage[utf8]{inputenc}
\usepackage[T1]{fontenc}
%\setlength{\parindent}{0pt}
\usepackage{parskip}

\usepackage{lmodern}
\usepackage{color}
\usepackage[ngerman]{babel}
\usepackage{csquotes}

\usepackage[scale=0.95]{opensans}	% new font OpenSans
\newcommand*{\osfamily}{\fontfamily{fos}\selectfont}
\DeclareTextFontCommand{\textos}{\fosfamily}

\let\emph\textbf
\newcommand{\begriff}[1]{\textbf{#1}}

\addtokomafont{disposition}{\fosfamily}
\addtokomafont{chapter}{\Huge}
\addtokomafont{section}{\LARGE}
\addtokomafont{subsection}{\Large}

\usepackage{scrlayer-scrpage}
\usepackage[linesnumbered,inoutnumbered]{algorithm2e}
%\usepackage[defernumbers=true,
%            maxbibnames=99,  % Never write 'et al.' in a bibliography
%            giveninits=true, % Abbreviate first names
%            sortcites=true,  % Sort citations when there are more than one
%            natbib=true,     % Allow commands like \citet and \citep
%            backend=biber]{biblatex}
%\addbibresource{skript-numerik-sander.bib}

\usepackage[makeroom]{cancel}


\usepackage{enumerate}
\usepackage[inline]{enumitem} 		%customize label

\renewcommand{\labelitemi}{\raisebox{1pt}{\scalebox{.5}{$\blacksquare$}}}
\renewcommand{\labelitemii}{$\vartriangleright$}
\renewcommand{\labelitemiii}{--}
% Variantionen des Dreiecks als Aufzählungszeichen $\blacktriangleright$ / $\vartriangleright$ / $\triangleright$

%\renewcommand{\labelenumi}{(\alph{enumi})}
%\renewcommand{\labelenumii}{\roman{enumii}.}
%\renewcommand{\labelenumiii}{\roman{enumiii}.}



\usepackage{microtype}
\usepackage{todonotes}
\usepackage{pdfcomment}  % todonotes without tikz -- the two don't play well together
\newcommand{\todoannot}[2]{
  \pdffreetextcomment[width=\textwidth,height=#1,color=orange]{#2}
}

\usepackage{xcolor}
\usepackage[table,dvipsnames]{tudscrcolor}

\usepackage{esdiff}
\usepackage{url}
\usepackage{overpic}
\usepackage{hyperref}
\usepackage{array}
\usepackage{tikz}
\usetikzlibrary{arrows,calc,cd,decorations.pathmorphing,matrix,patterns,positioning,shapes}
\usepackage{graphicx}
\usepackage{float}


\usepackage{pgfplots}
\usepgfplotslibrary{fillbetween,colormaps}
\pgfplotsset{compat=newest}


\usepackage[ntheorem,framemethod=TikZ]{mdframed}
\usepackage{amsmath,amssymb,amsfonts,mathtools}
\usepackage[amsmath,thmmarks,framed]{ntheorem}
\usepackage{colonequals}
\usepackage{wasysym}

% allow to smash underbraces:
% use \smashedUnderbrace instead of \underbrace
% surround formula with \reboxSmashedUnderbraces
% https://web.archive.org/web/20160820203342/http://tex.stackexchange.com/questions/38930/ignoring-vertical-space-due-to-underbraces-in-cases-environment
\newcommand\realSmashC[1]{\smash{\mathclap{#1}}}
\newcommand\realSmashR[1]{\smash{\mathrlap{#1}}}
\newcommand\smashedUnderbrace[2]{\underbrace{#1}_{#2}}
\newcommand\actualSmashUnderbrace[2]{%
	\realSmashR{\underbrace{#1}_{\realSmashC{#2}}} \phantom{#1}}
\newcommand\reboxSmashedUnderbraces[1]{%
	{\renewcommand{\smashedUnderbrace}{\actualSmashUnderbrace}#1}%
	\vphantom{#1}}

\allowdisplaybreaks[0]


\newcommand{\R}{\mathbb R}
\newcommand{\N}{\mathbb N}
\newcommand{\K}{\mathbb K}
\renewcommand{\C}{\mathbb C}
\newcommand{\D}{\mathcal{D}}
\newcommand{\spann}{\operatorname{span}}
\newcommand{\extendadd}{\;\; \updownarrow \hspace{-12pt} \longleftrightarrow}
\newcommand{\qq}[1]{\glqq #1\grqq}

\renewcommand{\Re}{\operatorname{Re}}
\renewcommand{\Im}{\operatorname{Im}}

\newcommand{\abs}[1]{\lvert #1 \rvert}
\newcommand{\norm}[1]{\lVert #1 \rVert}

%\newcommand{\diff}{\mathrm{d}}
\newcommand{\diffskip}[1]{\enskip \mathrm{d}#1}
\newcommand{\dx}{\diffskip{x}}
\newcommand{\dt}{\diffskip{t}}


\newcommand{\skiparound}{10pt}

\theoremstyle{plain}
\theoremheaderfont{\osfamily\normalsize\bfseries\upshape}
\theorembodyfont{\normalsize}
\theoremseparator{.}
\theoremsymbol{}


\newmdtheoremenv[%
	%style=boxedtheorem,%
	backgroundcolor=cdblue!10,%
	linecolor=cddarkblue!90,%
	%innertopmargin=\boxskip,%
	%innerbottommargin=\boxskip,%
	innerleftmargin=4pt,%
	outerlinewidth=5pt,%
	topline=false,%
	rightline=true,%
	leftline=false,%
	bottomline=false,%
	%innertopmargin=3pt,%
	%innerbottommargin=3pt,%
	leftmargin=-4pt,%
	%rightmargin=-10pt,%
	skipabove=\skiparound,%
	skipbelow=\skiparound,%
	nobreak,%
]{satz}{Satz}

\newmdtheoremenv[%
	%style=boxedtheorem,%
	backgroundcolor=cdindigo!5,%
	linecolor=cdindigo!50,%
	%innertopmargin=\boxskip,%
	%innerbottommargin=\boxskip,%
	innerleftmargin=4pt,%
	outerlinewidth=5pt,%
	topline=false,%
	rightline=true,%
	leftline=false,%
	bottomline=false,%
	%innertopmargin=3pt,%
	%innerbottommargin=3pt,%
	leftmargin=-4pt,%
	%rightmargin=-10pt,%
	%frametitlefont=\osfamily\bfseries,%
	%skipabove=5pt,%
	%skipbelow=5pt,%
	skipabove=\skiparound,%
	skipbelow=\skiparound,%
	nobreak,%
]{lemma}{Lemma}


\theoremstyle{nonumberplain}

\newmdtheoremenv[%
	%style=boxedtheorem,%
	backgroundcolor=cdorange!10,%
	linecolor=cdorange,%
	%innertopmargin=\boxskip,%
	%innerbottommargin=\boxskip,%
	innerleftmargin=4pt,%
	outerlinewidth=5pt,%
	topline=false,%
	rightline=true,%
	leftline=false,%
	bottomline=false,%
	%innertopmargin=3pt,%
	%innerbottommargin=3pt,%
	leftmargin=-4pt,%
	%rightmargin=-10pt,%
	%frametitlefont=\osfamily\bfseries,%
	%skipabove=5pt,%
	%skipbelow=5pt,%
	skipabove=\skiparound,%
	skipbelow=\skiparound,%
	nobreak,%
]{defi}{Definition}

\newmdtheoremenv[%
	%style=boxedtheorem,%
	backgroundcolor=cdorange!10,%
	linecolor=cdorange,%
	%innertopmargin=\boxskip,%
	%innerbottommargin=\boxskip,%
	innerleftmargin=4pt,%
	outerlinewidth=5pt,%
	topline=false,%
	rightline=true,%
	leftline=false,%
	bottomline=false,%
	%innertopmargin=3pt,%
	%innerbottommargin=3pt,%
	leftmargin=-4pt,%
	%rightmargin=-10pt,%
	%frametitlefont=\osfamily\bfseries,%
	%skipabove=5pt,%
	%skipbelow=5pt,%
	skipabove=\skiparound,%
	skipbelow=\skiparound,%
	nobreak,%
]{definition}{Definition}


\theoremstyle{nonumberplain}
\theorempreskip{\skiparound}
\theorempostskip{\skiparound}

\newtheorem{bsp}{Beispiel}
\newtheorem{algo}{Algorithmus}
\newtheorem{folg}{Folgerung}
\newtheorem{kor}{Korollar}

\theoremheaderfont{\normalfont \bfseries}
\theorembodyfont{\itshape}
\newtheorem{bem}{Bemerkung}
\newtheorem{hinw}{Hinweis}

\theoremheaderfont{\normalfont \bfseries}
\newtheorem{idea}{Idee}
\newtheorem{aim}{Ziel}
\newtheorem{question}{Frage}

\theorembodyfont{}
\newtheorem{aufw}{Aufwand}

\newtheorem{vSfS}{vSfS}

%\newtheoremstyle{proofstyle}%
%{\item[\hskip\labelsep {\theorem@headerfont ##1}\theorem@separator]}%
%{\item[\hskip\labelsep {\theorem@headerfont ##1}\ (##3)\theorem@separator]}

\theoremstyle{nonumberplain}
\theoremheaderfont{\normalsize\slshape}
\theorembodyfont{}
\theoremseparator{.}
\theorempreskip{0pt}
\theorempostskip{5pt}
\theoremsymbol{$\square$}
\newtheorem{proof}{Beweis}

\usepackage{chngcntr}
\counterwithin{satz}{chapter}
\counterwithin{lemma}{chapter}

%\pagestyle{scrheadings}
%\clearscrheadings
%\ihead{Mathematik}
%\chead{\textbf{Numerik}}
%\ohead{Prof.~Dr.~O.\,Sander}
%\setheadsepline{0.4pt}
\cfoot{\pagemark}

%%  All graphics files may be in this subdirectory
\graphicspath{{gfx/}}
%\includeonly{numerische-quadratur}

\usepackage{accents}
\renewcommand*{\dot}[1]{\accentset{\mbox{\large\bfseries .}}{#1}}
\renewcommand*{\ddot}[1]{\accentset{\mbox{\large\bfseries .\hspace{-0.25ex}.}}{#1}}


\begin{document}

\setlength{\marginparwidth}{2cm}
\setlength{\marginparsep}{3pt}


%\frontmatter

%\begin{titlepage}
%	\pagecolor{cddarkblue!90} \color{white}%
%	\raggedright \fosfamily%
%	\setlength{\parindent}{0pt}% 
%	% Logo / Kopf
%	\hspace{-18.6mm} %
%	\includegraphics[scale=0.6]{TUD-white.pdf} \\
%	\vspace{3mm} 
%	\begin{tabular}{m{\textwidth}}
%		\hline
%		\hspace{-4pt}\small{\textbf{Fakultät Mathematik} Institut für Numerik, Professur für Numerik partieller Differentialgleichungen} \\
%		\hline
%	\end{tabular} \\
%	% Titel
%	\vspace{5cm}
%	{\Huge\bfseries \MakeUppercase Optimierung \& Numerik --- Teil 2 \par}
%		\vspace{0.5cm}%
%		{\Large \itshape Numerik gewöhnlicher Differentialgleichungen} \\%
%	\vspace{1.5cm}
%	\textbf{{\Large Prof. Dr. Oliver Sander}} \par
%	\vspace{0.5cm}
%	{\large Sommersemester 2020} \par
%	
%	\vspace{1.5cm}
%	
%	
%	\vfill%
%	% Fußzeile
%	\begin{tabular}{lll}
%		Name  & : & Eric Kunze \\
%		E-Mail & : & \url{eric.kunze@mailbox.tu-dresden.de} \\
%		Datum & : & \today
%	\end{tabular}%
%\end{titlepage}
%\nopagecolor
%
%%%%%%%%%%%%%%%%%%%%%%%%%%%%%%%%%%%%%%%%%%%%%%%%%%%%%%%%%%%%%%%%%%%%%%%%%%%%%%%%%
%%%  Copyright notice
%%%%%%%%%%%%%%%%%%%%%%%%%%%%%%%%%%%%%%%%%%%%%%%%%%%%%%%%%%%%%%%%%%%%%%%%%%%%%%%%%
%
%\vspace*{\fill}
%
%\begin{center}
%	\slshape
%	Das vorliegende Skript basiert nahezu vollständig auf dem Vorlesungsskript von Prof. Dr. Oliver Sander für die Vorlesung \enquote{Optimierung und Numerik} an der Technischen Universität Dresden. Das aktuelle Original ist unter \\ \url{https://gitlab.mn.tu-dresden.de/osander/skript-numerik} \\
%	zu finden.
%\end{center}
%
%\vspace*{\fill}
%
%\pagebreak


%%%%%%%%%%%%%%%%%%%%%%%%%%%%%%%%%%%%%%%%%%%%%%%%%%%%%%%%%%%%%%%%%%%%%%%%%%%

\renewcommand*\contentsname{\huge \centering Optimierung \& Numerik --- Vorlesung 19}
\tableofcontents

%%%%%%%%%%%%%%%%%%%%%%%%%%%%%%%%%%%%%%%%%%%%%%%%%%%%%%%%%%%%%%%%%%%%%%%%%%%

\setcounter{chapter}{12}
\setcounter{section}{5}
\setcounter{satz}{8}
\setcounter{lemma}{1}
\setcounter{equation}{4}

\subsection{Idee der variationellen Integratoren}

Wir ersetzen das Integral im Hamiltonschen Prinzip durch eine diskrete Approximation:

Wir führen ein Zeitgitter $t_0<t_1<\hdots < t_N = T$ und die approximative Wirkung
\begin{equation*}
L_h(q_k,q_{k+1})\approx \int_{t_k}^{t_{k+1}} L(q(t),\dot q(t))\,dt
\end{equation*}
ein (z.B.\ durch eine Quadraturformel).

Hier könnte man denken dass $L_h$ ein schlechtes Symbol ist, weil es sich ja schließlich um eine Wirkung handelt. Andererseits fungiert $L_h$ später bei der Definition der diskreten Impulse wie eine Lagrange-Funktion (siehe~\eqref{eq:diskrete_lagrange_transformation}).

$q$ ist hier die Lösung der Lagrange-Gleichung auf $[t_k,t_{k+1}]$ mit gegebenen Start- und Endwerten $q_k,q_{k+1}$.

Definiere das diskrete Wirkungsfunktional
\begin{equation*}
S_h\Big(\big\{q_k\big\}_{k=0}^N\Big) \colonequals \sum_{k=0}^{N-1} L_h(q_k,q_{k+1})
\end{equation*}

\begin{definition}[Diskretes Hamilton-Prinzip]
	Finde $\{q_k\}^N_{k=0}$  mit gegebenen $q_0, q_N$, so dass $S_h$ stationär wird.
\end{definition}

Wie kann man $L_h$ wählen?

\begin{bsp}[{{\cite[1992]{mackay:1992}}}]
	Approximiere $q$ auf $[t_k, t_{k+1}]$ als linear Interpolierende von $q_k$ und $q_{k+1}$.
	
	Approximiere das Integral durch die Trapezregel
	\begin{equation*}
	L_h(q_k, q_{k+1}) =
	\tau \cdot \frac{1}{2} \Big[L\Big(q_k, \frac{q_{k+1} - q_k}{\tau}\Big) +  L\Big(q_{k+1}, \frac{q_{k+1} - q_k}{\tau}\Big)\Big].
	\end{equation*}
\end{bsp}

\begin{bsp}[{{\cite[1997]{wendlandt_marsden:1997}}}]
	Nimm statt der Trapezregel die Mittelpunktsregel
	\begin{equation*}
	L_h(q_k, q_{k+1}) = \tau \Big[L\Big(\frac{q_{k+1} + q_k}{2}, \frac{q_{k+1} - q_k}{\tau}\Big)\Big].
	\end{equation*}
\end{bsp}


Wie kommen wir an stationäre Punkte von $S_h$?

Wir können die Ableitung ausrechnen und gleich Null setzen --- partielle Ableitung für $k=1, \dots, N-1$:
\begin{equation*}
\frac{\partial S_h}{\partial q_k}
=
\frac{\partial}{\partial q_k} \sum_{i=0}^{N-1} L_h (q_i, q_{i+1})
=
\frac{\partial}{\partial q_k} L_h(q_{k-1}, q_k) +  \frac{\partial}{\partial q_k} L_h(q_k, q_{k+1})
\end{equation*}
Dieser Ausdruck $=0$ sind die \emph{diskreten Euler--Lagrange-Gleichungen} -- ein System von algebraischen Gleichungen (mit Bandstruktur).

\textbf{Fragen:}
\begin{itemize}
	\item Wann kriege ich mit diesem Ansatz symplektische Integratoren?
	\item Kann ich das Verfahren so umschreiben dass ich wieder einen Zeitschritt nach dem anderen berechnen kann?
\end{itemize}

\subsection{Erzeugendenfunktionen}

Wir brauchen ein weiteres Kriterium für Symplektizität:
\begin{itemize}
	\item Betrachte ein gegebenes Hamiltonsches System $H$ auf einem festen Zeitintervall $[t_0,t_1]$
	\item Seien $p_0 \in \R^d$ und $q_0 \in \R^d$ die Startwerte zur Zeit $t_0$
	\item Bezeichne die Werte zur Zeit $t_1$ mit  $p_1 \in \R^d$ und $q_1 \in \R^d$
	\item Es gibt eine Abbildung $\Phi^{t_0,t_1}(p_0,q_0) = (p_1,q_1)$.
\end{itemize}

Wie wir wissen, ist diese symplektisch.
{[Achtung: der folgende Satz enthält überdurchschnittlich viel didaktische Reduktion]}

\begin{satz}[{{\cite[Satz VI.5.1]{hairer_lubich_wanner:2006}}}]
	\label{thm:erzeugendenfunktion}
	Eine Abbildung $\varphi\colon (p_0,q_0) \mapsto (p_1,q_1)$ ist genau dann symplektisch, wenn lokal eine Funktion
	\begin{equation*}
	S\colon (q_0,q_1)\mapsto S(q_0,q_1)\in\R
	\end{equation*}
	existiert, so dass
	\begin{equation}\label{eq:abbildung_symplektisch}
	\nabla S 0
	\begin{pmatrix}
	\frac{\partial S}{\partial q_0}  \\
	\frac{\partial S}{\partial q_1}
	\end{pmatrix} = \begin{pmatrix}
	-p_0 \\ p_1
	\end{pmatrix}
	\end{equation}
\end{satz}

Wenn man eine symplektische Abbildung $(p_0,q_0)\mapsto(p_1,q_1)$ hat, dann kann sie durch \eqref{eq:abbildung_symplektisch} aus der Funktion $S$ rekonstruiert werden.

Aber der obige Satz ist vom Typ \enquote{Äquivalenz}. Es gilt also auch die Umkehrung: Jede hinreichen glatte (und in einem gewissen Sinne nicht degenerierte) Funktion $S$ \emph{erzeugt} via \eqref{eq:abbildung_symplektisch} eine symplektische Abbildung $(p_0,q_0)\mapsto (p_1,q_1)$.

Man kann also auf systematische Art symplektische Abbildungen erzeugen. Die Funktion $S$ heißt deshalb Erzeugendenfunktion.

Wir betrachten jetzt das Wirkungsintegral $S$ als Funktion der Start- und Endposition
\marginpar{\osfamily \footnotesize $q_k$ und $q_{k+1}$ sind zwei Zustände, dazwischen existiert genau ein Pfad $q$, der die Lagrange-Gleichungen löst.}
\begin{equation*}
S(q_0,q_1) = \int_{t_0}^{t_1} L(q(t),\dot q(t))\,dt 
\end{equation*}
Dabei ist $q$ die zu $q_0,q_1$ gehörige Lösung der Lagrange-Gleichung.





Große Überraschung: Diese Funktion $S$ ist gerade die Erzeugendenfunktion einer symplektischen Abbildung!

Berechne die partielle Ableitung:
\begin{align*}
\frac{\partial S}{\partial q_0}
& =
\int_{t_0}^{t_1} \bigg( \frac{\partial L}{\partial q} \frac{\partial q}{\partial q_0} + \frac{\partial L}{\partial \dot q}\frac{\partial\dot q}{\partial q_0}\bigg)\,dt \\
%
\intertext{Partielle Integration des zweiten Terms in der Klammer liefert}
&=
\frac{\partial L}{\partial \dot q}\frac{\partial q}{\partial q_0}\Bigg\vert_{t_0}^{t_1} + \int_{t_0}^{t_1} \underbrace{\bigg( \frac{\partial L}{\partial q}  - \frac{d}{dt} \frac{\partial L}{\partial \dot q}\bigg)}_{=0} \frac{\partial q}{\partial q_0}\,dt \\
&= \frac{\partial L(q_1,\dot q_1)}{\partial \dot q} \cdot \underbrace{\frac{\partial q_1}{\partial q_0}}_{=0}  - \frac{\partial L(q_0,\dot q_0)}{\partial \dot q} \cdot \underbrace{\frac{\partial q_0}{\partial q_0}}_{=1}  \\
&= - \frac{\partial L(q_0,\dot q_0)}{\partial \dot q} \\
&= -p_0 \tag{Def. des Impulses}
\end{align*}
Ebenso berechnet man
\begin{equation*}
\frac{\partial S}{\partial q_1} = p_1.
\end{equation*}
Wir erhalten also
\begin{equation}
\nabla S =
\begin{pmatrix}
\frac{\partial S}{\partial q_0}  \\
\frac{\partial S}{\partial q_1}
\end{pmatrix}
=
\begin{pmatrix}
-p_0 \\ p_1
\end{pmatrix}
\end{equation}
Dies ist gerade die Formel \eqref{eq:abbildung_symplektisch} für Erzeugendenfunktionen von symplektischen Abbildungen.

Daraus folgt dass die entsprechende Abbildung $(p_0,q_0) \mapsto (p_1,q_1)$ symplektisch ist.

\bibliographystyle{alpha}
\bibliography{skript-numerik-sander}

\end{document}