\documentclass[german]{scrreprt}

\usepackage[top=2.5cm,bottom=2.5cm,left=2.5cm,right=2.5cm]{geometry}

\usepackage[utf8]{inputenc}
\usepackage[T1]{fontenc}
%\setlength{\parindent}{0pt}
\usepackage{parskip}

\usepackage{lmodern}
\usepackage{color}
\usepackage[ngerman]{babel}
\usepackage{csquotes}

\usepackage[scale=0.95]{opensans}	% new font OpenSans
\newcommand*{\osfamily}{\fontfamily{fos}\selectfont}
\DeclareTextFontCommand{\textos}{\fosfamily}

\let\emph\textbf
\newcommand{\begriff}[1]{\textbf{#1}}

\addtokomafont{disposition}{\fosfamily}
\addtokomafont{chapter}{\Huge}
\addtokomafont{section}{\LARGE}
\addtokomafont{subsection}{\Large}

\usepackage{scrlayer-scrpage}
\usepackage[linesnumbered,inoutnumbered]{algorithm2e}
%\usepackage[defernumbers=true,
%            maxbibnames=99,  % Never write 'et al.' in a bibliography
%            giveninits=true, % Abbreviate first names
%            sortcites=true,  % Sort citations when there are more than one
%            natbib=true,     % Allow commands like \citet and \citep
%            backend=biber]{biblatex}
%\addbibresource{skript-numerik-sander.bib}

\usepackage[makeroom]{cancel}


\usepackage{enumerate}
\usepackage[inline]{enumitem} 		%customize label

\renewcommand{\labelitemi}{\raisebox{1pt}{\scalebox{.5}{$\blacksquare$}}}
\renewcommand{\labelitemii}{$\vartriangleright$}
\renewcommand{\labelitemiii}{--}
% Variantionen des Dreiecks als Aufzählungszeichen $\blacktriangleright$ / $\vartriangleright$ / $\triangleright$

%\renewcommand{\labelenumi}{(\alph{enumi})}
%\renewcommand{\labelenumii}{\roman{enumii}.}
%\renewcommand{\labelenumiii}{\roman{enumiii}.}



\usepackage{microtype}
\usepackage{todonotes}
\usepackage{pdfcomment}  % todonotes without tikz -- the two don't play well together
\newcommand{\todoannot}[2]{
  \pdffreetextcomment[width=\textwidth,height=#1,color=orange]{#2}
}

\usepackage{xcolor}
\usepackage[table,dvipsnames]{tudscrcolor}

\usepackage{esdiff}
\usepackage{url}
\usepackage{overpic}
\usepackage{hyperref}
\usepackage{array}
\usepackage{tikz}
\usetikzlibrary{arrows,calc,cd,decorations.pathmorphing,matrix,patterns,positioning,shapes}
\usepackage{graphicx}
\usepackage{float}


\usepackage{pgfplots}
\usepgfplotslibrary{fillbetween,colormaps}
\pgfplotsset{compat=newest}


\usepackage[ntheorem,framemethod=TikZ]{mdframed}
\usepackage{amsmath,amssymb,amsfonts,mathtools}
\usepackage[amsmath,thmmarks,framed]{ntheorem}
\usepackage{colonequals}
\usepackage{wasysym}

% allow to smash underbraces:
% use \smashedUnderbrace instead of \underbrace
% surround formula with \reboxSmashedUnderbraces
% https://web.archive.org/web/20160820203342/http://tex.stackexchange.com/questions/38930/ignoring-vertical-space-due-to-underbraces-in-cases-environment
\newcommand\realSmashC[1]{\smash{\mathclap{#1}}}
\newcommand\realSmashR[1]{\smash{\mathrlap{#1}}}
\newcommand\smashedUnderbrace[2]{\underbrace{#1}_{#2}}
\newcommand\actualSmashUnderbrace[2]{%
	\realSmashR{\underbrace{#1}_{\realSmashC{#2}}} \phantom{#1}}
\newcommand\reboxSmashedUnderbraces[1]{%
	{\renewcommand{\smashedUnderbrace}{\actualSmashUnderbrace}#1}%
	\vphantom{#1}}

\allowdisplaybreaks[0]


\newcommand{\R}{\mathbb R}
\newcommand{\N}{\mathbb N}
\newcommand{\K}{\mathbb K}
\renewcommand{\C}{\mathbb C}
\newcommand{\D}{\mathcal{D}}
\newcommand{\spann}{\operatorname{span}}
\newcommand{\extendadd}{\;\; \updownarrow \hspace{-12pt} \longleftrightarrow}
\newcommand{\qq}[1]{\glqq #1\grqq}

\renewcommand{\Re}{\operatorname{Re}}
\renewcommand{\Im}{\operatorname{Im}}

\newcommand{\abs}[1]{\lvert #1 \rvert}
\newcommand{\norm}[1]{\lVert #1 \rVert}

%\newcommand{\diff}{\mathrm{d}}
\newcommand{\diffskip}[1]{\enskip \mathrm{d}#1}
\newcommand{\dx}{\diffskip{x}}
\newcommand{\dt}{\diffskip{t}}


\newcommand{\skiparound}{10pt}

\theoremstyle{plain}
\theoremheaderfont{\osfamily\normalsize\bfseries\upshape}
\theorembodyfont{\normalsize}
\theoremseparator{.}
\theoremsymbol{}


\newmdtheoremenv[%
	%style=boxedtheorem,%
	backgroundcolor=cdblue!10,%
	linecolor=cddarkblue!90,%
	%innertopmargin=\boxskip,%
	%innerbottommargin=\boxskip,%
	innerleftmargin=4pt,%
	outerlinewidth=5pt,%
	topline=false,%
	rightline=true,%
	leftline=false,%
	bottomline=false,%
	%innertopmargin=3pt,%
	%innerbottommargin=3pt,%
	leftmargin=-4pt,%
	%rightmargin=-10pt,%
	skipabove=\skiparound,%
	skipbelow=\skiparound,%
	nobreak,%
]{satz}{Satz}

\newmdtheoremenv[%
	%style=boxedtheorem,%
	backgroundcolor=cdindigo!5,%
	linecolor=cdindigo!50,%
	%innertopmargin=\boxskip,%
	%innerbottommargin=\boxskip,%
	innerleftmargin=4pt,%
	outerlinewidth=5pt,%
	topline=false,%
	rightline=true,%
	leftline=false,%
	bottomline=false,%
	%innertopmargin=3pt,%
	%innerbottommargin=3pt,%
	leftmargin=-4pt,%
	%rightmargin=-10pt,%
	%frametitlefont=\osfamily\bfseries,%
	%skipabove=5pt,%
	%skipbelow=5pt,%
	skipabove=\skiparound,%
	skipbelow=\skiparound,%
	nobreak,%
]{lemma}{Lemma}


\theoremstyle{nonumberplain}

\newmdtheoremenv[%
	%style=boxedtheorem,%
	backgroundcolor=cdorange!10,%
	linecolor=cdorange,%
	%innertopmargin=\boxskip,%
	%innerbottommargin=\boxskip,%
	innerleftmargin=4pt,%
	outerlinewidth=5pt,%
	topline=false,%
	rightline=true,%
	leftline=false,%
	bottomline=false,%
	%innertopmargin=3pt,%
	%innerbottommargin=3pt,%
	leftmargin=-4pt,%
	%rightmargin=-10pt,%
	%frametitlefont=\osfamily\bfseries,%
	%skipabove=5pt,%
	%skipbelow=5pt,%
	skipabove=\skiparound,%
	skipbelow=\skiparound,%
	nobreak,%
]{defi}{Definition}

\newmdtheoremenv[%
	%style=boxedtheorem,%
	backgroundcolor=cdorange!10,%
	linecolor=cdorange,%
	%innertopmargin=\boxskip,%
	%innerbottommargin=\boxskip,%
	innerleftmargin=4pt,%
	outerlinewidth=5pt,%
	topline=false,%
	rightline=true,%
	leftline=false,%
	bottomline=false,%
	%innertopmargin=3pt,%
	%innerbottommargin=3pt,%
	leftmargin=-4pt,%
	%rightmargin=-10pt,%
	%frametitlefont=\osfamily\bfseries,%
	%skipabove=5pt,%
	%skipbelow=5pt,%
	skipabove=\skiparound,%
	skipbelow=\skiparound,%
	nobreak,%
]{definition}{Definition}


\theoremstyle{nonumberplain}
\theorempreskip{\skiparound}
\theorempostskip{\skiparound}

\newtheorem{bsp}{Beispiel}
\newtheorem{algo}{Algorithmus}
\newtheorem{folg}{Folgerung}
\newtheorem{kor}{Korollar}

\theoremheaderfont{\normalfont \bfseries}
\theorembodyfont{\itshape}
\newtheorem{bem}{Bemerkung}
\newtheorem{hinw}{Hinweis}

\theoremheaderfont{\normalfont \bfseries}
\newtheorem{idea}{Idee}
\newtheorem{aim}{Ziel}
\newtheorem{question}{Frage}

\theorembodyfont{}
\newtheorem{aufw}{Aufwand}

\newtheorem{vSfS}{vSfS}

%\newtheoremstyle{proofstyle}%
%{\item[\hskip\labelsep {\theorem@headerfont ##1}\theorem@separator]}%
%{\item[\hskip\labelsep {\theorem@headerfont ##1}\ (##3)\theorem@separator]}

\theoremstyle{nonumberplain}
\theoremheaderfont{\normalsize\slshape}
\theorembodyfont{}
\theoremseparator{.}
\theorempreskip{0pt}
\theorempostskip{5pt}
\theoremsymbol{$\square$}
\newtheorem{proof}{Beweis}

\usepackage{chngcntr}
\counterwithin{satz}{chapter}
\counterwithin{lemma}{chapter}

%\pagestyle{scrheadings}
%\clearscrheadings
%\ihead{Mathematik}
%\chead{\textbf{Numerik}}
%\ohead{Prof.~Dr.~O.\,Sander}
%\setheadsepline{0.4pt}
\cfoot{\pagemark}

%%  All graphics files may be in this subdirectory
\graphicspath{{gfx/}}
%\includeonly{numerische-quadratur}

\usepackage{accents}
\renewcommand*{\dot}[1]{\accentset{\mbox{\large\bfseries .}}{#1}}
\renewcommand*{\ddot}[1]{\accentset{\mbox{\large\bfseries .\hspace{-0.25ex}.}}{#1}}

\begin{document}

%\frontmatter

%\begin{titlepage}
%	\pagecolor{cddarkblue!90} \color{white}%
%	\raggedright \fosfamily%
%	\setlength{\parindent}{0pt}% 
%	% Logo / Kopf
%	\hspace{-18.6mm} %
%	\includegraphics[scale=0.6]{TUD-white.pdf} \\
%	\vspace{3mm} 
%	\begin{tabular}{m{\textwidth}}
%		\hline
%		\hspace{-4pt}\small{\textbf{Fakultät Mathematik} Institut für Numerik, Professur für Numerik partieller Differentialgleichungen} \\
%		\hline
%	\end{tabular} \\
%	% Titel
%	\vspace{5cm}
%	{\Huge\bfseries \MakeUppercase Optimierung \& Numerik --- Teil 2 \par}
%		\vspace{0.5cm}%
%		{\Large \itshape Numerik gewöhnlicher Differentialgleichungen} \\%
%	\vspace{1.5cm}
%	\textbf{{\Large Prof. Dr. Oliver Sander}} \par
%	\vspace{0.5cm}
%	{\large Sommersemester 2020} \par
%	
%	\vspace{1.5cm}
%	
%	
%	\vfill%
%	% Fußzeile
%	\begin{tabular}{lll}
%		Name  & : & Eric Kunze \\
%		E-Mail & : & \url{eric.kunze@mailbox.tu-dresden.de} \\
%		Datum & : & \today
%	\end{tabular}%
%\end{titlepage}
%\nopagecolor
%
%%%%%%%%%%%%%%%%%%%%%%%%%%%%%%%%%%%%%%%%%%%%%%%%%%%%%%%%%%%%%%%%%%%%%%%%%%%%%%%%%
%%%  Copyright notice
%%%%%%%%%%%%%%%%%%%%%%%%%%%%%%%%%%%%%%%%%%%%%%%%%%%%%%%%%%%%%%%%%%%%%%%%%%%%%%%%%
%
%\vspace*{\fill}
%
%\begin{center}
%	\slshape
%	Das vorliegende Skript basiert nahezu vollständig auf dem Vorlesungsskript von Prof. Dr. Oliver Sander für die Vorlesung \enquote{Optimierung und Numerik} an der Technischen Universität Dresden. Das aktuelle Original ist unter \\ \url{https://gitlab.mn.tu-dresden.de/osander/skript-numerik} \\
%	zu finden.
%\end{center}
%
%\vspace*{\fill}
%
%\pagebreak


%%%%%%%%%%%%%%%%%%%%%%%%%%%%%%%%%%%%%%%%%%%%%%%%%%%%%%%%%%%%%%%%%%%%%%%%%%%

\renewcommand*\contentsname{\huge \centering Optimierung \& Numerik --- Vorlesung 13}
\tableofcontents

%%%%%%%%%%%%%%%%%%%%%%%%%%%%%%%%%%%%%%%%%%%%%%%%%%%%%%%%%%%%%%%%%%%%%%%%%%%

\setcounter{chapter}{12}
\setcounter{section}{1}
\setcounter{satz}{1}
\setcounter{lemma}{0}
\setcounter{equation}{0}

\subsection{Die Hamiltonschen Gleichungen}

Eine Transformation der Lagrange-Gleichung; quasi \enquote{die andere Seite der Medaille}

Definiere die Impulse
\begin{equation*}
p_k \colonequals \frac{\partial L}{\partial \dot q_k}(q, \dot q) \qquad \text{für } k = 1, \dots, d
\end{equation*}
Diese Abbildung heißt \emph{Legendre-Transformation}.

\begin{definition}
	Die Hamilton-Funktion ist
	\begin{equation*}
	H(p, q) \colonequals p^T \dot q - L(q, \dot q)
	\end{equation*}
\end{definition}

Dabei geht man natürlich davon aus, dass die Legendre-Transformation eine $C^1$-Bijektion $\dot q \leftrightarrow p$ darstellt.

\begin{bsp}
	kinetische Energie ist quadratisch:
	\begin{equation*}
	T = \frac 12 \dot q^T M \dot q \qquad \text{ mit $M$ s.p.d.}
	\end{equation*}
	
	\begin{itemize}
		\item Legendre-Transformation: Für festes $q$ hat man $p = M \dot q$. Die Transformation ist also tatsächlich eine glatte Bijektion.
		\item Hamilton-Funktion
		\begin{align*}
		H(p, q)
		& = p^T \dot q - L(q, \dot q) \\
		& = p^T M^{-1} p - L(q, M^{-1} p) \\
		& = p^T M^{-1} p - T(q, M^{-1} p) + U(q) \\
		& = p^T M^{-1} p - \frac12 (M^{-1}p)^T M (M^{-1} p) + U(q) \\
		& = \frac12 (M^{-1}p)^T M (M^{-1}p) + U(q) \\
		& = T + U
		\end{align*}
		Die Hamilton-Funktion ist die Gesamtenergie!
	\end{itemize}
\end{bsp}

Auch mit Hilfe der Hamilton-Funktion kann man das Verhalten des mechanischen Systems einfach ausdrücken.

\begin{satz}[{{{\cite[Thm.\,VI.1.3]{hairer_lubich_wanner:2006}}}}]
	Die Lagrange-Gleichung ist äquivalent zu den Hamilton-Gleichungen
	\begin{equation*}
	\dot p_k = -\frac{\partial H}{\partial q_k}(p, q),
	\quad
	\dot q_k = \frac{\partial H}{\partial p_k}(p, q),
	\qquad
	k = 1, \dots, d
	\end{equation*}
\end{satz}
%
%\begin{proof}
%  Lagrange $\implies$ Hamilton (die andere Richtung ist ähnlich)
%  \begin{align*}
%    \frac{\partial H}{\partial q}
%    & = \frac\partial{\partial q}\left( p^T \dot q - L(q, \dot q) \right) & \text{(Def. von $H$)} \\
%    & = p^T \frac{\partial \dot q}{\partial q} - \frac{\partial L}{\partial q} - \underbrace{\frac{\partial L}{\partial \dot q}}_{= p^T} \frac{\partial \dot q}{\partial q} & \text{(Kettenregel)} \\
%    & = - \frac{\partial L}{\partial q} & \text{(Def. von $p = \frac{\partial L}{\partial \dot q}$)} \\
%    & = - \frac{d}{dt}\left( \frac{\partial L}{\partial \dot q} \right) & \text{(Lagrange-Gleichung)} \\
%    & = - \dot p & \text{(Def. von $p$)}
%  \end{align*}
%  Und:
%  \begin{align*}
%    \frac{\partial H}{\partial p}
%    & = \frac{\partial}{\partial p}\left( p^T \dot q - L(q, \dot q) \right) \\
%    & = \dot q + p^T \frac{\partial \dot q}{\partial p} - \underbrace{\frac{\partial L}{\partial \dot q}}_{=p^T} \frac{\partial \dot q}{\partial p} & \text{(Produktregel; $q$ hängt nicht von $p$ ab)} \\
%    & = \dot q \qedhere
%  \end{align*}
%\end{proof}
%
%Sowohl die Lagrangesche als auch die Hamiltonsche Formulierungen haben ihre Daseinsberechtigung.
%\begin{itemize}
%\item Die Lagrange-Formulierung ist besonders fundamental: sie beruht auf Variationsprinzipien
%\item Die Hamilton-Formulierung ist besonders fundamental: sie beruht auf der Gesamtenergie des Systems.
%\end{itemize}
%
%\emph{Beispiel}: Pendel (mit $q = \alpha$)
%\begin{itemize}
%\item Kinetische Energie
%  \begin{equation*}
%    T = \frac12 m l^2 \dot q^2
%  \end{equation*}
%\item Potentielle Energie
%  \begin{equation*}
%    U = -mgl \cos q
%  \end{equation*}
%
%\item Impuls
%  \begin{equation*}
%    p
%    \colonequals \frac{\partial L}{\partial\dot q}
%    = \frac{\partial}{\partial\dot q}\Big( \frac12 m l^2 \dot q^2 + mgl \cos q \Big)
%    = m l^2 \dot q
%  \end{equation*}
%\item Kinetische Energie ist quadratisch, also
%  \begin{align*}
%    H(p, q)
%    & = T(p, q) + U(q) \\
%    & = \frac12 m l^2 \dot q^2 - m g l \cos q \\
%    & = \frac12 \frac{1}{m l^2} p^2 - m g l \cos q
%  \end{align*}
%\item Bewegungsgleichungen:
%  \begin{eqnarray*}
%    \dot p = -\frac{\partial H}{\partial q} & \Leftrightarrow & \dot p = - m g l \sin q \\
%    \dot q = \frac{\partial H}{\partial p} & \Leftrightarrow & \dot q = \frac{1}{m l^2} p
%  \end{eqnarray*}
%\end{itemize}
%
%In der letzten Vorlesung hatten wir gesehen, dass das Pendel die Größe
%\begin{equation*}
%  \frac12 \frac{1}{ml^2} p^2 - mgl \cos q = T(p, q) + U(p)
%\end{equation*}
%erhält.
%Das ist kein Zufall.
%
%\begin{satz}
%  Die Hamilton-Funktion $H$ ist Invariante des Flusses der Hamiltonschen Gleichung.
%\end{satz}
%
%\begin{proof}
%  \begin{align*}
%    \frac{d}{dt} H(p, q)
%    & = \frac{\partial H}{\partial p} \dot p + \frac{\partial H}{\partial q} \dot q & \text{(Kettenregel)} \\
%    & = \frac{\partial H}{\partial p} \Big(-\frac{\partial H}{\partial q}\Big)
%      + \frac{\partial H}{\partial q} \Big(\frac{\partial H}{\partial p}\Big) & \text{(Hamiltonsche Gl.)} \\
%    & = 0 \qedhere
%  \end{align*}
%\end{proof}
%
%Diese sehr allgemeine Erhaltungseigenschaft wollen wir natürlich ins Diskrete übertragen!


\bibliographystyle{alpha}
\bibliography{skript-numerik-sander}

\end{document}