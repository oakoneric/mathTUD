\begin{itemize}[nolistsep, leftmargin=*]
	\item Risikomessung aus der Perspektive einer Bank / Versicherung / Aufsichtsbehörde
	\item Fokus auf ungünstigste (d.h. auf verlustreichste) Szenarien, d.h. Finanzkrise, Kreditausfälle
	\item Wie können wir verhindern, dass Banken / Versicherungen in solchen Szenarien pleite gehen?
	\item aus gesellschaftlicher Sicht wichtig, da \enquote{Bail-out}-Kosten vom Staat / Steuerzahler übernommen werden müssen
\end{itemize}

\begin{description}
	\item[Prinzip:] Risiko muss mit \textit{angemessenem} Risikokapital hinterlegt werden, welches mögliche Verluste abfedern kann
	\item[Analogie:] Damm / Sturmflut --- Wie hoch muss der Damm sein um auch ein Jahrhunderthochwasser standhalten zu können?
	\item[Rechtliches Rahmenwerk:] Basel I, II und II für Banken sowie Solveny I und II für Versicherungen.
\end{description}



In dieser Vorlesung betrachten wir die mathematischen Aspekte.

Betrachte Zufallsvariable $\abb{L}{\Omega}{\R}$ (\enquote{loss} / Verlust), d.h. hohe Werte entsprechen hohen Verlusten.