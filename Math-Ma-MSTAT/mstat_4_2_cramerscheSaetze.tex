% This work is licensed under the Creative Commons
% Attribution-NonCommercial-ShareAlike 4.0 International License. To view a copy
% of this license, visit http://creativecommons.org/licenses/by-nc-sa/4.0/ or
% send a letter to Creative Commons, PO Box 1866, Mountain View, CA 94042, USA.

\subsection{Die Cramér'schen Sätze}
Zusammenstellung einiger Eigenschaften der schwachen Konvergenz und der Verteilungskonvergenz.

\begin{satz}[Teilfolgenprinzip für schwache Konvergenz]\label{satz4.11}\
	\begin{enumerate}[label=(\arabic*)]
		\item \label{it:4.11weakto} $\begin{aligned}
			Q_n\weakto  Q
			⇔
			\forall\text{ TF }(Q_{n'})\subseteq(Q_n)_{n\in\N}\: \exists \text{ TTF }(Q_{n''})\subseteq(Q_{n'}):Q_{n''}\weakto  Q
			% CHECKED: '''' used.
		\end{aligned}$
	\item \label{it:4.11distrto} $\begin{aligned}
			X_n\distto  X⇔\forall\text{ TF }(X_{n'})\subseteq(X_n)_{n\in\N}\: \exists\text{ TTF }(X_{n''})\subseteq(X_{n'}):X_{n''}\distto  X
			% CHECKED: '''' used.
		\end{aligned}$
	\end{enumerate}
\end{satz}

\begin{proof}
	Wir zeigen hier nur \ref{it:4.11weakto}
	\paragraph{Zeige \enquote{$\follows$}:}
	Folgt aus der Definition \ref{def4.1} und dem Teilfolgenprinzip für $(\int f\d Q)_{n\in\N}$ für alle $f\in C^b(\S)$.
	\paragraph{Zeige \enquote{$⇐$}:}
	Angenommen $Q_n\stackrelnew{w}{}{\nrightarrow} Q$.
	Dann gilt:
	\begin{align*}
		\exists f\in C^b(\S):\int f\d Q_n\not\longrightarrow\int f\d Q
	\end{align*}
	Also existiert ein $\epsilon>0$ und eine TF $(X_{n'})\subseteq(X_n)_{n\in\N}$ so, dass
	\begin{align}\label{eqProof4.11Stern}\tag{$\ast$}
		\abs{\int f\d Q_{n'}-\int f\d Q}\ge\epsilon\qquad\forall n'
	\end{align}
	im Widerspruch zur Voraussetzung, da \eqref{eqProof4.11Stern} insbesondere für alle $n''$ gilt, wenn $(Q_{n''})$ eine Teilfolge von $(Q_{n'})$ ist.
	% CHECKED: '''' used.
\end{proof}

\begin{satz}\label{satz4.12}
	\begin{align*}
		X_n\stochto X\follows X_n\distto  X
	\end{align*}
\end{satz}

\begin{proof}
	Sei $(X_{n'})$ eine Teilfolge von $(X_n)_{n\in\N}$.
	Dann existiert gemäß \ref{satz3.13} eine TF $(X_{n''})$ von $(X_{n'})$ mit $X_{n''}\overset{n''\to\infty}{\longrightarrow} X$ $\P$-fast sicher.
	% CHECKED: '''' used.
	Damit folgt aus Satz \ref{Satz3.8}:
	\begin{align*}
		f\brackets{X_{n''}} &\stochto f(X)~\P\text{-fast sicher}&\forall f\in C^b(\S)\\
		% CHECKED: '''' used.
		\overset{\text{dom. Konv}}{\follows}
		\E{f\brackets{X_{n''}}}
		% CHECKED: '''' used.
		\overset{n''\to\infty}&{\longrightarrow}\E{f\brackets{X}}
		% CHECKED: '''' used.
		&\forall f\in C^b(\S)\\
		% CHECKED: '''' used.
		\overset{\ref{satz4.7}}{\follows}
		X_{n''}&\distto  X\\
		% CHECKED: '''' used.
		\overset{\ref{satz4.11}\ref{it:4.11distrto}}{\follows}
		X_n &\distto X
		&&\qedhere
	\end{align*}
\end{proof}
Einfache Beispiele zeigen, dass in \ref{satz4.12} die Umkehrung im Allgemeinen \emph{nicht} gilt. Aber:
%
\begin{satz}\label{satz4.13}
	Sei $X_n\distto X$ mit $X$ fast sicher konstant.
	Dann gilt $X_n \stochto X$.
\end{satz}

\begin{proof}
	Nach Voraussetzung existiert eine Konstante $x\in\S$ mit $\P(X=c)=1$. Wir verwenden das folgende Lemma:
	%
	\begin{lem}[Lemma von Uryson] \label{lemmaVonUryson} \footnote{Uryson ist jung bei einer Tagung in Frankreich im Meer ertrunken.}
		Zu $A,B\in\F(\S)$ mit $A\cap B=∅$ existiert ein stetiges
		\begin{align*}
			f\colon\S\to[0,1]\qquad\mit\qquad f(x)=
			\begin{cases}
				0, &\falls x\in A\\
				1, &\falls x\in B
			\end{cases}
		\end{align*}
	\end{lem}

	Sei $\epsilon>0$.
	Dann sind $A\defeq\menge{c}$, $B=\brackets{B(c,\epsilon)}^C\in\F(\S)\mit A\cap B=∅$.
	Nun wenden wir das Lemma von Uryson an und erhalten die Existenz einer Abbildung $f\colon\S\to[0,1]$ stetig (also $f\in C^b(\S)$) mit der Eigenschaft
	\begin{align*}
		f(x)&=
			\begin{cases}
				0, &\falls x=c\\
				1, &\falls d(x,c)\ge\epsilon
			\end{cases}
		\\
		\follows 0 &\le \P\brackets{d(X_n,X)>\epsilon} \\
		\overset{\text{Vor}}&=
		\P\brackets{d(X_n,c)>\epsilon}\\
		&=\E{\underbrace{\indi_{\menge{ d(X_n,c)\ge\epsilon}}}_{\le f(X_n)}}\\
		&\le \E{f(X_n)}
		\stackrelnew{n\to\infty}{\text{Vor.}}{\longrightarrow}
		\E{\underbrace{f(X)}_{=f(c)=0 \text{ f.\,s.}}}\\
		&=0
	\end{align*}
	Das Einschließ-Kriterium liefert
	\begin{align*}
		\P\brackets{d(X_n,X)\ge\epsilon}\ntoinf 0\qquad\forall\epsilon>0
		&
		\qedhere
	\end{align*}
\end{proof}

\input{section4-2Cramer}

\begin{bemerkungnr}\label{bemerkung4.16} %4.16
\tauf die Forderung, dass $Y$ f.\,s.\ konstant ist, kann \emph{nicht} verzichtet werden.
	Also ist die Schlussfolgerung
	\begin{align*}
		X_n\distto  X\text{ in }\S\und Y_n\distto  Y\text{ in }\S'
		\follows(X_n,Y_n)\distto (X,Y)\text{ in }\S×\S'
	\end{align*}
	ist im Allgemeinen \emph{falsch}!
\end{bemerkungnr}

\begin{korollar}\label{korollar4.17}
	Seien $\S,\S'$ separable metrische Räume und sei $T$ beliebiger metrischer Raum.
	Dann gilt:
	\begin{align*}
		X_n\distto  X\und Y_n\stochto c\mit c\in\S'\text{ Konstante }
		\follows h(X_n,Y_n)\distto  h(X,c)\text{ in }T
	\end{align*}
	wobei $h \colon \S×\S'\to T$ messbar ist mit $\P\brackets{(X,c)\in D_h}=0$
\end{korollar}

\begin{proof}
	Folgt aus \ref{satz4.15CramerSlutsky} und Satz \ref{satz4.10ContinuousMappingTheorem} (CMT).
\end{proof}

\begin{beispiel}\label{beisp4.18}
	\begin{align*}
		X_n\distto  X\text{ in }\R^k\und Y_n\stochto c\text{ in }\R^k
	\end{align*}
	Dann gilt:
	\begin{enumerate}[label=(\arabic*)]
		\item $\begin{aligned}
			X_n+Y_n\distto  X+c\text{ in }\R^k
		\end{aligned}$
		\item $\begin{aligned}
				\scaProd {X_n}{Y_n} \distto \scaProd Xc \text{ in }\R
		\end{aligned}$
		\item \label{it:4.18einDim} Für $k=1$ speziell:
		$\begin{aligned}
			Y_n X_n \distto cX\text{ in } \R
		\end{aligned}$ und für $c=0$:
		$\begin{aligned}
			Y_n· X_n\distto 0\text{ in }\R
		\end{aligned}$
		\item $\begin{aligned}
			\frac{X_n}{Y_n}\distto  \frac{X}{c},&~\falls c\neq0\und\forall n\in\N:Y_n\neq0
		\end{aligned}$
		\item Konstante (= nicht zufällige) $Y_n$, d.\,h.
		\begin{align*}
			Y_n(ω)=c_n\qquad\forallω\inΩ,\forall n\in\N
		\end{align*}
		sind natürlich zugelassen:
		\begin{align*}
			(X_n,Y_n)\distto (X,c)\text{ in }\R^k×\R^k
		\end{align*}
	\end{enumerate}
\end{beispiel}

Es gilt:
\begin{align}\label{eqUnder4.18}\tag{$\ast$}
	(X_n, Y_n) \distto (X,Y)\text { in } \S × \S'
	\follows X_n \distto X \text{ in } \S \und
	Y_n \distto  Y \text{ in } \S'
\end{align}
denn aus dem CMT \ref{satz4.10ContinuousMappingTheorem} folgt
\begin{align*}
	X_n=π_1(X_n,Y_n)\stackrelnew{n\to\infty}{\L}{\longrightarrow}π_1(X,Y)=X,
\end{align*}
da die \begriff{Projektion}
\begin{align*}
	π_1:\S×\S'\to\S,~π_1(x,y)\defeqx\qquad\forall (x,y)\in\S×\S'
\end{align*}
stetig ist. Analog: $Y_n\distto  Y$.

Die Umkehrung in \eqref{eqUnder4.18} gilt im Allgemeinen \emph{nicht}!
Vergleiche Bemerkung \ref{bemerkung4.16}.
Sie gilt tatsächlich, falls $Y$ fast sicher konstant ist, vergleiche Satz \ref{satz4.15CramerSlutsky}.
\paragraph{Ziel} Finde eine andere zusätzliche Forderungen neben der linken Seite von \eqref{eqUnder4.18}, die die umgekehrte Implikation sicherstellt.

Dazu seien $(\S_1,d_1)$ und $(\S_2,d_2)$ zwei separable metrische Räume und $\S \defeq \S_1 × \S_2$ mit  $d \defeq d_1 × d_2$ der zugehörige Produktraum.
(Es folgt, dass $(\S,d)$ separabel ist.)
Eine geringfügige Modifikation des Beweises von Satz \ref{satz3.3} zeigt:
\begin{align*}
	\B_d(\S_1×\S_2)=\B_{d_1}(\S_1)\tens\B_{d_2}(\S_2)
\end{align*}
Für ein Wahrscheinlichkeitsmaß $\P$ auf $\B(\S_1×\S_2)$ definiere
\begin{align*}
	P_1(A_1)&\defeq\P(A_1×\S_2) &\forall A_1\in\B(\S_1)\\
	P_2(A_2)&\defeq\P(\S_1× A_2) &\forall A_2\in\B(\S_2)
\end{align*}
$P_1$ und $P_2$ heißen \begriff{Randverteilungen von $\P$}.
% in der Vorlesung bezeichnungen P, P_1, P_2 (beide = P_i), P_n (\neq P_i). alle ohne \P

\begin{theorem}\label{theorem4.19}
	Sei $(\S,d)$ separabel.
	Dann sind folgende Aussagen äquivalent:
	\begin{enumerate}[label=(\arabic*)]
		\item \label{it:4.19weakto} $\begin{aligned}
			\P_n\stackrelnew{ω}{}{\longrightarrow} \P
		\end{aligned}$
	\item \label{it:4.19randlos} $\begin{aligned}
			\P_n(A_1× A_2)\ntoinf  \P(A_1× A_2)\qquad\forall A_i\in\B(\S_i)~P_i\text{-randlos mit } i=1,2
		\end{aligned}$
	\end{enumerate}
\end{theorem}

\begin{proof}
	Seien $\rand$, $\rand_1$, $\rand_2$ die Randoperatoren in $(\S,d),(\S_1,d_1),(\S_2,d_2)$, respektive.
	\paragraph{Zeige \ref{it:4.19weakto} $\follows$ \ref{it:4.19randlos}:}
	Sei $A\defeqA_1× A_2$ mit $A_i\in\B(\S_i)$ $P_i$-randlos, $i\in\menge{1,2}$.
	Es gilt
	\begin{align}\label{eqProof4.19Stern}\tag{$\ast$}
		\rand A\subseteq((\rand_1 A_1) × \S_2) ∪ (\S_1 × (\rand_2 A_2))
	\end{align}
	Daraus folgt
	\begin{align*}
		0 &\le \P(\rand A)\le \underbrace{\P((\rand A_1) × \S_2)}_{=P_1(\rand_1 A_1)=0}+\underbrace{\P(\S_1×(\rand_2 A_2))}_ {=P_2(\rand_2 A_2)=0}=0\\
		&\follows
		A\in\B(\S_1×\S_2)\text{ ist $\P$-randlos}\\
		&\overset{\ref{satz4.2} \ref{it:4.2BorelSets}}{\follows} \ref{it:4.19randlos}
	\end{align*}

	\paragraph{Zeige \ref{it:4.19randlos}$\follows$ \ref{it:4.19weakto}:}
	Wir wollen Korollar \ref{korollar4.4} anwenden. Dazu:

	Wähle $d$ wie folgt aus:
	\begin{align}\label{eqProof4.19Plus}\tag{+}
		d\brackets{(x_1,x_2),(y_1,y_2)}&\defeq\max\menge{ d_1(x_1,y_1),d_2(x_2,y_2)}
	\end{align}
	Sei
	\begin{align*}
	\U \defeq \menge{A_1× A_2:A_i\in\B(\S_i)~P_i\text{-randlos}, i\in \menge{1,2} }
	\end{align*}
	$\U$ ist durchschnittsstabil, denn für
	$A = A_1 × A_2$, $B = B_1 × B_2 \in \U$
	gilt
	\begin{align}\label{eqProof4.19.1}\tag{1}
		A \cap B = (A_1 \cap B_1) × (A_2 \cap B_2) \in \B(S_1) \tens \B(S_2)
	\end{align}
	Ferner gilt
	\begin{align}\label{eqProof4.19.2}\tag{2}
		\rand_i(A_i\cap B_i) &\subseteq (\partial_i A_i)∪(\partial_i B_i) & \forall i\in\menge{1,2} \\
		\follows 0 \le P_i(\rand(A_i \cap B_i)) &\le P_i(\rand A_i) + P_i(\rand B_i) \le 0 + 0 & \foralli \in \menge{1,2} \nonumber
	\end{align}
\taus \eqref{eqProof4.19.1} und \eqref{eqProof4.19.2} folgt:
	$\U$ ist durchschnittsstabil.

	Sei nun $x=(x_1,x_2)\in\S=\S_1×\S_2$ und $\epsilon>0$ beliebig.
\taus \eqref{eqProof4.19Plus} folgt:
	\begin{align*}
		B_d(x,\epsilon)=B_{d_1}(x_1,\epsilon)× B_{d_2}(x_2,\epsilon)
	\end{align*}
	Sei
	\begin{align*}
		A_δ\defeqB_{d_1}(x_1,δ)× B_{d_2}(x_2,δ)
		\stackeq{\eqref{eqProof4.19Plus}} B_d(x,δ)\qquad\forallδ>0
	\end{align*}
	Hierbei ist die Schwierigkeit, dass nicht sichergestellt ist, dass
	die $B_{d_i}(x_i, δ)$ $P_i$-randlos sind.

	Es gilt (vgl.\ Beweis von \ref{satz4.2} Portmanteau-Theorem)
	\begin{align*}
		\rand_i B_{d_i}(x_i,δ)\subseteq\menge{ y\in\S_i:d_i(y,x_i)=δ}\qquad\forall i\in\menge{1,2}.
	\end{align*}
	Folglich sind die Mengen $\rand_i B_{d_i}(x_i,δ),δ>0$ sind paarweise disjunkt.
	Die Mengen
	\begin{align*}
		E_i\defeq\menge{δ>0:P_i\brackets{\rand_i B_{d_i}(x_i,δ)}>0}\qquad\forall i\in\menge{1,2}
	\end{align*}
	sind höchstens abzählbar (vgl. Beweis von \ref{satz4.2}).
	Folglich ist die Vereinigung $E_1∪ E_2$ höchstens abzählbar und somit liegt $(E_1∪ E_2)^C$ dicht in $[0,\infty)$.
	Mit
	\begin{align*}
		(E_1 ∪ E_2)^C &= E_1^C \cap E_2^C
		= \menge{ δ>0 : B_{d_i}(x_i,δ) \text{ ist $P_i$-randlos}, i\in\menge{1,2} }\\
		\follows \exists δ &\in (0,\epsilon) : B_{d_1}(x_1,δ) ~ P_1 \text{-randlos und }B_{d_2}(x_2,δ) ~ P_2\text{-randlos}\\
		\follows A_δ&=B_{d_1}(x_1,δ)× B_{d_2}(x_2,δ)\in\U\\
		\overset{\eqref{eqProof4.19Plus}}&=
		B_d(x,δ)\subseteq B_d(x,\epsilon)
	\end{align*}
	Als offene Kugel ist $A_δ$ offen, also $\inner{A_δ}=A_δ$.
	Schließlich folgt
	\begin{align*}
		x\in\inner{A_δ}=A_δ\subseteq B_d(x,\epsilon)
	\end{align*}
	Also erfüllt $\U$ die Voraussetzungen von Korollar \ref{korollar4.4}.
	Daraus folgt nun die Behauptung mit der Annahme \ref{it:4.19weakto}.
\end{proof}

Theorem \ref{theorem4.19} liefert ein nützliches Resultat für Produktmaße.
Seien $\P^{(i)},\P_n^{(i)},n\in\N$ Wahrscheinlichkeitsmaße auf $\B(\S_i)$ mit $i\in\menge{1,2}$.
Dann sind
\begin{align*}
	\P\defeq\P^{(1)}\tens\P^{(2)},\qquad\P_n\defeq\P_n^{(1)}\otimes\P_n^{(2)}\qquad\forall n\in\N
\end{align*}
Produktmaße auf $\B(\S_1)\tens\B(\S_2)=\B(\S_1×\S_2)$ bei Separabilität.
Produktmaße sind beispielsweise für die Charakterisierung von Unabhängigkeit
von Zufallsgrößen relevant:
\begin{align*}
	X, Y \text{ unabhängig} ⇔ \P ∘ (X, Y)^{-1} = \P ∘ X^{-1} \tens \P ∘ Y^{-1}.
\end{align*}

\begin{theorem}\label{thoerem4.20}
	Sei $\S=\S_1×\S_2$ separabler Produktraum. Dann sind äquivalent:
	\begin{enumerate}[label=(\arabic*)]
		\item \label{it:4.20product} $\begin{aligned}
			\P_n\weakto \P
		\end{aligned}$, also
		$\brackets{\P_n^{(1)} \tens \P_n^{(2)}} \weakto \P^{(1)} \tens \P^{(2)}$
		\item \label{it:4.20einzeln} $\begin{aligned}
			\P_n^{(1)}\weakto \P^{(1)}\und
			\P_n^{(2)}\weakto \P^{(2)}
		\end{aligned}$
	\end{enumerate}
\end{theorem}

\begin{proof}
	Gemäß Definition gilt:
	\begin{align*}
		\P(A_1× A_2)&\stackeq{\text{Def}}\P^{(1)}(A_1)·\P^{(2)}(A_2)\\
		\P_n(A_1× A_2)&\stackeq{\text{Def}}\P_n^{(1)}(A_1)·\P_n^{(2)}(A_2)\\
	\end{align*}
	\paragraph{Zeige \ref{it:4.20product} $\follows$ \ref{it:4.20einzeln}:}
	Setze $A_2 = \S_2$ und erhalte
	\begin{align*}
		\P_n^{(1)}(A_1)&=\P_n(A_1×\S_2) \overset{\ref{theorem4.19}}{\longrightarrow} \P(A_1×\S_2)=\P^{(1)}(A_1)
		\qquad\forall A_1\in\B(\S_1)~\P^{(1)}\text{-randlos}
	\end{align*}
	gemäß Theorem \ref{theorem4.19}, denn $\rand\S_2=∅$.
	Also ist $\S_2\in\B(\S_2)~\P^{(2)}$-randlos.
	Also folgt aus Satz \ref{satz4.2} \ref{it:4.2BorelSets}:
	\begin{align*}
		\P_n^{(1)}\weakto \P_n^{(2)}
	\end{align*}
	Analog zeigt man:
	$\P_n^{(2)}\weakto \P^{(2)}$.
	\paragraph{Zeige \ref{it:4.20einzeln} $\follows$ \ref{it:4.20product}:}
	\begin{align*}
		&\P_n(A_1 × A_2)
		= \underbrace{\P_n^{(1)}(A_1)}_{\ntoinf \P^{(1)}(A_1)}
		· \underbrace{\P_n^{(2)}(A_2)}_{\ntoinf \P^{(2)}(A_2)}
		\qquad \forall A_i ~ \P_i \text{-randlos}, i=1, 2\\
		&\follows
		\P^{(1)}(A_1) · \P^{(2)}(A_2)
		= \P(A_1 × A_2) \qquad \forall A_i ~ \P_i \text{-randlos}, i\in\menge{1, 2}\\
		&\overset{\ref{theorem4.19}}{\follows}\P_n \ntoinf \P
	\end{align*}
\end{proof}

Mit Theorem \ref{thoerem4.20} erhalten wir neben Satz \ref{satz4.15CramerSlutsky} ein weiteres
Resultat über die \begriff{ \enquote{koordinatenweise Verteilungskonvergenz}}
(die ja im Allgemeinen nicht gilt).

\begin{satz}[Koordinatenweise Konvergenz bei Unabhängigkeit]\label{satz4.21}
	Sei $\S=\S_1×\S_2$ separabel, $X,X_n,n\in\N$ Zufallsvariablen in $\S_1$, $Y,Y_n,n\in\N$ Zufallsvariablen in $\S_2$ über $(Ω,\A,\P)$ und gelte
	\begin{enumerate}
		\item $X_n$ und $Y_n$ sind unabhängig für alle $n\in\N$
		\item $X$ und $Y$ sind unabhängig
	\end{enumerate}
	Dann gilt:
	\begin{align*}
		X_n\distto  X\text{ in }\S_1\und Y_n\distto  Y\text{ in }\S_2
		⇔ (X_n,Y_n)\distto (X,Y)\text{ in }\S_1×\S_2
	\end{align*}
\end{satz}

\begin{proof}
	Da
	\begin{align*}
		\P∘ (X_n,Y_n)^{-1}&\stackeq{\text{unab}}\P∘ X_n^{-1}\tens\P∘ Y_n^{-1}\\
		\P∘ (X,Y)^{-1}&\stackeq{\text{unab}}\P∘ X^{-1}\tens\P∘ Y^{-1}
	\end{align*}
	folgt die Behauptung aus Theorem \ref{thoerem4.20} und Definition \ref{def4.1}.
\end{proof}

\begin{bemerkungnr}\label{bemerkung4.22}
	Die Aussage in Satz \ref{satz4.21} lässt sich problemlos auf $\S_1×\dots×\S_d$ mit $d\ge2$ übertragen.
\end{bemerkungnr}

\begin{satz}[Cramér]\label{satz4.14Cramer}
	Seien $(X_n)_{n\in\N},(Y_n)_{n\in\N}$ zwei Folgen in separablem\footnote{wird gebraucht, damit $d(X_n, Y_n)$ eine Zufallsvariable ist} metrischen Raum $(\S,d)$, die \begriff{stochastisch äquivalent sind}, d.\,h.
	\begin{align}\label{eq4.14StochastischAquivalent}\tag{Ä}
		d(X_n,Y_n) \stochto 0
	\end{align}
	Dann gilt:
	\begin{align*}
		X_n \distto X ⇔ Y_n \distto  X
	\end{align*}
\end{satz}

\begin{proof}
	Sei $f\in C^b(\S)$ gleichmäßig stetig, d.\,h.
	\begin{align*}
		\forall\epsilon>0:
		\existsδ_\epsilon>0:
		\forall x,y\in\S:\quad
		d(x,y) \le δ_\epsilon \follows \abs{f(x)-f(y)} \le \epsilon
	\end{align*}
	Damit folgt
	\begin{align*}
		&\abs{\E{f(X_n)}-\E{f(Y_n)}}\\
		\overset{\text{Lin}}&=
		\abs{\int f(X_n)-f(Y_n)\d\P}\\
		\overset{Δ-\neq}&{\le}
		\int\abs{f(X_n)-f(Y_n)}\d\P\\
		\overset{\text{Lin}}&=
		\int\underbrace{\indi_{\menge{ d(X_n,Y_n) \le δ_{\epsilon} }}·\abs{f(X_n)-f(Y_n)}}_{\le \epsilon\text{ wegen glm.\ Stetigkeit}} \d \P\\
		&\qquad+
		\int\indi_{\menge{ d(X_n,Y_n)>δ_{\epsilon}}}·\underbrace{\abs{f(X_n)-f(Y_n)}}_{\le\abs{f(X_n)}+\abs{f(Y_n)}\le2·\norm{ f}_\infty}\d\P\\
		&\le \epsilon + 2·\norm{ f}_\infty·\underbrace{\P\brackets{d(X_n,Y_n)>δ_\epsilon}}_{\ntoinf 0}\\
		\follows
		0 &\le \liminf_{n\to\infty}\abs{\E{f(X_n)}-\E{f(Y_n)}}\\
			&\le\limsup_{n\to\infty} \abs{\E{f(X_n)}-\E{f(Y_n)} }\\
		& \le \epsilon + 2·\norm{ f}_\infty· 0 \\
		&=\epsilon\qquad\forall\epsilon>0\\
		&\overset{\epsilon\to0}{\follows}
		\E{f(X_n)}-\E{f(Y_n)}\ntoinf  0\qquad\forall f\in C^b(\S)\text{ glm. stetig}
	\end{align*}
	Es folgt:
	\begin{align*}
		X_n\distto  X
		&\overset{\ref{satz4.7}}{⇔}
		\E{f(X_n)} \to \E{f(X)}\qquad\forall f\in C^b(\S)\text{ glm. stetig}\\
		&⇔
		% original:
		% \underbrace{\E{f(X_n)}}_{=\E{f(X_n)}+\Big(\E{f(Y_n)}-\E{ f(X_n)}\Big)\overset{\text{oben}}{\longrightarrow}0}\overset{}{\longrightarrow}\E{f(X)}~\forall f\in C^b(\S)\text{ glm. stetig}\\
		\E{f(X_n)}
		+\underbrace{\brackets{\E{f(Y_n)}-\E{ f(X_n)}}}_{\overset{\text{oben}}{\longrightarrow}0} \to \E{f(X)} \\
		& \qquad \qquad \qquad \qquad \forall f\in C^b(\S)\text{ glm. stetig}\\
		&\overset{\ref{satz4.7}}{⇔}
		Y_n\distto  X
		\qedhere
	\end{align*}
\end{proof}

% \pagebreak[4]
\begin{satz}[Cramér-Slutsky]\label{satz4.15CramerSlutsky}
	Seien $(\S,d)$, $(\S',d')$ separable metrische Räume.
	Dann gilt:
	\begin{align*}
		\left.\begin{array}{l}
		% CHECKED: '\left' used.
			X_n\distto  X\text{ in }(\S,d)\\
			Y_n\distto  Y\text{ in }(\S',d')\\
			Y\text{ f.\,s.\ konstant}
		\end{array}\right\rbrace
		% CHECKED: '\right', \rbrace used.
		\follows (X_n,Y_n)\distto (X,Y)\text{ in }(\S×\S',d× d')
	\end{align*}
\end{satz}

\begin{proof}
	Nach Voraussetzung existiert ein $c\in\S'$ mit $\P(Y=c)=1$.
	Mit Satz \ref{satz4.13} gilt:
	\begin{align}\label{eqProof4.15Stern}\tag{$\ast$}
		Y_n \stochto  c
	\end{align}
	Ferner:
	\begin{align*}
		&d× d'\brackets{(X_n,Y_n),(X_n,c)}
		=\underbrace{d(X_n,X_n)}_{=0}+d'(Y_n,c)
		=d'(Y_n,c)\\
		&\follows\P\brackets{d× d'\brackets{(X_n,Y_n),(X_n,c)}>\epsilon}
		=\P\brackets{d'(Y_n,c)>\epsilon}
		\ntoinf  0\quad\forall\epsilon>0\text{ wg. } \eqref{eqProof4.15Stern}\\
		&\follows
		\brackets{(X_n,Y_n)}_{n\in\N}\text{ und }\brackets{(X_n,c)}_{n\in\N}\text{ sind stochastisch äquivalent.}
	\end{align*}
	Gemäß Satz \ref{satz4.14Cramer} reicht es zu zeigen, dass
	\begin{align}\label{eqProof4.15Plus}\tag{+}
		(X_n,c)\distto (X,c)\stackrelnew{\L}{\text{f.\,s.}}{=}(X,Y)\text{ in }(\S×\S',d× d')
	\end{align}
	gilt.
	Dazu sei $f\in C^b(\S×\S')$ beliebig und $f_c \colon \S \to \R$, $f_c(x)\defeqf(x,c)$.
	Damit folgt $f_c\in C^b(\S)$.
	Somit:
	\begin{align*}
		\E{f(X_n,c)}
		&\stackeq{\text{Def}}
		\E{f_c(X_n)}
		\ntoinf \E{f_c(X)}\stackeq{\text{Def}}
		\E{f(X,c)}
		\overset{\text{Def}}{\follows}
		\eqref{eqProof4.15Plus}
		&\qedhere
	\end{align*}
\end{proof}

