% !TEX root = MSTAT19.tex
% This work is licensed under the Creative Commons
% Attribution-NonCommercial-ShareAlike 4.0 International License. To view a copy
% of this license, visit http://creativecommons.org/licenses/by-nc-sa/4.0/ or
% send a letter to Creative Commons, PO Box 1866, Mountain View, CA 94042, USA.

\subsection{Anwendung}

Sei $X_1,…,X_n$ \iid\ Kopien von $X$ mit $μ$-Dichte
\begin{align*}
	f_θ(x)&=c(θ)· h(x)·\exp\klammern[\big]{q(θ)· T(x)}\qquad∀ x∈\X,∀θ∈Θ=ℝ
\end{align*}
Sei $θ_0$ der wahre Parameter.
Dann gilt:
%
\begin{align}\label{eqAnwendung_1}\tag{1}
	\text{MLS $\hat{θ}_n$ für $θ_0$ ist }\hat{θ}_n&=\argmin_{t∈ℝ}M_n(t)\qquad\mit\\
	M_n(t) &= -\log \argu[\big]{c(t)} - q(t) · \overline{T}_n, \qquad
	\overline{T}_n = \frac{1}{n} · \sum_{i=1}^n T(X_i) \nonumber
\end{align}
Ziel: Nachweis von
\begin{align*}
	√{n}·\klammern[\big]{\hat{θ}_n-θ_0}\distrto ζ
\end{align*}
und Identifizierung von $ζ$.

Dazu: Sei
\begin{align*}
	Z_n(t):=n·\klammern{M_n \argu{θ_0 + \frac{t}{√{n}}} - M_n\argu[\big]{θ_0}} \qquad ∀ t ∈ ℝ
\end{align*}
der \define{reskalierte Prozess} zu $M_n$.
Offenbar ist $Z_n$ stetiger stochastischer Prozess \emph{und}
\begin{align}\label{eqAnwendung_2}\tag{2}
	\underbrace{√{n}·\klammern[\big]{\hat{θ}_n-θ_0}}_{=:σ_n}
	&=\argmin_{t∈ℝ} Z_n(t)
\end{align}
denn:
Zu zeigen:
\begin{align*}
	Z_n(σ_n)≤ Z_n(t)\qquad∀ t∈ℝ
\end{align*}
Dazu nutze Äquivalenzen:
\begin{align*}
	&Z_n(σ_n)≤ Z_n(t) &∀& t∈ℝ\\
	\overset{t=σ_n}%
	&{⇔}
	n \klammern[\bigg]{M_n· \argu[\bigg]{\underbrace{θ_0+\frac{σ_n}{√{n}}}_{
		\begin{subarray}{c}		
			=θ_0+\frac{√{n}\klammern[\big]{\hat{θ}_n-θ_n}}{√{n}}\\
			=θ_n
		\end{subarray}
	}} - M_n(θ_0) }
	≤ n·\klammern{M_n \argu{θ_0 + \frac{t}{√{n}}} - M_n(θ_0)}
	& ∀ & t ∈ ℝ \\
	&⇔ n·\klammern[\Big]{M_n\argu[\big]{\hat{θ}_n} - M_n\argu[\big]{θ_0} }
	≤ n· \klammern{M_n \argu{θ_0 + \frac{t}{√{n}} } - M_n(θ_0) } & ∀ & t ∈ ℝ \\
	&⇔ M_n \argu[\big]{\hat{θ}_n}
	≤ M_n \argu{θ_0+\frac{t}{√{n}}} &∀&t∈ℝ\\
	&⇔ M_n \argu[\big]{\hat{θ}_n}
	≤ M_n \argu{θ_0+\frac{t}{√{n}}} &∀& t∈ℝ\\
	&⇔ M_n \argu[\big]{\hat{θ}_n}≤ M_n(u) &∀& u∈ℝ\\
	\overset{\eqref{eqAnwendung_1}}&{⇔} \text{ Wahr}
\end{align*}
Wegen \eqref{eqAnwendung_2} versuche Satz \ref{satz8.8} anzuwenden:

Man kann zeigen:
\begin{align}\label{eqAnwendung_i}\tag{i}
	Z_n\distrto  Z\text{ in }\klammern[\big]{C(ℝ),d}
\end{align}
wobei
\begin{align*}
	Z(t)
	&=-q'(θ_0)· N· t+\frac{1}{2}·\klammern[\big]{q'(θ_0)}^2·σ^2(θ_0)· t^2 \qquad∀ t∈ℝ
\end{align*}
wobei wir fordern:
\begin{align*}
	σ^2(θ_0):=\Var_{θ_0}\klammern[\big]{T(X)}\overset{\text{Forderung}}&>0\\
	q'(θ_0)\overset{\text{Forderung}}&\neq0\\
	N &\sim \Nor \argu[\big]{0,σ^2(θ_0)}
\end{align*}
Also ist $Z$ eine zufällige nach oben geöffnete Parabel.
\begin{align*}
	0
	&=Z'(t)
	=-q'(θ_0)· N+\klammern[\big]{q'(θ_0)}^2·σ^2(θ_0)· t\\
	ζ
	&=\frac{1}{q'(θ_0)·σ^2(θ_0}· N
	\text{ ist die eindeutige Minimalstelle von }Z.
\end{align*}
Man kann zeigen:
\begin{align*}
	\limsup_{n⟶∞}\P \argu{√{n}·\abs[\big]{\hat{θ}_n-θ_0}≥ j}
	≤\frac{4}{\klammern[\big]{q'(θ_0)}^2·σ^2(θ_0)}·\frac{1}{j^2}\qquad∀ n∈ℕ
\end{align*}
Grenzübergang $j⟶∞$ liefert, dass zweite Bedingung \eqref{eqSatz8.9_2} in Satz \ref{satz8.8} erfüllt ist.
Somit liefert Satz \ref{satz8.8}:
\begin{align*}
	√{n}·\klammern[\big]{\hat{θ}_n-θ_0}\distrto ζ
\end{align*}
