% This work is licensed under the Creative Commons
% Attribution-NonCommercial-ShareAlike 4.0 International License. To view a copy
% of this license, visit http://creativecommons.org/licenses/by-nc-sa/4.0/ or
% send a letter to Creative Commons, PO Box 1866, Mountain View, CA 94042, USA.

\subsection{Das Continuous Mapping Theorem (CMT)}
Sei $h\colon(\S,d)\to(\S',d')$ messbar.
\paragraph{Ziel:} Finde Bedingungen an $h$, so dass gilt:
\begin{enumerate}[label=(\arabic*)]
	\item \label{it:4.1conditionhweakto} $\begin{aligned}
		\P_n\weakto \P\follows \P_n∘ h^{-1}\weakto  \P∘ h^{-1}
	\end{aligned}$
	\item \label{it:4.1conditionhDistrto} $\begin{aligned}
		X_n\distto  X\text{ in }(\S,d)\follows h(X_n)\distto  h(X)\text{ in }(\S',d')
	\end{aligned}$
\end{enumerate}
Beachte:
\begin{align*}
	\P∘\brackets{h(X_n)}^{-1}
	&=\P∘\brackets{h∘ X_n}^{-1}
	=\brackets{\P∘ X_n^{-1}}∘ h^{-1}\\
	\P\brackets{h(X)}^{-1}
	&=\brackets{\P∘ X^{-1}}∘ h^{-1}
\end{align*}
Also folgt \ref{it:4.1conditionhDistrto} aus \ref{it:4.1conditionhweakto}. Zunächst gilt \ref{it:4.1conditionhweakto}, wenn $h$ stetig auf $\S$  ist, denn: Sei $f\in C^b(\S')$ beliebig. Dann gilt:
\begin{align*}
	\int_{\S'} f\d \brackets{\P_n∘ h^{-1}}
	\stackeq{\eqref{eqTrafo}}
	\int_{\S} \underbrace{ f∘ h}_{\in C^b(\S)}\d \P_n
	\overset{\ref{def4.1}}&{\longrightarrow}\int f∘ h\;\d \P
	\stackeq{\eqref{eqTrafo}}
	\int f\d(\P∘ h^{-1})\\
	\overset{\ref{def4.1}}{\follows}
	\P_n∘ h^{-1} &\weakto  \P∘ h^{-1}
\end{align*}
Auf Stetigkeit von $h$ kann i.\,A.\ nicht verzichtet werden, denn es gilt:%
%
% Nebenbeibemerkung: Prof. Ferger ist bei research gate angemeldet, das Facebook der Wissenschaftler: dort bekommt er etwas Rückmeldung zu den Papern. Bei üblichen Veröffentlichungen gibt es das nicht. Schade.
%
\begin{beispiel}\label{beisiel4.8}
	% Beweis unvollständig, als Übung belassen, hier vollständig
	Sei $\S=\S'=[0,1]$, $h\colon[0,1]\to[0,1]$ mit
	\begin{align*}
		h(x)\defeq
		\begin{cases}
			1, &\falls x\in\menge{0} ∪ \menge{\frac{1}{2· n}:n\in\N}\\
			0, & \sonst
		\end{cases}
	\end{align*}
	Sei $P_n\defeqδ_{\frac{1}{n}}$, $n\in\N$, $\P\defeqδ_{0}$ wobei $δ_x$ das Dirac-Maß bezeichnet.
	Dann gilt $\P_n\weakto  \P$, denn
	\begin{align*}
	\int f\d \P_n&=f\brackets{\frac{1}{n}}\ntoinf  f(0)=\int f\d \P\qquad\forall f\in C^b\brackets{\intervall01}
	\end{align*}
	Aber wegen
	\begin{align*}
		δ_x∘ h^{-1}=δ_{h(x)}
	\end{align*}
	gilt für
	\begin{align*}
		Q_n\defeq\P_n∘ h^{-1}=δ_{\frac{1}{n}}∘ h^{-1}=δ_{h\brackets{\frac{1}{n}}},\qquad Q\defeq\P∘ h^{-1}=δ_1
	\end{align*}
	das Folgende:
	\begin{align*}
		\int f\d Q_n=f\brackets{h\brackets{\frac{1}{n}}}=
		\begin{cases}
			f(1), & \falls n\text{ gerade }\\
			0, & \falls n\text{ ungerade }
		\end{cases}
	\end{align*}
	Sei $f\in C^b\brackets{\intervall01}$ mit $f(0)\neq f(1)$ (z.\,B.\ $f=\id$).
	Folglich ist die Folge $\brackets{\int f\d Q_n}_{n\in\N}$ divergent.
	Somit:
	\begin{align*}
		\follows\int f\d Q_n\nrightarrow\int f\d Q\follows Q_n\stackrelnew{w}{}{\nrightarrow} Q
	\end{align*}
\end{beispiel}

Die Forderung der Stetigkeit lässt sich aber abschwächen so, dass \ref{it:4.1conditionhweakto} noch gilt.
Dazu definiere die \begriff{Menge der Unstetigkeitsstellen}
\begin{align*}
	D_h\defeq\menge{ x\in\S: h\text{ \emph{nicht} stetig in }x}.
\end{align*}

\begin{lemma}\label{lemma4.9}
	\begin{align*}
		D_h\in\B(\S)\qquad\forall h\colon \S\to\S'\text{ beliebig (nicht einmal messbar)}
	\end{align*}
\end{lemma}

\begin{proof}\footnote{Findet sich im Appendix von \cite{billingsley2013convergence}.}
	Mit der Dreiecksungleichung überlegt man sich leicht:
	\begin{align*}
		h\text{ stetig in }x
		⇔
		&\forall 0<\epsilon\inℚ:\exists 0<δ\inℚ:\forall y,z\in\S:\\
		&d(x,y)<δ\und d(x,z)<δ
		\follows d'\brackets{h(y),h(z)}<\epsilon
	\end{align*}
	Damit folgt:
	\begin{align}\label{eqProof4.9Plus}\tag{+}
		D_h=\bigcup_{0<\epsilon\inℚ}\bigcap_{0<δ\inℚ}\underbrace{\menge{x\in\S:
		\exists y,z\in\S\mit d(x,y)<δ\und d(x,y)<δ\und d'\brackets{h(y),h(z)}\ge\epsilon}}_{=:A_{\epsilon,δ}}
	\end{align}
	Zeige
	\begin{align}\label{eqProof4.9Stern}\tag{$\ast$}
		A_{\epsilon,δ}\in\G(\S)\qquad\forall\epsilon,δ>0
	\end{align}
	Dazu sei $x_0\in A_{\epsilon,δ}$.
	Dann existieren $y,z\in\S$ mit $d(x_0,y)<δ$ und $d(z_0,z)<δ$, aber $d'\brackets{h(y),h(z)}\ge\epsilon$.
	Wähle
	\begin{align*}
		δ_0\defeq\min\menge{δ- d(x_0,y),δ-d(x_0,z)}>0
	\end{align*}
	Ferner gilt:
	\begin{align}\label{eqProof4.9i}\tag{i}
		B(x_0,δ_0)\subseteq A_{\epsilon,δ},
	\end{align}
	denn: Sei $x\in B(x_0,δ_0)$.
	Dann gilt
	\begin{align*}
		d(x,y)\le d(x,x_0)+d(x_0,y)<δ_0+d(x_0,y)<δ
	\end{align*}
	und analog
	\begin{align*}
		d(x,z)<δ.
	\end{align*}
	Also folgt $x\in A_{\epsilon,δ}$, denn $d'\brackets{(h(y),h(z)}\ge\epsilon$.
	Wegen \eqref{eqProof4.9i} ist $x_0$ innerer Punkt von $A_{\epsilon,δ}$.
	Also folgt \eqref{eqProof4.9Stern} und mit \eqref{eqProof4.9Plus} dann die Behauptung.
\end{proof}

\begin{satz}[Continuous Mapping Theorem (CMT)]\enter\label{satz4.10ContinuousMappingTheorem}
	Sei $h\colon(\S,d)\to(\S',d')$ $\B_d(\S)$-$\B_{d'}(\S')$-messbar.
	% hier ist D_h BDSM (B_d(\S) messbar)
	Dann gilt:
	\begin{enumerate}[label=(\arabic*)]
		\item \label{it:4.10CMTweakto} $\begin{aligned}
			\P_n\weakto  \P\und \P(D_h)=0
			\follows \P_n∘ h^{-1}\weakto  \P∘ h^{-1}
		\end{aligned}$
	\item \label{it:4.10CMTdistrto} $\begin{aligned}
			X_n\distto X\text{ in }(\S,d)\und\P(X\in D_h)=0
			\follows h(X_n)\distto  h(X)\text{ in }(\S',d')
		\end{aligned}$
	\item \label{it:4.10CMThcontinous} Sei $h$ stetig. Dann gilt: $X_n\distto X\text{ in }(\S,d)\follows h(X_n)\distto  h(X)\text{ in }(\S',d')$
	\end{enumerate}
\end{satz}

\begin{proof}
	\footnote{Der Beweis ist eine beliebte Prüfungsfrage laut Prof.\ Ferger.}
	Wegen des Portmanteau-Theorems \ref{satz4.2} zeigen wir die äquivalente Aussage über $\limsup$.
	\paragraph{Zeige \ref{it:4.10CMTweakto}}
	Sei $F\in\F(\S')$ (d.\,h.\ $F\subseteq\S'$ abgeschlossen) beliebig.
	Dann gilt:
	\begin{align}\label{eqProof1.4.10(i)}\tag{i}
		\limsup_{n\to\infty} \P_n∘ h^{-1}(F)
		&=\limsup_{n\to\infty} \P_n\brackets{\underbrace{h^{-1}(F)}_{\subseteq \abschluss{h^{-1}(F)}}}\\\nonumber
		% CHECKED: '\abschluss' used.
		&\le \limsup_{n\to\infty} \P_n\brackets{\underbrace{\abschluss{h^{-1}(F)}}_{\in\F(\S)}}\\\nonumber
		% CHECKED: '\abschluss' used.
		\overset{\ref{satz4.2}}&{\le}
		\P\brackets{\abschluss{h^{-1}(F)}}
		% CHECKED: '\abschluss' used.
	\end{align}
	Es gilt
	\begin{align}\label{eqProof1.4.10(ii)}\tag{$*$}
		\abschluss{h^{-1}(F)} \subseteq h^{-1}(F) ∪ D_h,
		% CHECKED: '\abschluss' used.
	\end{align}
	denn: Sei $x\in\quer{h^{-1}(F)}$.
	\subparagraph{Fall 1: $x\in D_h$}
	\begin{align*}
		x\in D_h\follows x\in h^{-1}(F)∪ D_h
	\end{align*}
	\subparagraph{Fall 2: $x\not\in D_h$}
	Also ist %$x\in C_h$, d.\,h.\ %
	$h$ %ist
	stetig in $x$.
	Also gilt für alle Folgen $(x_n)_{n\in\N}\subseteq h^{-1}(F)\subseteq\S$ mit $x_n\ntoinf x$ folgendes
	(eine solche Folge existiert nach Definition des Abschlusses):
	%Ferner existiert eine Folge $(x_n)_{n\in\N}\subseteq h^{-1}(F)$ mit $x_n\ntoinf  x$.
	%Wegen der Stetigkeit gilt
	\begin{align*}
		\underbrace{h(x_n)}_{\in F~\forall n}\ntoinf  h(x)
		\follows h(x)\in\abschluss{F}=F\follows x\in h^{-1}(F)
		% CHECKED: '\abschluss' used.
		\follows x\in h^{-1}(F)∪ D_h
	\end{align*}
	Mit \eqref{eqProof1.4.10(ii)} setzt sich die Ungleichungskette \eqref{eqProof1.4.10(i)} fort:
	\begin{align*}
		\P\brackets{\quer{h^{-1}(F)}}
		\overset{\eqref{eqProof1.4.10(ii)}}{\le}
		\P\brackets{h^{-1}(F)∪ D_h}
		\le \P\brackets{h^{-1}(F)}+\underbrace{\P(D_h)}_{\overset{\Vor}{=}0}
		=\P∘ h^{-1}(F)\\
		\overset{\eqref{eqProof1.4.10(i)},\ref{satz4.2} \ref{it:4.2ClosedSets}}{\follows}
		\P_n∘ h^{-1}\weakto  \P∘ h^{-1}
	\end{align*}
	\paragraph{Zeige \ref{it:4.10CMTdistrto}:} folgt direkt aus \ref{it:4.10CMTweakto} nach Definition \ref{def4.1}:
	\begin{align*}
		\P ∘ (h(X_n))^{-1} = \P∘ (h ∘ X_n)^{-1} = (\P ∘ X_n^{-1}) ∘ h^{-1}
		&\weakto \P ∘ X^{-1} \quad \text{ (nach Vor.)} \\
		&\weakto (\P ∘ X^{-1}) ∘ h^{-1} = \P ∘ (h(X))^{-1} \\
		\overset{\text{Def}}{\follows} h(X_n) &\distto h(X)
	\end{align*}
	\paragraph{Zeige \ref{it:4.10CMThcontinous}:} folgt direkt aus \ref{it:4.10CMTdistrto}.
\end{proof}

\begin{bemerkung}[Choque-Kapazität]  \label{note:choque_kapazitat}
  Es gibt eine Verallgemeinerung des Wahrscheinlichkeitsbegriffs,
	nämlich die so genannte \begriff{Choque-Kapazität}. Auf der
	Menge alle Choque-Kapazitätäten lässt sich eine Topologie $\tau_{\text{weak}}$
	einführen (die so genannte \emph{schwache Topologie}) und ein
	entsprechendenes Portmanteau-Theorems beweisen, vgl.
	\enquote{Dietmar Ferger 2018 Weak convergence of probability measures to a Choquet-capacity, Turkish Journal of Mathematics}
	% TODO: ordentliche Zitation
	und
	\enquote{Dietmar Ferger (2019) unveröffentlicht}
\end{bemerkung} % end note Choque-Kapazität
