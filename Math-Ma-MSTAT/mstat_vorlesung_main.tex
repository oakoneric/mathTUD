% This work is licensed under the Creative Commons
% Attribution-NonCommercial-ShareAlike 4.0 International License. To view a copy
% of this license, visit http://creativecommons.org/licenses/by-nc-sa/4.0/ or
% send a letter to Creative Commons, PO Box 1866, Mountain View, CA 94042, USA.
% (c) Eric Kunze, 2020
%%%%%%%%%%%%%%%%%%%%%%%%%%%%%%%%%%%%%%%%%%%%%%%%%%%%%%%%%%%%%%%%%%%%%%%%%%%%%%%
% Template for lecture notes at TU Dresden.
%%%%%%%%%%%%%%%%%%%%%%%%%%%%%%%%%%%%%%%%%%%%%%%%%%%%%%%%%%%%%%%%%%%%%%%%%%%%%%%

\documentclass[ngerman, a4paper, 11pt]{report}

\usepackage[ngerman]{babel}
\usepackage[top=2.5cm,bottom=2.5cm,left=2.5cm,right=2.5cm]{geometry}

\usepackage{parskip}    % split paragraphs by vspace instead of intendations
\usepackage[onehalfspacing]{setspace} % increase row-space

\usepackage{xifthen}

%%%%%%%%%%%%%%%%%%%%%%%%%%%%%%%%%%%%%%%%%%%%%%%%%%%%%%%%%%%%%%%%%%%%%%%%%%%%%%%
% SCHRIFTEN
\usepackage[utf8]{inputenc}
\usepackage{chngcntr}
\usepackage{eufrak}

\usepackage{lmodern}
\usepackage[normalem]{ulem}

\usepackage[autostyle=true,english=british]{csquotes}

% new font OpenSans
\usepackage[scale=1]{opensans}
\newcommand*{\osfamily}{\fontfamily{fos}\selectfont}
\DeclareTextFontCommand{\textos}{\osfamily}

%%%%%%%%%%%%%%%%%%%%%%%%%%%%%%%%%%%%%%%%%%%%%%%%%%%%%%%%%%%%%%%%%%%%%%%%%%%%%%%
% TABLE OF CONTENTS
\usepackage{tocloft}    % customize toc
\renewcommand{\cfttoctitlefont}{\titlefont\Huge\bfseries}
\setcounter{tocdepth}{1}
\renewcommand{\cftbeforetoctitleskip}{0pt} % change appearence of heading of toc: 0 space above, bold, titlefont, huge toc-heading
\renewcommand{\cftchapnumwidth}{2em} % change indentations due to width of capital roman numbers
\renewcommand{\cftsecindent}{2em}
\renewcommand{\cftsecnumwidth}{2em}
\renewcommand{\cftsubsecindent}{4em}

%%%%%%%%%%%%%%%%%%%%%%%%%%%%%%%%%%%%%%%%%%%%%%%%%%%%%%%%%%%%%%%%%%%%%%%%%%%%%%%
% GRAPHICS
\usepackage{graphicx}
\usepackage[table,dvipsnames]{tudscrcolor} % package xcolor loaded automatically
\usepackage[font=small,labelfont=bf]{caption} % captions of non-floated figures

\usepackage{tcolorbox}
\usepackage{tikz}

\usepackage[table,dvipsnames]{tudscrcolor} % package xcolor loaded automatically

% tabulars
\usepackage{tabularx}   % tabularx-environment (explicitly set width of columns)
\usepackage{multirow}
\usepackage{booktabs}   % improved rules

%%%%%%%%%%%%%%%%%%%%%%%%%%%%%%%%%%%%%%%%%%%%%%%%%%%%%%%%%%%%%%%%%%%%%%%%%%%%%%%
% TITLE STYLES
\usepackage{titlesec}   % change title headings look
\usepackage{relsize}    % relative font size (smaller[i], larger[i], ...)

\newcommand{\titlefont}{\osfamily}
\newcommand{\chaptersize}{\huge}
\newcommand{\sectionsize}{\LARGE}

\renewcommand{\thepart}{\Alph{part}}

% \titleformat{<command>}[<shape>]{<format>}{<label>}{<sep>}{<before-code>}[<after-code>]
% \titlespacing*{<command>}{<left>}{<before-sep>}{<after-sep>}[<right-sep>]

%%%%%%%%% framed chapter
\titleformat{\chapter}[frame]{\bfseries\titlefont\color{cddarkblue}}{\enskip \chaptersize \smaller \chaptername\;\thechapter \enskip}{10pt}{\chaptersize\centering\MakeUppercase}%
\titlespacing{\chapter}{0pt}{0pt}{10pt}%

%%%%%%%%% chapter.section
\counterwithin{section}{chapter}%
\titleformat*{\section}{\bfseries\titlefont\sectionsize}%{\thesection}{8pt}{}%
\titlespacing{\section}{0pt}{10pt}{5pt}
\titleformat*{\subsection}{\bfseries\titlefont\sectionsize\smaller}

\pretocmd{\chapter}{\setcounter{section}{0}}{}{} % automatic reset of section after chapter ended
\pretocmd{\section}{\setcounter{equation}{0}}{}{} % automatic reset of equation counter in each section

%%%%%%%%%%%%%%%%%%%%%%%%%%%%%%%%%%%%%%%%%%%%%%%%%%%%%%%%%%%%%%%%%%%%%%%%%%%%%%%
% ENUMERATIONS
\usepackage{enumerate}
\usepackage[inline]{enumitem}       % customize label

\renewcommand{\labelitemi}{\raisebox{1pt}{\scalebox{.4}{$\blacksquare$}}}
\renewcommand{\labelitemii}{$\vartriangleright$}
\renewcommand{\labelitemiii}{--}
% Variantionen des Dreiecks als Aufzählungszeichen $\blacktriangleright$ / $\vartriangleright$ / $\triangleright$

\renewcommand{\labelenumi}{(\arabic{enumi})}
\renewcommand{\labelenumii}{\alph{enumii}.}
\renewcommand{\labelenumiii}{\roman{enumiii}.}

%%%%%%%%%%%%%%%%%%%%%%%%%%%%%%%%%%%%%%%%%%%%%%%%%%%%%%%%%%%%%%%%%%%%%%%%%%%%%%%
% PAGE STYLES
\newcommand*{\rightinfo}{Vorlesung \enquote{Mathematische Statistik} bei Prof. Dr. Ferger im Wintersemester 2020/21}

\usepackage{fancyhdr}   % customize header / footer

\fancypagestyle{myStyle}{%
    \fancyhf{} %
    \fancyfoot[C]{\thepage} %
    \renewcommand{\headrulewidth}{0pt}     % Line at the header invisible
    \renewcommand{\footrulewidth}{0pt}     % Line at the footer visible
    \fancyhead[C]{\textcolor{gray}\leftmark} %
    \fancyhead[R]{%
        \begin{tikzpicture}[overlay,remember picture]
            \node [
                fill=none,  % Farbe des Randstreifens
                text=gray,  % Textfarbe
                font=\osfamily\normalsize,  % Einstellungen für die Schrift
                inner xsep=\footskip,       % Abstand des Textes von unten
                % maximale Textbreite = Papierhöhe - 2*Abstand des Textes von unten:
                text width={\dimexpr\paperheight-2\footskip\relax},
                align=center,
                minimum height=7mm,% Breite des Randstreifens
                anchor=south west,
                rotate=90
            ] at ($(current page.south east)+(-10mm,0mm)$)
            {\rightinfo};
        \end{tikzpicture}%
    }
}
\fancypagestyle{rightinfo}{%
    \fancyhf{} %
    \fancyfoot[C]{\thepage} %
    \renewcommand{\headrulewidth}{0pt}     % Line at the header invisible
    \renewcommand{\footrulewidth}{0pt}     % Line at the footer visible
    \fancyhead[R]{%
        \begin{tikzpicture}[overlay,remember picture]
            \node [
                fill=none,  % Farbe des Randstreifens
                text=gray,  % Textfarbe
                font=\sffamily\normalsize,  % Einstellungen für die Schrift
                inner xsep=\footskip,       % Abstand des Textes von unten
                % maximale Textbreite = Papierhöhe - 2*Abstand des Textes von unten:
                text width={\dimexpr\paperheight-2\footskip\relax},
                align=center,
                minimum height=7mm,% Breite des Randstreifens
                anchor=south west,
                rotate=90
            ] at ($(current page.south east)+(-10mm,0mm)$)
            {\rightinfo};
        \end{tikzpicture}%
    }
}
% changes pagestyle on first page of each chapter; instead of empty page the normal footer is printed
\patchcmd{\chapter}{\thispagestyle{plain}}{\thispagestyle{rightinfo}}{}{}

\pagestyle{myStyle}
\pagenumbering{arabic}

%%%%%%%%%%%%%%%%%%%%%%%%%%%%%%%%%%%%%%%%%%%%%%%%%%%%%%%%%%%%%%%%%%%%%%%%%%%%%%%
% TITLEPAGE
\newcommand{\makeTUtitle}[1][]{%
    \begin{titlepage}
        \pagecolor{cddarkblue!90}
        \color{white}

        \raggedright
        \fosfamily
        \setlength{\parindent}{0pt}

        \hspace{-18.6mm} %
        \includegraphics[scale=0.6]{TUD-white.pdf} \\
        %       \includegraphics[scale=0.6]{TUD-blue.pdf} \\

        \vspace{3mm}
        \begin{tabular}{m{\textwidth}}
            \hline
            \hspace{-4pt}\small{\textbf{Fakultät Mathematik} Institut für Stochastik, Professur für Statistik} \\
            \hline
        \end{tabular} \\

        \vspace{5cm}
        {\Huge\bfseries \MakeUppercase Mathematische Statistik \par}
        %
        \ifthenelse{\isempty{#1}}{}{
            \vspace{0.5cm}%
            {\Large \itshape #1} \\%
        }
        %
        \vspace{1.5cm}
        \textbf{{\Large Prof. Dr. Dietmar Ferger}} \par
        \vspace{0.5cm}
        {\large Wintersemester 2020/21}

        \vfill
        \begin{tabular}{lll}
            Mitschrift  & : & Eric Kunze \\
            E-Mail      & : & \url{eric.kunze@mailbox.tu-dresden.de} \\
        \end{tabular}
    \end{titlepage}
    \nopagecolor
}

%%%%%%%%%%%%%%%%%%%%%%%%%%%%%%%%%%%%%%%%%%%%%%%%%%%%%%%%%%%%%%%%%%%%%%%%%%%%%%%
% MATH ENVIRONMENTS

\usepackage{../mathoperatorsMathTUD}

\usepackage[ntheorem,framemethod=TikZ]{mdframed}
\usepackage{amsmath,amssymb,amsfonts,mathtools}
\usepackage[amsmath,thmmarks,framed]{ntheorem}

\newcommand{\skiparound}{10pt}

\newcounter{thmcount}
\newcounter{defcount}
\counterwithin{thmcount}{chapter}
\counterwithin{defcount}{chapter}

% possible settings:
% style=boxedtheorem,%
% % backgroundcolor=cdpurple!5,%
% % linecolor=cdpurple,%
% innertopmargin=\boxskip,%
% innerbottommargin=\boxskip,%
% % innerleftmargin=4pt,%
% % outerlinewidth=5pt,%
% % topline=false,%
% % rightline=true,%
% % leftline=false,%
% % bottomline=false,%
% innertopmargin=3pt,%
% innerbottommargin=3pt,%
% % leftmargin=-4pt,%
% rightmargin=-10pt,%
% frametitlefont=\osfamily\bfseries,%
% skipabove=5pt,%
% skipbelow=5pt,%
% % skipabove=6pt,%
% % skipbelow=6pt,%
% % nobreak,%

\theorempreskip{\skiparound}
\theorempostskip{\skiparound}

\theoremstyle{nonumberplain}
\theoremseparator{}
\theoremheaderfont{\osfamily \bfseries \upshape}
\theorembodyfont{}

% DEFINTION
\newmdtheoremenv[%
	backgroundcolor=cdorange!10,%
	linecolor=cdorange,%
	innerleftmargin=4pt,%
	outerlinewidth=5pt,%
	topline=false,%
	rightline=true,%
	leftline=false,%
	bottomline=false,%
	leftmargin=-4pt,%
	skipabove=6pt,%
	skipbelow=6pt,%
	nobreak,%
]{definition}[thmcount]{Definition}

\theoremstyle{plain}
\theoremseparator{}

% THEOREM
\newmdtheoremenv[%
	backgroundcolor=cdblue!10,%
	linecolor=cddarkblue!90,%
	innerleftmargin=4pt,%
	outerlinewidth=5pt,%
	topline=false,%
	rightline=true,%
	leftline=false,%
	bottomline=false,%
	leftmargin=-4pt,%
	skipabove=6pt,%
	skipbelow=6pt,%
	nobreak,%
]{theorem}[thmcount]{Theorem}

% SATZ
\newmdtheoremenv[%
	backgroundcolor=cdblue!10,%
	linecolor=cddarkblue!90,%
	innerleftmargin=4pt,%
	outerlinewidth=5pt,%
	topline=false,%
	rightline=true,%
	leftline=false,%
	bottomline=false,%
	leftmargin=-4pt,%
	skipabove=6pt,%
	skipbelow=6pt,%
]{satz}[thmcount]{Satz}

% LEMMA
\newmdtheoremenv[%
	backgroundcolor=cdindigo!4,%
	linecolor=cdindigo!50,%
	innerleftmargin=4pt,%
	outerlinewidth=5pt,%
	topline=false,%
	rightline=true,%
	leftline=false,%
	bottomline=false,%
	leftmargin=-4pt,%
	skipabove=6pt,%
	skipbelow=6pt,%
	nobreak,%
]{lemma}[thmcount]{Lemma}

% PROPOSITION
\newmdtheoremenv[%
	backgroundcolor=cdindigo!4,%
	linecolor=cdindigo!50,%
	innerleftmargin=4pt,%
	outerlinewidth=5pt,%
	topline=false,%
	rightline=true,%
	leftline=false,%
	bottomline=false,%
	leftmargin=-4pt,%
	skipabove=6pt,%
	skipbelow=6pt,%
	nobreak,%
]{proposition}[thmcount]{Proposition}

% KOROLLAR
\newmdtheoremenv[%
	backgroundcolor=cdpurple!5,%
	linecolor=cdpurple,%
	innerleftmargin=4pt,%
	outerlinewidth=5pt,%
	topline=false,%
	rightline=true,%
	leftline=false,%
	bottomline=false,%
	leftmargin=-4pt,%
	skipabove=6pt,%
	skipbelow=6pt,%
	nobreak,%
]{korollar}[thmcount]{Korollar}

\theoremstyle{plain}
\theoremheaderfont{\osfamily \bfseries}
\theorembodyfont{}

\newtheorem{beispiel}[thmcount]{Beispiel}
\newtheorem{erinnerung}[thmcount]{Erinnerung}
\newtheorem{wiederholung}[thmcount]{Wiederholung}
\newtheorem{bemerkung}[thmcount]{Bemerkung}

% theorems without numbers
\theoremstyle{nonumberplain}
\theoremheaderfont{\osfamily \bfseries}
\theorembodyfont{}
\theoremseparator{.}

\newmdtheoremenv[%
	backgroundcolor=cdorange!10,%
	linecolor=cdorange,%
	innerleftmargin=4pt,%
	outerlinewidth=5pt,%
	topline=false,%
	rightline=true,%
	leftline=false,%
	bottomline=false,%
	leftmargin=-4pt,%
	skipabove=6pt,%
	skipbelow=6pt,%
	nobreak,%
]{*definition}{Definition}

\newtheorem{*interpretation}{Interpretation}
\newtheorem{*bemerkung}{Bemerkung}
\newtheorem{*beispiel}{Beispiel}
\newtheorem{*erinnerung}{Erinnerung}
\newtheorem{*notation}{Notation}

% >> proof
\makeatletter
\newtheoremstyle{proofstyle}%
{\item[\hskip\labelsep {\theorem@headerfont ##1}\theorem@separator]}%
{\item[\hskip\labelsep {\theorem@headerfont ##1}\ (##3)\theorem@separator]}
\makeatother

\theoremstyle{proofstyle}
\theoremheaderfont{\normalsize\slshape}
\theorembodyfont{}
\theoremseparator{.}
\theorempreskip{5pt}
\theorempostskip{5pt}
\theoremsymbol{$\square$}
\newtheorem{proof}{Beweis}

% >> Equivalences
\newcommand{\hinrichtung}{\item[\bfseries ($\boldsymbol{\Rightarrow}$)]}
\newcommand{\rueckrichtung}{\item[\bfseries ($\boldsymbol{\Leftarrow}$)]}
\newenvironment{equivalence}[1][]{%
        \ifthenelse{\isempty{#1}}%
        {\begin{description}} % no optional argument
            {\begin{description}[topsep=-\parskip]} % any optional argument
    }{ \end{description} }%

    % >> inductions (needs description environment)
    \newcommand{\ianfang}[1][]{%
        \ifthenelse{\isempty{#1}}{%
            \item[\textbf{(IA)}] % no parameter
        }{  \item[\textbf{(IA)}] {#1 :}  } % parameter exists
    }

    \newcommand{\ivorraussetzung}{\item[\bfseries (IV)]}

    \newcommand{\ischritt}[1][]{%
        \ifthenelse{\isempty{#1}}{%

            \item[\textbf{(IS)}] % no parameter
        }{  \item[\textbf{(IS)}] {#1 :} } % parameter exists
    }

%%%%%%%%%%%%%%%%%%%%%%%%%%%%%%%%%%%%%%%%%%%%%%%%%%%%%%%%%%%%%%%%%%%%%%%%%%%%%%%
% HYPERLINKS
\usepackage[unicode,bookmarks=true]{hyperref}
\hypersetup{
    % pdfborder={0 0 0}         % no boxed around links
    pdfborderstyle={/S/U/W 1},  % underlining insteas of boxes
    linkbordercolor=cdblue,
    urlbordercolor=cdblue
    %   colorlinks,
    %   citecolor=black,
    %   filecolor=cddarkblue!80,
    %   linkcolor=black,
    %   urlcolor=cddarkblue!80
}
\usepackage{bookmark}       % pdf-bookmarks

\usepackage{cleveref}
\crefname{theorem}{Theorem}{Theoreme}
\crefname{satz}{Satz}{Sätze}
\crefname{lemma}{Lemma}{Lemmata}
\crefname{proposition}{Proposition}{Propositionen}
\crefname{folgerung}{Folgerung}{Folgerungen}
\crefname{korollar}{Korollar}{Korollare}
\crefname{definition}{Definition}{Definitionen}
\crefname{bemerkung}{Bemerkung}{Bemerkungen}
\crefname{beispiel}{Beispiel}{Beispiele}
\crefname{erinnerung}{Erinnerung}{Erinnerungen}

%%%%%%%%%%%%%%%%%%%%%%%%%%%%%%%%%%%%%%%%%%%%%%%%%%%%%%%%%%%%%%%%%%%%%%%%%%%%%%%
% HIGHLIGHTING
\newcommand{\begriff}[1]{\textbf{#1}}
\newcommand{\person}[1]{\textsc{#1}}

%%%%%%%%%%%%%%%%%%%%%%%%%%%%%%%%%%%%%%%%%%%%%%%%%%%%%%%%%%%%%%%%%%%%%%%%%%%%%%%
% ADDITIONAL COMMANDS

\usepackage[numbers, square]{natbib}

\renewcommand{\dx}{\mathrm{d}x}
\newcommand{\A}{\mathcal{A}}
\newcommand{\B}{\mathcal{B}}
\renewcommand{\S}{\mathcal{S}}
\renewcommand{\G}{\mathcal{G}}
\renewcommand{\L}{\mathcal{L}}
\newcommand{\ntoinf}{\overset{n \to \infty}{\longrightarrow}}

%%%%%%%%%%%%%%%%%%%%%%%%%%%%%%%%%%%%%%%%%%%%%%%%%%%%%%%%%%%%%%%%%%%%%%%%%%%%%%%

\begin{document}
    \makeTUtitle

    \tableofcontents

    \chapter{Median}
    \label{chapter_1_median}
    Viele Schätzer in der Statistik sind definiert als Minimal- oder Maximalstelle von bestimmten \begriff{Kriteriumsfunktionen}, z.\,B. der \begriff{Maximum-Likelihood-Schätzer (MLS)} oder \begriff{Mi\-ni\-mum-Quadrat-Schätzer (MQS, KQS)} oder \begriff{Bayes-Schätzer}. Allgemein nennt man solche Schätzer \begriff{M-Schätzer}.

\begin{beispiel}[Maximum-Likelyhood-Schätzer]
	Gegeben seien $X_1, ..., X_n$ iid. $\sim f_\theta$. Dann ist $\hat{\theta}_n$ die Maximumstelle von der Funktion $l \colon \theta \mapsto \sum_{i = 0}^n \log f_{\theta}(X_i)$ (sogenannte Log-Likelyhood-Funktion). Dabei kommen alle $\theta$ aus einer möglichen Menge $\Theta$ in Frage.
\end{beispiel}

Ziel: Untersuchung des asymptotischen Verhaltens ($n \to \infty$) von M-Schätzern über einen funktionalen Ansatz. Als Beispiel betrachten wir nun den Median

    % This work is licensed under the Creative Commons
% Attribution-NonCommercial-ShareAlike 4.0 International License. To view a copy
% of this license, visit http://creativecommons.org/licenses/by-nc-sa/4.0/ or
% send a letter to Creative Commons, PO Box 1866, Mountain View, CA 94042, USA.

Sei $X \colon (\Omega,\A,\P) \to \R$ eine reelle Zufallsvariable mit Verteilungsfunktion $F_X\colon \R \to [0,1]$, d.\,h.
\begin{align*}
	F_X(x) \defeq \P(X \le x) 
	\defeq \P(\menge{\omega \in \Omega : X(\omega) \le x})
	= \P\brackets{X^{-1}\brackets{(-\infty, x]}},
\end{align*}
also $X \sim F_X$. Definiere
\begin{align}
	Y(t) &\defeq \E(\abs{X-t}) \label{eq: DefY} \tag{1.0} \\ \nonumber
	&= \int_\Omega \abs{X(\omega)-t} \, \P(\diff\omega) \\ \nonumber
	\overset{\cref{eqTrafoR}}&{=} \int_\R \abs{x-t} \, (\P \circ X^{-1})(\diff x) \\ \nonumber
	&= \int_\R \abs{x-t} \, F_X(\diff x) \qquad \forall t \in \R \\
	\intertext{und}
	m &\in \argmin_{t \in \R} Y(t) \label{eq: MedianDef} \tag{M}
\end{align}
als (irgendeine) Minimalstelle der Funktion $Y$.

\begin{definition}
	Hierbei heißt $m \in \R$ \begriff{Median} von der Zufallsgröße $X$.
\end{definition}

Wir haben den \begriff{Korrespondenzsatz} genutzt, der besagt, dass das Bildmaß $\P \circ  X^{-1}$ eindeutig von der Verteilungsfunktion $F$ bestimmt ist. Das heißt also wir schreiben
$F(\dx)$ und meinen $\P \circ X^{-1} (\dx)$

\textbf{Notation:} $F(m-) \defeq F(m-0) \defeq \lim_{t \uparrow m} F(t)$. Der Ausdruck ist wohldefiniert für Verteilungsfunktionen, denn diese sind rechtsseitig stetig und haben linksseitige Limiten.


In folgendem kleinen Lemma wollen wir die Menge aller Mediane charakterisieren.
\setcounter{thmcount}{-1}
\begin{lemma}\label{lemma: median}
	Sei $X \sim F_X \defqe F$ integrierbar und $m \in \R$. Dann sind äquivalent:
	\begin{enumerate}[label=(\arabic*)]
		\item \label{it:medianHalf} $F(m-)\le \frac{1}{2} \le F(m)$
		\item \label{it:medianMin} $\E\sqbrackets{\abs{X-t}}\ge\E{\abs{X-m}} \qquad\forall t \in \R$
		\item \label{it:medianMedian} $m$ ist Median
	\end{enumerate}
\end{lemma}
\begin{beispiel}
	Für $F$, $a < b \in \R$ mit
	\begin{align*}
		F(x) \begin{cases}
			< \frac12, &\text{ für } x < a \\
			= \frac12, &\text{ für } a \le x < b \\
			> \frac12, &\text{ für } x \ge b
		\end{cases}
	\end{align*}
	ist die Menge der Mediane genau das Intervall $[a, b]$.
	% Bild: Diagramm mit Funktion F mit Plateau bei 1/2 (1/2 eingetragen auf y-Achse
	% Punkt 1 exkl. unter 1/2, von Punkt 1 inkl. bis Punkt 2 exkl. Plateau,
	% an Punkt 2 Sprung, dann ansteigend
	% auf x-Achse Intervall [Punkt 1, Punkt 2] farbig = alle Mediane
\end{beispiel}

\begin{proof}
	\begin{description}
		\item[\ref{it:medianHalf} $\Rightarrow$ \ref{it:medianMin}:] Sei $\Q \defeq \P \circ X^{-1}$ das zu $F$ gehörige Bildmaß. Setze
		\begin{equation*}
			h(t) \defeq \E\sqbrackets{\abs{X - t} - \abs{X - m}} \overset{lin.}{=} Y(t) - Y(m)
		\end{equation*}
		Dann ist \ref{it:medianMin} äquivalent zu $h(t) \ge 0$ für alle $t\in\R$. Dies wollen wir nun zeigen.
		\begin{enumerate}[label=Fall \Alph*:] 
			\item Sei $t < m$.
			\begin{align*}
					h(t) 
					\overset{\text{Trafo}}&{=}
					\int_{\R} \abs{x-t}-\abs{x-m} \, F(\dx)\\
					&=\int_{(-\infty,t]} \underbrace{\abs{x-t}-\abs{x-m}}_{= t-x-(m-x)=-(m-t)} \, F(\dx)
					+ \int_{(t,m)} \underbrace{\underbrace{\abs{x-t}-\abs{x-m}}_{\underbrace{x-t}_{\ge 0}-\underbrace{(m-x)}_{\le m-t}}}_{\ge-(m-t)} \, F(\dx)\\
					&\phantom{=}
					+\int_{[m,\infty)}\underbrace{\abs{x-t}-\abs{x-m}}_{x-t-(x-m)=m-t} \, F(\dx) \\
					&\ge-(m-t) * \underbrace{Q((-\infty,t])}_{F(t)}+
					\brackets{-(m-t) * F(m-)-F(t)}
					+(m-t) * \underbrace{\Q([m,\infty))}_{1-\underbrace{\Q((-\infty,m))}_{F(m-)}}\\
					&=-\underbrace{(m-t)}_{\ge 0} * (\underbrace{1-2 * F(m-)}_{\text{wegen 1.\ Ungl. in }\ref{it:medianHalf}: \ge 0})\\
					&\ge0
				\end{align*}
			\item Sei $t > m$.  Wir verfahren ganz ähnlich:
				\begin{align*}
					h(t)&=	\int_{(-\infty,m]} \abs{x-t} - \abs{x-m} F(\dx)
						+ \int_{(m,t]} \abs{x-t} - \abs{x-m} F(\dx) \\
					&\phantom{=}\ + \int_{(t,\infty)} \abs{x-t} - \abs{x-m} F(\dx) \\
					&= \int_{(-\infty,m]} t - x - (m-x) F(\dx)
						+ \int_{(m,t]} t - x - (x - m) F(\dx) \\
					&\phantom{=}\ + \int_{(t,\infty)} x - t - (x - m) F(\dx) \\
					&\ge (t-m)F(m) - (t-m)(F(t) - F(m)) + (m - t)(1 - F(t)) \\
					&= (t-m)(F(m) - F(t) + F(m) - 1 + F(t)) \\
					&=\underbrace{(t-m)}_{> 0}*(\underbrace{2 * F(m)-1}_{\text{wegen 2.\ Ungl.\ in }\ref{it:medianHalf}\ge0})\\
					&\ge 0
				\end{align*}
			\item Für $t = m$ ist die Aussage trivial.
		\end{enumerate}
		\item[\ref{it:medianMin} $\Rightarrow$ \ref{it:medianHalf}:]
			Nach Annahme ist $h(t) \ge 0$ für alle $t \in \R$.
			\begin{enumerate}[label=Fall \Alph*:] 
				\item Sei $t<m$. Die obige Rechnung im Fall 1 bei $\Rightarrow$ zeigt
				\begin{align*}
					0 \le h(t)
					&= -(m-t) * F(t)+ \int_t^m \underbrace{\underbrace{x}_{<m}-t-(m-x) }_{=2x-t-m \le m-t} F(\dx)+(m-t) * (1-F(m-)) \\
					& \le -(m-t)* \Big( F(t)-1 \underbrace{+F(m-)-F(m-)}_{=0}+F(t) \Big) \\
					&=\underbrace{(m-t)}_{>0}*(1-2* F(t)) \\
					&\Rightarrow\forall t<m: \enskip 0 \le(m-t)*(1-2* F(t))\\
					&\Rightarrow\forall t<m: \enskip 0 \le 1-2* F(t)\\
					&\Rightarrow\forall t<m: \enskip F(t) \le \frac{1}{2}\\
					&\overset{t \uparrow m}{\Rightarrow} F(m-)\le\frac{1}{2} \tag{Def. linksseitiger Limes}
				\end{align*}
			\item Sei $t > m$. Siehe 2. Fall, analog:
			\begin{align*}
				0 \le h(t)
				&= (t-m)* F(m) + \int_m^t \underbrace{t - x - (x - m) }_{=t + m - 2x \le t-m} F(\dx) + (m-t)*(1-F(t)) \\
				&\le (t-m)*\brackets{F(m)+ F(t) - F(m)-1 + F(t)} \\
				&= \underbrace{(t-m)}_{>0}(2F(t) - 1) \\
				&\Rightarrow \forall t<m: \enskip 0\le 2F(t) - 1\\
				&\Rightarrow \forall t<m: \enskip F(t) \ge \frac{1}{2}\\
				&\overset{t\downarrow m}{\Rightarrow} F(m) \ge \frac{1}{2} \tag{Rechtsstetigkeit von $F$}
			\end{align*}
			\end{enumerate}
		\item[\ref{it:medianMin} $\Rightarrow$ \ref{it:medianMedian}:]
			Die Aussage \ref{it:medianMin} ist offensichtlich äquivalent zur Definition des Medians.
	\end{description}
\end{proof}

\begin{bemerkung}~
	\begin{enumerate}[label=(\arabic*), leftmargin=*]
		\item Aussage \cref{it:medianHalf} in \cref{lemma: median} besagt, dass $\menge{ m \in \R : m \text{ erfüllt } \eqref{eqMedianDef}}$ die Menge aller Mediane von $F$ ist. Der Median ist im Allgemeinen nicht eindeutig bestimmt.
		\item Im Allgemeinen gibt es mehrere Mediane. Üblicherweise \textit{wählt} man $m \defeq F^{-1} \brackets{\frac{1}{2}}$, wobei
			\begin{align*}
				F^{-1}(u) \defeq \inf\menge{ x \in \R: F(x) \ge u} \quad \forall u \in (0,1)
			\end{align*}
			die \begriff{Quantilfuntion} oder auch \begriff{verallgemeinerte Inverse} ist. Für weiterführende Literatur siehe \cite[Seite 20]{witting1985mathStat}. Da
			\begin{align*}
				F\brackets{F^{-1}(u)-} \le u \le F\brackets{F^{-1}(u)} \qquad \forall u \in (0,1),
			\end{align*}
			erfüllt $m = F^{-1}\brackets{\frac{1}{2}}$ die Bedingung \ref{it:medianHalf} in \cref{lemma: median} und ist somit ein Median, nämlich der kleinste.
		\item Die obige Funktion \eqref{eq: DefY}
			\begin{align*}
				Y \colon \R \to \R \text{ mit } Y(t) = \int_\R \abs{x-t}~F(\dx) \qquad \forall t\in\R
			\end{align*}
			ist stetig\footnote{zur Stetigkeit von $Y$: nutze Folgenkriterium + dominierte Konvergenz mit Majorante $\abs{X} + \abs{t}$}, aber im Allgemeinen nicht differenzierbar, z.\,B.\ falls $F \sim X$ eine diskrete Zufallsvariable ist. In diesem Fall ist somit die Minimierung über Differentiation nicht möglich.
		\item Sei $\mu = \E(X)$ der Erwartungswert von $X$. Dann gilt (Übung):
			\begin{equation*}
				\mu = \argmin_{t \in \R} \E\sqbrackets{(X-t)^2} = \argmin_{t \in \R} \E\sqbrackets{(X-t)^2 - X^2}.
			\end{equation*}
			(Das zweite $X^2$ wird abgezogen, da das $\argmin$ nicht davon betroffen ist und so die Bedingung $\E\sqbrackets{X^2}<\infty$ entfällt.)
			Begründung:
			\begin{align*}
				\E\sqbrackets{(X-t)^2) - X^2}
				&= \E\sqbrackets{X^2 - 2tX + t^2 - X^2} \\
				&= \E\sqbrackets{X^2} - 2t\mu + t^2 - \E\sqbrackets{X^2}
				= t^2 - 2t\mu
			\end{align*}
			Minimiere nun diese quadratische Funktion und erhalte die gewünschte Resultat.
	\end{enumerate}
\end{bemerkung}

%\begin{notation}
	In der Statistik identifizieren wir oft stillschweigend Zufallsgrößen mit ihren Realisationen, also $X \leftrightsquigarrow x$ was sich formal $X(\omega_0) = x$ schreiben lässt.
%\end{notation}

Zur Schätzung von $m$ seien $X_1,\dots, X_n \sim F$ iid Zufallsvariablen mit zugehöriger \begriff{empirischer Verteilungsfunktion}
\begin{equation*}
	F_n \colon \R \to [0,1] 
	\quad \text{ mit } \quad
	F_n(x) \defeq \frac{1}{n} * \sum_{i=1}^n \one_{\menge{ x_i \le x}}(x) \qquad \forall x \in \R.
\end{equation*}
Tatsächlich ist $F_n$ die Verteilungsfunktion zum \begriff{empirischen Maß}
\begin{equation*}
	Q_n \colon \borel\R \to [0,1]
	\quad \text{ mit } \quad
	Q_n(B)\defeq\frac{1}{n}*\sum_{i=1}^n\delta_{x_i}(B)
\end{equation*}
wobei das \begriff{Dirac-Maß} in $t$ definiert ist als
\begin{equation*}
	\delta_t \colon \borel\R \to [0,1]
	\quad \text{ mit } \quad
	\delta_t(B) \defeq
	\begin{cases}
		0, &\falls t \notin B \\
		1, &\falls t \in B
	\end{cases}
	\qquad\forall B \in \borel\R \enskip \forall t \in \R
\end{equation*}

Gemäß dem Satz von Gliwenko-Cantelli gilt:
\begin{equation*}
	\sup_{x \in \R} \abs{F_n(x)-F(x)} \overset{n \to \infty}{\longrightarrow} 0 \qquad \P\text{-fast sicher für alle Verteilungsfunktionen } F
\end{equation*}
Die empirische Verteilungsfunktion konvergiert also gegen die wahre Verteilungsfunktion.

\begin{*erinnerung}
	Für das Dirac-Maß $\delta_x \colon \A \to \R_+$ mit $\delta_x(A) \defeq \one_A(x)$ gilt:
	\begin{equation}\label{eq: dirac} \tag{Dir}
		\int_\R f(t) \, \delta_x(\diff{t}) = f(x) \qquad \forall x \in \R
	\end{equation}
\end{*erinnerung}

\begin{*notation}
	Wir identifizieren eine Verteilung $\P\circ X^{-1}$ mit der zugehörigen Verteilungsfunktion $F_X$, also $\P \circ X^{-1} \leftrightarrow F_X$ und $F_X(\dx) \defeq (\P\circ X^{-1} \, \dx)$.
\end{*notation}

\begin{*erinnerung}
	Das Lebesgue-Maß ist linear im Maß, d.\,h.:
	\begin{align}\label{eqLinMasse}\tag{Lin}
		\int_\omega f \diff{(a*\mu + b*\nu)} = a*\int_\omega f \diff{\mu} + b*\int_\omega f \, \diff{\nu}
	\end{align}
\end{*erinnerung}
%
%Ein vages Stetigkeitsargument motiviert folgenden Schätzer für $m$,
%den wir durch Ersetzen von $F$ in $Y$ durch $F_n$ erhalte (\define{plug-in}-Methode):
%\begin{align*}
%	\hat{m}_n&\defeq\argmin_{t\in\R}Y_n(t)\defeq\text{ (irgendeine) Minimalstelle der Funktion}\\
%	Y_n(t)&\defeq\int_\R \abs{x-t}F_n(\d x)\\
%	&=\int_\R \abs{x-t}Q_n(\d x)\\
%	\overset{\Def~Q+\eqref{eqLinMasse}}&=
%	\frac{1}{n}*\sum_{i=1}^n\int_\R \abs{x-t}~\delta_{X_i}(\d x)\\
%	\overset{\eqref{eq: dirac}}&=
%	\frac{1}{n}*\sum_{i=1}^n\abs{X_i-t}
%\end{align*}
%
%$\hat{m}_n$ heißt \define{empirischer Median} von $X_1,\dots,X_n$ mit üblicher Auswahl $\hat{m}_n=F_n^{-1}\klammern{\frac{1}{2}}$ gemäß Lemma \ref{lemma: median} (da empirische Verteilungsfunktion eine Verteilungsfunktion ist).
%
%\begin{bemerkung}\
%	\begin{itemize}
%		\item Wenn man eine ungerade Anzahl von Daten hat, ist der Median der mittlere Wert, nachdem man die Daten der Größe nach geordnet hat.
%		\item Hat man hingegen eine gerade Anzahl an Daten, dann ist der Median der kleinere der beiden mittleren Werte.
%	\end{itemize}
%\end{bemerkung}
%
%Mit dem starken \undefine{Gesetz der großen Zahlen (SGGZ)} gilt
%\begin{align}\label{eq1.1}\tag{1.1}
%	\forall t \in \R: \klammern{Y_n(t)\ntoinf \Earg{\abs{X_1 - t}} = \int_{\R} \abs{x-t} F(\d x) = Y(t) \text{ $\P$-fast sicher}}
%\end{align}
%Problem: Folgt aus \eqref{eq1.1} bereits, dass
%\begin{align*}
%	\argmin_{t\in\R} Y_n(t)
%	\ntoinf
%	% CHECKED: '\longrightarrow' used.
%	\argmin_{t\in\R}
%	Y(t)\text{ fast sicher?}
%\end{align*}
%Wenn ja, dann hätten wir:
%\begin{align*}
%	\hat{m}_n
%	\ntoinf
%	% CHECKED: '\longrightarrow' used.
%	m\text{ fast sicher (\define{starke Konsistenz})}
%\end{align*}
%(Hierbei stellt sich auch wieder die Frage, was wir genau meinen, wenn $m$ nicht eindeutig ist.)
%
%Wir formalisieren und verallgemeinern:
%\begin{align*}
%	&X_i:(\omega, \A,P)\to(\R,\B(\R))\text{ messbar},\qquad\omega↦ X_i(\omega)\\
%	&\Rightarrow
%	Y_n(t)\defeqY_n(t,\omega)=
%	\frac{1}{n}*\sum_{i=1}^n\abs{X_i(\omega)-t}\\
%	&\Rightarrow
%	Y_n(t,*):(\omega,\A)\to(\R,\B(\R))\text{ messbar }\forall t\in\R
%\end{align*}
%
%\begin{defi}
%	Die \define{Kollektion}
%	\begin{align*}
%		Y_n\defeq\set{ Y_n(t,*):t\in\R}
%		=\set{ Y_n(t):t\in\R}
%	\end{align*}
%	heißt \define{stochastischer Prozess (SP)}. (Ein stochastischer Prozess
%	ist etwas allgemeiner eine Kollektion von Zufallsvariablen, die von
%	einem Parameter abhängen, hier $t\in\R$.) Die Abbildung
%	\begin{align*}
%		Y_n(*,\omega):\R\to\R,\qquad t↦ Y_n(t,\omega)
%	\end{align*}
%	heißt \define{Trajektorie/ Pfad} des SP $Y_n$ zu festem $\omega\in\omega$.
%\end{defi}
%
%In unserem Beispiel sind für \emph{alle} $\omega\in\omega$ die Pfade stetig auf $\R$. Die Abbildung
%\begin{align*}
%	Y_n:\omega\to C(\R;\R),\qquad\omega↦ Y_n(*,\omega)
%\end{align*}
%heißt \define{Pfadabbildung} des SP $Y_n$. Wir identifizieren also den SP $Y_n$ mit seiner Pfadabbildung. Damit ist $Y_n$ eine Abbildung von $\omega$ in den Funktionenraum
%\begin{align*}
%	C(\R)\defeqC(\R;\R)\defeq\set{ f:\R\to\R: f\text{ ist stetig}}.
%\end{align*}
%
%Sei $d:C(\R)× C(\R) → \R$ die Metrik der gleichmäßigen Konvergenz auf Kompakta auf $C(\R)$ (formale Definition kommt später) und sei
%\begin{align*}
%	\B(C(\R))
%	\defeq\B_d\klammern[\big]{C(\R)}
%	% CHECKED: '\big' used.
%	\defeq σ \klammern[\big]{\set{ G⊆ C(\R):G\text{ ist offen bzgl. }d}}
%	% CHECKED: '\big' used.
%\end{align*}
%die von $d$ induzierte \define{Borel-$σ$-Algebra}.
%
%Wir werden sehen, dass die Abbildung
%\begin{align*}
%	Y_n:(\omega,\A,\P)\to\klammern[\Big]{C(\R),\B\klammern[\big]{C(\R)}}
%	% CHECKED: '\big' '\Big' used.
%\end{align*}
%messbar ist. $Y_n$ ist also eine Zufallsvariable mit Werten im metrischen Raum $(C(\R),d)$.
%
%\paragraph{Formulierung des Problems im allgemeinen Rahmen}
%Seien $Y_n,~n\inℕ$ mit $Y$ SP mit stetigen Pfaden (\define{stetige SP}).
%Was lässt sich sagen über die Gültigkeit der folgenden Implikation?
%\begin{align}\label{eq1.2}\tag{1.2}
%	Y_n
%	\ntoinf
%	% CHECKED: '\longrightarrow' used.
%	Y\text{ fast sicher }
%	\Rightarrow
%	\argmin_{t\in\R} Y_n(t)
%	\ntoinf
%	% CHECKED: '\longrightarrow' used.
%	\argmin_{t\in\R} Y(t)\text{ fast sicher}
%\end{align}
%\subparagraph{Ziel} Welche Art der Konvergenz $Y_n\ntoinf  Y$ reicht für obige Implikation aus? Gleichmäßige Konvergenz, gleichmäßige Konvergenz auf Kompakta, punktweise Konvergenz oder sogar nur \eqref{eq1.1}?
%% CHECKED: '\longrightarrow' used.
%
%$Y$ besitzt womöglich (mit positiver Wahrscheinlichkeit) keine eindeutige Minimalstelle. Und dann?
%
%Betrachten wir nun wieder eine andere Art der Konvergenz:
%Für die Konstruktion von (asymptotischen) Konfidenzintervallen für $m$ benötigt man (in der Statistik) \define{Verteilungskonvergenz} (Wiederholung: $\distrto $ ist die Notation für \undefine{konvergiert in Verteilung}):
%\begin{align}\label{eq1.3}\tag{1.3}
%	a_n(\hat{m}_n-m)
%	\distrto ζ\text{ in }\R
%	% CHECKED: '\longrightarrow' used.
%\end{align}
%wobei $a_n\to\infty$ eine normalisierende Folge ist und $ζ$ eine Grenzvariable ist, die es zu identifizieren gilt.
%
%\begin{beispiel}[Zentraler Grenzwertsatz für den Durchschnitt]
%	Für den Mittelwert von $\overline{X_n}$ von integrierbaren ZV $(X_n)_n$ mit Mittelwert $\mu$ ist diese Folge $a_n = √{n}$. Dafür besagt das Gesetz der großen Zahlen, dass
%	\begin{equation*}
%		\overline{X_n} - \mu → 0 \text{ fast sicher}.
%	\end{equation*}
%	Der \undefine{zentrale Grenzwertsatz} liefert uns
%	\begin{equation*}
%		√{n}\klammern{\overline{X_n} - \mu} \overset{\L}{→} N(0, σ^2)
%	\end{equation*}
%	also die Konvergenz in Verteilung gegen die Normalverteilung.
%\end{beispiel}
%Für die Herleitung von \eqref{eq1.3} favorisiere wieder einen \undefine{funktionalen Ansatz}. Sei
%\begin{align*}
%	Z_n(t)\defeqβ_n*\klammern{ Y_n\klammern{m+\frac{t}{a_n}}-Y_n(m)}\qquad t\in\R
%\end{align*}
%der sogenannte \define{reskalierte Prozess zu $Y_n$}, wobei $β_n$ (zunächst) irgendeine positive Folge ist. Damit folgt
%\begin{align}\label{eq1.4}\tag{1.4}
%	σ_n \defeq a_n(\hat{m}_n-m) \in \argmin_{t\in\R} Z_n(t),
%\end{align}
%denn:
%\begin{proof}
%  Zu zeigen ist:
%	\begin{align*}
%		&& Z_n(σ_n) &\le Z_n(t) & \forall t \in \R \\
%		&⇔& β_n \klammern{Y_n\klammern{m + \frac{σ_n}{a_n}} - Y_n(m)} &\le Z_n(t) & \forall t \in \R\\
%		&⇔& β_n \klammern{Y_n\klammern{m + \hat{m}_n - m} - Y_n(m)}
%		&\le β_n \klammern{Y_n\klammern{ m + \frac {t}{a_n}}- Y_n(m)} & \forall t \in \R \\
%		&⇔& β_n\klammern{Y_n(\hat m_n) - Y_n(m)}
%		&\le β_n \klammern{Y_n\klammern{ m + \frac {t}{a_n}} - Y_n(m)} & \forall t \in \R \\
%		&⇔& Y_n(\hat m_n) &\le Y_n\klammern{m + \frac t {a_n}} & \forall t \in \R \\
%		&⇔& Y_n(\hat m_n) &\le Y_n(u) & \forall u \in \R.
%	\end{align*}
%	Das gilt aber, weil $\hat m_n \in \argmin_{t \in \R} Y_n(t)$.
%\end{proof}
%Klar: $Z_n$ ist wieder ein stetiger stochastischer Prozess und damit $(Z_n)_{n\inℕ}$ eine Folge von Zufallsvariablen in $(C(\R),d)$.
%% CHECKED: '\big' used.
%Wünschenswert auch hier wäre die Gültigkeit folgender Implikation:
%\begin{align}\label{eq1.5}\tag{1.5}
%	% CHECKED: ist Tatsächlich das L hier bei den argmin gemeint? Macht das Sinn, sind die argmin auch eine Folge von Zufallsvariablen? -> ja, sind sie
%	Z_n
%	\distrto
%	% CHECKED: '\longrightarrow' used.
%	Z\text{ in } (C(\R),d)
%	\Rightarrow
%	\argmin_{t\in\R} Z_n(t)
%	\distrto
%	% CHECKED: '\longrightarrow' used.
%	\argmin_{t\in\R} Z(t)
%\end{align}
%Da wir $σ_n = a_n(\hat m_n - m) → ζ$ zeigen wollen und $\argmin_{t\in\R}Z_n(t) = σ_n$, haben wir jetzt die erste Charakterisierung von $ζ$ gefunden.
%
%Für die Aussage \ref{eq1.5} ist erforderlich, das Konzept der \define{Verteilungskonvergenz von Zufallsvariablen in metrischen Räumen}, damit \eqref{eq1.5} eine wohldefinierte Bedeutung erhält.
%Dies folgt später.
%
%Natürlich auch hier wieder das Problem: $Z$ besitzt mit positiver Wahrscheinlichkeit mindestens 2 Minimalstellen.
%Und dann?
%
%Im Falle einer fast sicher eindeutigen Minimalstelle von $Z$ würde aber aus \eqref{eq1.4} und \eqref{eq1.5} folgen:
%\begin{align*}
%	a_n(\hat{m}_n-m)
%	\distrto
%	% CHECKED: '\longrightarrow' used.
%	\argmin_{t\in\R} Z(t)
%\end{align*}

    
    \chapter{Metrische Räume}
    \label{chapter_2_metrischeRaeume}
    
Sei $(\S,d)$ ein metrischer Raum.
%
\begin{beispiel}[Supremums-Metrik] %2.1
	\begin{align*}
		\S = C([0,1]) & \defeq \menge{f \colon [0,1] \to \R : f\text{ stetig}} \\
		d(f,g) &\defeq \sup_{t \in [0,1]} \abs{f(t)-g(t)} \qquad \forall f,g \in C([0,1])
	\end{align*}
\end{beispiel}

\begin{definition}
	\begin{enumerate}[label={(\arabic*)}]
		\item Für $x \in\S$ und $r>0$ ist
		\begin{align*}
			B(x,r) \defeq B_d(x,r) \defeq \menge{y \in \S : d(x,y) < r}
		\end{align*}
		die \begriff{offene Kugel um Mittelpunkt $x$ und Radius $r$}.
		\item Sei $A \subseteq \S$. Dann ist:
		\begin{itemize}[nolistsep]
			\item $\inner A \dots$ das \begriff{Innere} von $A$
			\item $\abschluss{A} \dots$ die \begriff{abgeschlossene Hülle} von $A$
			\item $\rand A \defeq \abschluss{A} \cap \abschluss{A^C} =  \abschluss{A} \setminus \inner A \dots$ der \begriff{Rand} von $A$
			\item $A^\complement \defeq \S \setminus A \dots$ das \begriff{Komplement} von $A$
		\end{itemize}
		\item Die durch $d$ induzierte \begriff{Topologie} ist
		\begin{align*}
			\mathcal{G} \defeq \mathcal{G}(\S) &\defeq \menge{G \subseteq \S: G \text{ ist offen bzgl. } d} \\
			&\phantom{:}= \menge{G \subseteq \S: \forall x \in G: \exists \, r > 0 : B_d(x,r) \subseteq G}
			%
			\intertext{und}
			%
			\mathcal{F} \defeq \mathcal{F}(\S) 
			&\defeq \menge{F \subseteq \S : F \text{ ist abgeschlossen}} 
			= \menge{F : F = G^\complement, G \in \mathcal{G}}
		\end{align*}
		ist die \begriff{Menge der abgeschlossenen Mengen}.
		\item Sei $\emptyset \neq A \subseteq \S$ und $x \in \S$. Dann ist $d(x,A) \defeq \inf\menge{d(x,a) : a \in A} \ge 0$ der \begriff{Abstand von $x$ zu $A$}.
		\item Es ist
		\begin{align*}
			C(\S) &\defeq \menge{f \colon S \to \R : f \text{ stetig}} \\
			C^b(\S) &\defeq \menge{f \in C(\S) : f \text{ beschränkt}} \\
			\norm{f} &\defeq \norm{ f}_\infty \defeq \sup_{x\in\S} \abs{f(x)}
		\end{align*}
	\end{enumerate}
\end{definition}

\begin{lemma} \label{lemma: 2.3} Sei $A \neq \emptyset$.
	\begin{enumerate}[label={(\arabic*)}]
		\item \label{it: distCharakterisierung} Es ist $x \in\abschluss{A} \equivalent d(x,A)=0$.
		\item \label{it: distDreieck} Für alle $x,y \in \S$ gilt $\abs{d(x,A)-d(y,A)} \le d(x,y)$.
		\item \label{it: distStetig} Die Abbildung $d(*, A) \colon \S \to [0,\infty)$, $x \mapsto d(x,A)$ ist gleichmäßig stetig.
	\end{enumerate}
\end{lemma}

\begin{proof}
	\begin{enumerate}[label=(zu \arabic*), leftmargin=*]
		\item \begin{equivalence}
			\hinrichtung Sei $x \in \abschluss{A}$. Dann existiert für alle $\epsilon > 0$ ein $a \in A$ mit $d(x,a) < \epsilon$. Somit 
			\begin{equation*}
				0 \le d(x,A) \le d(x,a) < \epsilon \quad \forall \epsilon > 0 
				\overset{\epsilon \to 0}{\follows} d(x,A) = 0
			\end{equation*}
			\rueckrichtung Sei $d(x,A)=0$. Dann folgt aus der Infimumseigenschaft:
			\begin{equation*}
				\forall \epsilon > 0 \enskip \exists \, a \in A : \quad 0 \le d(x,a) \le 0 + \epsilon = \epsilon \\
				\follows x \in \abschluss{A}
			\end{equation*}
		\end{equivalence}
		\item Seien $x,y \in \S$. Dann gilt $d(x,a) \le d(x,y) + d(y,a)$ mit der Dreiecksungleichung für alle $a \in A$ und somit
		\begin{equation*}
			d(x,A) \le d(x,y) + d(y,A) 
			\follows d(x,A) - d(y,A) \le d(x,y)
		\end{equation*}
		Vertauschen von $x$ und $y$ liefert $d(y,A) - d(x,A) \le d(y,x) = d(x,y)$ und daraus folgt die Behauptung.
		\item Folgt aus \cref{it: distDreieck} da die Funktion $d(\cdot,A)$ Lipschitz-stetig und damit gleichmäßig stetig ist.
	\end{enumerate}
\end{proof}

\begin{satz}[Stetige Approximation von Indikatorfunktionen] \label{satz: 2.4}
	Zu $A \subseteq \S$ und $\epsilon > 0$ existiert eine gleichmäßig stetige Funktion
	\begin{equation*}
		f \colon \S \to [0,1] \quad \text{ mit der Eigenschaft} \quad f(x) =
		\begin{cases}
			1 & \falls x \in A \\
			0 & \falls d(x,A) \ge \epsilon
		\end{cases}
	\end{equation*}
\end{satz}
\begin{proof}
	Setze
	\begin{equation*}
		\phi \colon \R \to [0,1] 
		\quad \mit \quad 
		\phi(t) \defeq
		\begin{cases}
			1    & \falls t \le 0 \\
			1-t  & \falls 0 < t < 1 \\
			0    & \falls t \ge 1
		\end{cases}
	\end{equation*}
	Dann ist $\phi$ gleichmäßig stetig auf $\R$. Sei
	\begin{equation*}
		f(x) \defeq \phi \brackets{\frac{1}{\epsilon} *  d(x,A)} \qquad \forall x \in \S
	\end{equation*}
	Dann hat dieses $f$ die gewünschte Eigenschaft wegen \cref{lemma: 2.3}.
\end{proof}
%
\begin{definition}\label{def: 2.5}
	Ein metrischer Raum $(\S,d)$ heißt \begriff{separabel}
	\begin{align*}
		&:\Leftrightarrow \exists \text{ abzählbares } S_0 \subseteq \S :\S \subseteq \abschluss{S_0} \\
		&\phantom{:} \Leftrightarrow \exists \text{ abzählbares } S_0 \subseteq \S : \S = \abschluss{S_0} \\
		&\phantom{:}\Leftrightarrow \exists \text{ abzählbares } S_0 \subseteq \S : S_0 \text{ liegt dicht in } \S
	\end{align*}
\end{definition}

\begin{beispiel} \label{bsp: 2.6}
	Der metrische Raum $C([0,1])$ mit Supremums-Metrik ist separabel.
\end{beispiel}
\begin{proof}
	Die Menge $S_0 \defeq \menge{P : P \text{ ist Polynom mit rationalen Koeffizienten}}$ ist abzählbar. Aus dem \textit{Approximationssatz von Weierstraß} und der Dichtheit von $\Q$ folgt die Behauptung.
\end{proof}

\begin{definition} %2.7
	$\G_0 \subseteq \G$ heißt \begriff{Basis} von $\G$ genau dann, wenn sich jedes $G \in \G$ als Vereinigung von Mengen aus $\G_0$,
	so genannte \begriff{$\G_0$-Mengen}, darstellen lässt.
\end{definition}
%
\begin{beispiel}\label{bsp: 2.8} %2.8
	Die Menge $\menge{ B(x,r) : x \in \S, 0 < r \in \Q}$ ist Basis von $\G$.
\end{beispiel}
\begin{proof}
	Sei $G \in \G$. Dann gilt:
	\begin{equation*}
		\forall x \in G \enskip \exists \,  0 < r_x \in \Q : B(x,r_x) \subseteq G 
		\follows
		G = \bigcup_{x\in G} \menge{x}
		\subseteq \bigcup_{x \in G} \underbrace{B(x,r_x)}_{\subseteq G} \subseteq G 
%		\follows 
%		G = \bigcup_{x\in G} \underbrace{B(x,r_x)}_{\in \G_0}
	\end{equation*}
	Damit gilt also Gleichheit.
\end{proof}


\begin{satz} \label{satz: 2.9}
	$\S$ separabel $\Leftrightarrow$ $\G$ hat eine abzählbare Basis
\end{satz}

\begin{proof}
	\begin{equivalence}
		\hinrichtung Sei $S_0\subseteq \S$ abzählbar und dicht in $\S$. Wir zeigen, dass
		\begin{equation*}
			\G_0 \defeq \menge{ B(x,r) : x \in S_0 , 0 < r \in \Q}\subseteq \G
		\end{equation*}
		eine Basis ist.
		Sei also $G$ offen. Dann folgt aus Beispiel \cref{bsp: 2.8}, dass
		\begin{equation} \label{proof: 2.9-sternchen} \tag{$\star$}
			G = \bigcup_{x \in G} B(x,r_x), \qquad 0 < r_x \in\Q \enskip \forall x \in G
		\end{equation}
		Da $\quer{S_0} = \S$ gilt:
		\begin{align*}
			&\forall x \in G:\exists y_x \in S_0: d(x,y_x) < \frac{r_x}{2} \\
			&\Rightarrow \quad d(x,y)
			\overset{\triangle-\text{Ungl.}}{\le}
			d(x,y_x) + d(y_x,x) < \frac{r_x}{2} + \frac{r_x}{2} = r_x
			\qquad \forall y \in B\brackets{y_x,\frac{r_x}{2}} \\
			&\Rightarrow \quad B\brackets{y_x,\frac{r_x}{2}} \subseteq B(x,r_x)
			\qquad \forall x \in G \\
			&\Rightarrow \quad G \overset{\eqref{proof: 2.9-sternchen}}{\supseteq}
			\bigcup_{x \in G} \underbrace{ B\brackets{y_x,\frac{r_x}{2}} }_{\supseteq \menge{x}}
			\supseteq \bigcup_{x \in G} \menge{x} = G % \\
%			&\Rightarrow \quad G = \bigcup_{x\in G} \underbrace{B\brackets{y_x,\frac{r_x}{2}}}_{\in \G_0}
		\end{align*}
		Damit gilt also Gleichheit und $\G_0$ ist eine Basis. Da $S_0$ abzählbar ist, ist $\G_0$ abzählbar. 
		%
		\rueckrichtung Sei $\G_0$ abzählbare Basis von $\G$ und sei oBdA $\emptyset \notin \G_0$. Wähle für jedes $G \in \G_0$ ein $x_G \in G$ fest aus. Setze
		$S_0 \defeq \menge{x_G : G \in \G_0}$. Dann ist $S_0$ auch abzählbar. Es verbleibt die Dichtheit zu zeigen.
		Sei $x \in \S$ und $\epsilon > 0$ beliebig. Da $B(x,\epsilon)$ offen und $\G_0$ eine Basis ist, gilt:
		\begin{equation*}
			\exists \, \G_{x,\epsilon}\subseteq\G_0\mit B(x,\epsilon)=\bigcup_{G\in\G_{x,\epsilon}} G
			\follows  G\subseteq B(x,\epsilon)\qquad\forall G\in\G_{x,\epsilon}
		\end{equation*}
		Wähle ein $G$ von diesen aus. Dann gilt:
		\begin{equation*}
			x_G \in G \subseteq B(x,\epsilon)
			\follows  x_G\in B(x,\epsilon)
			\follows  d(\underbrace{x_G}_{\in S_0},x) < \epsilon
		\end{equation*}
	\end{equivalence}	
\end{proof}

\begin{satz}\label{satz: 2.10} %2.10
	Seien $(\S,d)$ und $(\S',d')$ metrische Räume.
	\begin{enumerate}[label={(\arabic*)}]
		\item \label{it: produktmetrik} Auf $\S \times \S'$ sind Metriken definiert durch
			\begin{align*}
				d_1 \brackets{(x,x'),(y,y')}
				&\defeq \brackets{ \brackets{d(x,y)}^2 + \brackets{d'(x',y')}^2}^{\frac{1}{2}}
				&\forall(x,x'),(y,y') \in \S \times \S'\\
				d_2 \brackets{(x,x'),(y,y')}
				&\defeq \max\menge{ d(x,y),d'(x',y')}
				&\forall(x,x'),(y,y')\in \S \times \S' \\
				d_3\brackets{(x,x'),(y,y')}
				&\defeq d(x,y) + d'(x',y')
				&\forall(x,x'),(y,y') \in \S \times \S'
			\end{align*}
		\item \label{it: produkttopologie} Die Metriken $d_1$, $d_2$ und $d_3$ induzieren dieselbe Topologie $\mathcal{G}(\S \times \S')$ auf $\S\times\S'$,
			die sogenannte \begriff{Produkttopologie} von $\mathcal{G}(\S)$ und $\mathcal{G}(\S')$.
		\item \label{it: produktbasis}
			$\mathcal{G}(\S\times\S')=\menge{\bigcup_{
					\begin{subarray}{c} G\in\mathcal{O}\\ G'\in\mathcal{O}' \end{subarray}
				} G\times G': \mathcal{O}\subseteq\mathcal{G}(\S), \mathcal{O}'\subseteq\mathcal{G}(\S')
				}$, d.h.
			\begin{equation}
				\menge{ G \times G' : G \in \mathcal{G}(\S), G' \in \mathcal{G}(\S')}
				\tag{Menge offener Rechtecke}
			\end{equation}
			bildet eine Basis von $\mathcal{G}(S\times\S')$.
	\end{enumerate}
\end{satz}

\begin{proof}
	\begin{enumerate}[label=(zu \arabic*), leftmargin=*]
		\item \label{it: zu_ref_it_produktmetrik}
		Überprüfung der Eigenschaften einer Metrik (zur Übung).
%		
		\item \label{par:zu_ref_it_produkttopologie}
			Punktweise gelten die Beziehungen:
			\begin{equation*}
				d_2 \le d_1 \le\sqrt{2}* d_2,
				\qquad
				\frac{1}{\sqrt{2}} * d_3 \le d_1 \le d_3,
				\qquad
				d_2 \le d_3 \le 2 * d_2
			\end{equation*}
			Beachte beim Nachweis, dass die $d_i$'s als Metriken größer oder gleich Null sind. Aus obigen Beziehungen folgt u.\,a.:
			\begin{equation*}
				B_{d_2}\brackets{x,\frac{r}{\sqrt{2}}} \subseteq  B_{d_1}(x,r)
			\end{equation*}
			denn
			\begin{equation}
				r > \sqrt{2} * d_2(y,x) \ge d_1(y,x) \label{eq: metrikvergleich}
			\end{equation}
			Damit folgt aus $G$ $d_1$-offen schon, dass $G$ $d_2$-offen ist, denn für $x \in G$ existiert $r > 0$ mit $B_{d_1}(x, r) \subseteq G$ und \eqref{eq: metrikvergleich} liefert
			$B_{d_2}(x, \frac{r}{\sqrt{2}}) \subseteq G$. Damit ist $x$ auch ein ein $d_2$-innerer Punkt. Die anderen Relationen gelten analog.
			%
		\item \begin{description}
				\item[$\subseteq$:] 
				\label{it: zu_ref_it_produktbasis}
				Sei $G^\ast \in \mathcal{G}(\S \times \S')$ eine offene Menge in der Produkttopologie. Dann gilt
				\begin{equation*}
					\forall x^\ast = (x,x') \in G^\ast: \, \exists r = r_{x^\ast} > 0: \enskip 
					G^\ast = \bigcup_{x^\ast \in G^\ast} B\brackets{x^\ast,r_{x^\ast}}.
				\end{equation*}
				Wegen \cref{it: produkttopologie} sei oBdA. $\S^\ast \defeq \S \times \S'$ versehen mit der Metrik $d_2$. Dann gilt:
				\begin{align*}
					B_{d_2}\brackets{x^\ast,r_{x^\ast}}
					&= \menge{(y,y')\in \S \times \S': \max \menge{d(x,y),d'(x',y')} < r_{x^\ast}} \\
					&= \menge{(y,y') \in \S \times \S': d(x,y) < r_{x^\ast} \land d'(x',y') < r_{x^\ast}} \\ 
					&=  \underbrace{B_d\brackets{x,r_{x^\ast}}}_{\in \mathcal{G}(\S)} \times \underbrace{B_{d'}\brackets{x', r_{x^\ast}}}_{\in\mathcal{G}(\S')}
				\end{align*}
				\item[$\supseteq$:] Sei zunächst $G \times G'$ $G,G'$ offen und $x^\ast = (x,x') \in G \times G'$. Also ist $x \in G$ und $x' \in G'$ und somit
				\begin{equation*}
					\exists \, r,r' > 0: B_d(x,r) \subseteq G \land B_{d'}(x',r') \subseteq G'
				\end{equation*}
				Setze $r^\ast \defeq \min\menge{ r,r'}>0$. Damit folgt
				\begin{align*}
					B_{d_2}\brackets{x^\ast,r^\ast}
					&\subseteq B_d(x,r) \times B_{d'}\brackets{x',r'} \\
					&\subseteq G \times G' = G^\ast \\
					&\Rightarrow G\times G' \in \mathcal{G}(\S \times \S')\\
					&\Rightarrow  \bigcup_{\begin{subarray}{c} G \in \mathcal{O} \\ G' \in \mathcal{O}' \end{subarray}} G \times G'\subseteq \mathcal{G}(\S\times\S')
					\qquad \forall \mathcal{O} \subseteq \mathcal{G}(\S), \mathcal{O}' \subseteq \mathcal{G}(\S')
				\end{align*}
				da die Produkttopologie vereinigungsstabil ist.
			\end{description}
	\end{enumerate}		
\end{proof}

\begin{definition}
	Die Metriken $d_1, d_2$ und $d_3$ heißen \begriff{Produktmetriken}.
	Daher alternative Schreibweise $d \times d'$, also z.\,B.\ $d\times d' \defeq \max\menge{d, d'}$ usw.
\end{definition}

\begin{bemerkung} %2.11
	Analog zu obigen Definitonen lassen sich Produktmetriken für endlich viele metrische Räume $(\S_i, d_i)_{i \in \menge{1, \dots ,k}}$ definieren, z.\,B.
	\begin{equation*}
		d_1 \times \dots \times d_k \defeq \brackets{\sum_{i=1}^k d_i^2}^{\frac{1}{2}},
	\end{equation*}
	die wiederum dieselbe Produkttopologie induzieren.
	Die bisherigen Resultate gelten analog.
\end{bemerkung}


	\chapter{Zufallsvariablen in metrischen Räumen}
	\label{chapter_3_zufallsvariablenInMetrischenRaeumen}
	% This work is licensed under the Creative Commons
% Attribution-NonCommercial-ShareAlike 4.0 International License. To view a copy
% of this license, visit http://creativecommons.org/licenses/by-nc-sa/4.0/ or
% send a letter to Creative Commons, PO Box 1866, Mountain View, CA 94042, USA.

\section{Messbarkeit in metrischen Räumen}
\label{sec: 3.1}

\begin{definition}\label{definition: 3.1}
	Die durch den metrischen Raum $(\S,d)$ induzierte $\sigma$-Algebra
	\begin{align*}
		\B \defeq \B(\S) \defeq \B_d(\S) \defeq \sigma \brackets{(\G(\S)} \defeq \sigma(\G).
	\end{align*}
	heißt \begriff{Borel-$\sigma$-Algebra}.
	Elemente $B  \in  \B(\S)$ heißen \begriff{Borel-Mengen} in $\S$.
\end{definition}

Beachte: $\B(\S) = \B_d(\S)$ hängt im Allgemeinen von der Metrik $d$ ab.

\begin{lemma}\label{lemma: 3.2}
	Es gilt:
	\begin{enumerate}[label=(\arabic*)]
		\item \label{it: 3.2-OpenClosed} 
		$\B(\S) = \sigma\brackets{\mathcal{F}(\S)}$
		\item \label{it: 3.2-StetigMessbar} Ist $f \colon (\S,d) \to (\S',d)$ stetig, so ist $f$ auch $\B_d(\S)$-$\B_d(\S')$-messbar.
		\item \label{it: 3.2-BasisSigmaErzeuger}
			Sei $\G_0$ abzählbare Basis von $\G(\S)$. Dann gilt $\sigma(\G_0) = \B(\S)$.
	\end{enumerate}
\end{lemma}

\begin{proof}
	\begin{enumerate}[label=(zu \alph*), leftmargin=*]
		\item \begin{itemize}
			\item[$\subseteq$:] Sei $G \in \G(\S)$. Dann gilt ist $G^\complement \in \mathcal{F}(\S) \subseteq \sigma\brackets{\mathcal{F}(\S)}$. Da $\sigma\brackets{\mathcal{F}(\S)}$ stabil unter Bildung von Komplementen ist, ist auch $G = \brackets{G^\complement}^\complement \in \sigma(\mathcal{F}(\S))$. Weiter folgt aus $\G \subseteq \sigma(\mathcal{F})$ schon $\sigma(\G) \subseteq \sigma(\mathcal{F})$.
			\item[$\supseteq$:] analog
		\end{itemize}
		\item Per Definition gilt $f^{-1}\brackets{\B_{d'}(\S')}
		= f^{-1}\brackets{\sigma\brackets{\G(\S')}}$. 
		Ein Satz aus der Maßtheorie liefert uns weiter $f^{-1}\brackets{\sigma\brackets{\G(\S')}} = \sigma\brackets{f^{-1}\brackets{\G(\S')}}$. 		
		Bekannterweise sind für stetige Funktionen Urbilder offener Mengen wieder offen, d.h. es gilt $f^{-1}\brackets{\G(\S')} \subseteq \G(S)$. 
		Somit ist $\sigma\brackets{f^{-1}\brackets{\G(\S')}} \subseteq \sigma\brackets{\G(\S)} = \borel{\S}$.
		\item \begin{itemize}
			\item[$\subseteq$:] klar wegen $\G_0 \subseteq \G(\S)$ und $\sigma$ monoton
			\item[$\supseteq$:] Sei $G \in \G$. Dann existieren geeignete $G_i \in \G_0 \subseteq \sigma(\G_0)$, sodass $G = \bigcup_{i \in \N} G_i$. Damit ist $G \in \sigma(\G_0)$.
			Aus der Stabilität unter Vereinigungen folgt die Behauptung:
			\begin{align*}
				G=\bigcup_{i \in \N} G_i \subseteq \sigma\brackets{\G_0}=\B(\S) \in \sigma(\G_0),
			\end{align*}
			da $\sigma$-Algebren \textit{abzählbar} vereinigungsstabil sind.
		\end{itemize}
	\end{enumerate} 
\end{proof}

\begin{satz}\label{satz: 3.3}
	Sei $(\S,d)$ separabler metrischer Raum. Dann gilt:
	\begin{align*}
		\B_{d \times  d}(\S \times \S) = \B(S) \otimes \B(\S)
	\end{align*}
	Ohne Separabilität gilt nur \enquote{$\supseteq$}.
	Für $S_1$ und $S_2$ separabel gilt $\B_{d_1  \times  d_2}(\S_1 \times  \S_2) = \B(S_1) \otimes \B(\S_2)$ analog, auch erweiterbar auf endliche Produkte.
\end{satz}

\begin{proof}
	Seien
	\begin{align*}
		\pi_1 &\colon \S \times \S \to \S \mit \pi_1(x,y) \defeq x \qquad \forall(x,y) \in \S \times \S \\
		\pi_2 &\colon \S \times \S \to \S \mit \pi_2(x,y) \defeq y \qquad \forall(x,y) \in \S \times \S 
	\end{align*}
	die \begriff{Projektionsabbildungen}. Dann gilt
	\begin{align*}
		\B(\S) \otimes \B(\S)
		&= \sigma(\pi_1,\pi_2) 
		= \sigma \brackets{\pi_1^{-1} \brackets{\sigma(\G)} \cup \pi_2^{-1} \brackets{\sigma(\G)}} \tag{Definition} \\
		&= \sigma\brackets{\sigma\brackets{\pi_1^{-1}(\G)} \cup \sigma\brackets{\pi_2^{-1}(\G)}} \\
		\overset{(\star)}&{=}
		\sigma\brackets{\pi_1^{-1}(\G) \cup \pi_2^{-1}(\G)} \\
		&= \sigma\brackets{\menge{ G \times  S,S \times  G' : G,G' \in \G}} \\
		&= \sigma\brackets{\menge{ \overbrace{G \times  G'}^{=(G \times  S) \cap (S \times  G')} : G,G' \in \G}} 
		&& \tag{\enquote{$\subseteq$}, da $S  \in  \G$ ; \enquote{$\supseteq$}, da $\sigma$-Algebra $\cap$-stabil} \\
		\overset{(\star\star)}&=
		\sigma\brackets{\menge{
		\bigcup_{
			\begin{subarray}{c}
				G \in \mathcal{O}\\
				G' \in \mathcal{O}'
			\end{subarray}
		}
		G \times  G' : \mathcal{O} , \mathcal{O}' \subseteq \G
		}} \\
		&=
		\sigma\brackets{\G(\S \times \S)} \tag{\cref{satz: 2.10}, \cref{it: produktbasis}} \\
		\overset{\text{Def}}&{=}
		\B(\S \times \S)
	\end{align*}
	
	Zum Nachweis von ($\star$):
	\begin{itemize}
		\item[$\supseteq$:] Setze $\mathcal{E} \defeq
		\underbrace{\sigma\brackets{\pi_1^{-1}(\G)}}_{\supseteq \pi_1^{-1}(\G)}
		\cup \underbrace{\sigma\brackets{\pi_2^{-1}(\G)}}_{\supseteq \pi_2^{-1}(\G)}
		\supseteq \pi_1^{-1}(\G) \cup \pi_2^{-1}(\G) \defqe \mathcal{H}$. Dann ist $\sigma(\mathcal{E}) \supseteq \sigma(\mathcal{H})$.
		\item[$\subseteq$:] Es ist $\pi_1^{-1}(\G) \subseteq \brackets{\pi_1^{-1}(\G) \cup \pi_2^{-1}(\G)} = \mathcal{H}$. Also ist auch $\sigma\brackets{\pi_1^{-1}(\G)} \subseteq \sigma(\mathcal{H})$. Analog erhalten wir auch $\sigma\brackets{\pi_2^{-1}(\G)} \subseteq \sigma(\mathcal{H})$. Dann gilt
		\begin{equation*}
			\mathcal{E} = 		\underbrace{\sigma\brackets{\pi_1^{-1}(\G)}}_{ \subseteq \sigma(\mathcal{H})}
			 \cup \underbrace{\sigma\brackets{\pi_2^{-1}(\G)}}_{ \subseteq \sigma(\mathcal{H})}
			 \subseteq \sigma(\mathcal{H}) \enskip\text{,}
		\end{equation*}
		also auch $\sigma(\mathcal{E}) \subseteq \sigma(\mathcal{H})$.
	\end{itemize}

	Bleibt Nachweis von ($\star\star$):
	
	\begin{itemize}
		\item[$\subseteq$:] ist klar mit $\mathcal{O}$ und $\mathcal{O}'$ einelementig (gilt auch ohne Separabilität)
		\item[$\supseteq$:] Gemäß \cref{satz: 2.9} existiert abzählbare Basis $\G_0$ von $\G$. Seien $\mathcal{O}, \mathcal{O}* \subseteq \G$. Sei
		\begin{align*}
			G^\ast &=\bigcup_{\begin{subarray}{c}
					G \in \mathcal{O}\\
					G' \in \mathcal{O}'
			\end{subarray}}G \times  G'
			=
			\bigcup_{\begin{subarray}{c}
					G,G'\text{ offen}\\
					G \times G' \subseteq G^\ast
			\end{subarray}}
			G \times  G' 
			\overset{(!)}{=}
			\bigcup_{\begin{subarray}{c}
					G_0,G_0' \in \G_0\\
					G \times  G_0' \subseteq  G^\ast
			\end{subarray}}
			G_0 \times  G_0'
		\end{align*}
		eine abzählbare Vereinigung, da $\G_0$ Basis ist, also abzählbar. Somit gilt dann $G^\ast \in \sigma (\{G \times  G' : G,G' \in \G \})$.
	\end{itemize}
\end{proof}

\begin{definition} \label{definition: 3.4}
	Sei $(\Omega,\A)$ ein Messraum.
	Eine Abbildung $X \colon \Omega \to \S$, die $\A$-$\B(\S)$-messbar ist, heißt \begriff{Zufallsvariable} (ZV) in dem metrischen Raum $(\S,d)$ über $(\Omega,\A)$.

	Sei $\P$ ein Wahrscheinlichkeitsmaß auf $(\Omega,\A)$, also $(\Omega,\A,\P)$ ein Wahrscheinlichkeitsraum.
	Das Bildmaß
	\begin{align*}
		\P \circ  X^{-1} &\defeq \P_X \defeq \L(X) \defeq \L(X\,|\;\P) \\
		\brackets{\P \circ X^{-1}}(B)
		&\defeq \P\brackets{X^{-1}(B)}
		=\P\brackets{\menge{\omega \in \Omega : X(\omega) \in  B}}
		\defqe \P(X \in  B) \qquad\forall B \in \B(\S)
	\end{align*}
	heißt \begriff{Verteilung} von $X$ unter $\P$.
\end{definition}

\begin{satz} \label{satz: 3.5}
	Sei $(\S,d)$ ein separabler metrischer Raum und seien $X,Y$ Zufallsvariablen in $(\S,d)$ über $(\Omega,\A)$.
	Dann ist $d(X,Y)$ eine reelle Zufallsvariable.
\end{satz}

\begin{proof}[gute Prüfungsfrage]
	Die Abbildungen $X,Y \colon (\Omega,\A) \to (\S,\B(\S))$ sind messbar genau dann, wenn die Abbildung $(X,Y) \colon (\Omega,\A) \to \big( \S \times \S,\underbrace{\B(\S) \otimes \B(\S)}_{= \B(\S \times \S)}\big)$ messbar ist.
	Jede Metrik ist bekanntlich stetig, also auch $d \colon \brackets{\S \times \S , \G(\S \times \S)} \to \R$\footnote{$\R$ sei mit der natürlichen Topologie versehen}.
	Dann folgt aus \cref{lemma: 3.2}, dass $d \colon \B(\S \times \S) \to \B(\R)$	messbar ist. Damit folgt die Behauptung, denn $d(X,Y) = d \circ (X,Y)$ ist messbar als Komposition von messbaren Abbildungen.
\end{proof}


	% This work is licensed under the Creative Commons
% Attribution-NonCommercial-ShareAlike 4.0 International License. To view a copy
% of this license, visit http://creativecommons.org/licenses/by-nc-sa/4.0/ or
% send a letter to Creative Commons, PO Box 1866, Mountain View, CA 94042, USA.

\section{Fast sichere Konvergenz}
\begin{definition} \label{definition: 3.6}
	Seien $X, X_n$ ($n \in \N$) Zufallsvariablen in einem separablen, metrischen Raum $(\S,d)$ über $(\Omega,\A,\P)$.
	\begin{equation*}
		X_n \ntoinf X \quad \P\text{-fast sicher}
		\defequiv
		\P\bigg( \underbrace{\menge{\omega \in \Omega : d\brackets{X_n(\omega),X(\omega)} \ntoinf 0}}_{\defqe M} \bigg) = 1
	\end{equation*}
\end{definition}

Beachte: Die Definition von $M$ mengentheoretisch aufgeschrieben (Schnitt $\leadsto$ \enquote{für alle}; Vereinigung $\leadsto$ \enquote{Es gibt}):
\begin{equation*}
	M =
	\bigcap_{0 < \epsilon \in \Q}
	\bigcup_{m \in \N}
	\bigcap_{n \ge m}
	\Big\{ \underbrace{d(X_n,X)}_{\defqe \zeta_n} < \epsilon \Big\} \overset{\text{\cref{satz: 3.5}}}{\in} \A
	\text{, denn }\zeta_n^{-1} \brackets{(-\infty,\epsilon)} \in  \A
\end{equation*}
Hierbei gilt $M \in \A$ aufgrund der Abzählbarkeit der beteiligten Schnitte und Vereinigungen. 
Man schreibt auch:
\begin{align*}
	\P\brackets{ \lim_{n \to \infty} X_n = X} = 1 \quad \text{ oder } \quad d(X_n, X) \to  0 \enskip (n \to \infty) \enskip \text{ $\P$-f.\,s.}
\end{align*}

Die bekannten Regeln (Ergebnisse) für \textit{reelle} Zufallsvariablen lassen sich mühelos verallgemeinern; so zum Beispiel der folgende Satz.
\begin{satz} \label{satz: 3.7}
	Der fast-sichere Grenzwert ist $\P$-fast sicher eindeutig:
	\begin{equation*}
		X_n \ntoinf X \quad \P\text{-fast sicher } \und 
		X_n \ntoinf X' \quad\P\text{-fast sicher }
		\follows 
		X = X' \quad \P\text{-fast sicher}
	\end{equation*}
\end{satz}
\begin{proof} 
	Es ist $\menge{ X\neq X'} \subseteq \menge{X_n \not\to  X} \cup \menge{X_n \not\to  X'}$ und $\P(X_n \not\to X) + \P(X_n \not\to X') = 0+0$. Also ist auch $\P(X \neq X') = 0$ und mit dem Gegenereignis gilt $\P\brackets{X = X'} = 1-\P\brackets{X\neq X'} = 1$.
\end{proof}

\begin{satz} \label{satz: 3.8}
	Seien $X, X_n$ ($n \in \N$) Zufallsvariablen im separablen, metrischen Raum $(\S,d)$ und sei $f \colon (\S,d) \to (\S',d')$ $\B_d(S)$ - $\B_{d'}(S')$-messbar und stetig in $X$ $\P$-fast sicher.
	\footnote{\enquote{stetig in $X$ $\P$-fast sicher} bedeutet,
		dass $\P(\menge{\omega  \in  \Omega : f \text{ in } X(\omega) \text{ stetig}}) = 1$. Insbesondere ist $\menge{\omega  \in  \Omega : f \text{ in } X(\omega) \text{ stetig}}$ messbar, was keine Selbstverständlichkeit ist.}
	Dann gilt:
	\begin{equation*}
		X_n \ntoinf	X \quad \P\text{-fast sicher }
		\follows f(X_n) \ntoinf f(X) \quad \P\text{-fast sicher}
	\end{equation*}
\end{satz}

\begin{proof}
	Aufgrund der Folgenstetigkeit gilt
	\begin{equation*}
		\menge{X_n \ntoinf  X} \cap \menge{f \text{ stetig in }X}
		\subseteq
		\menge{f(X_n) \ntoinf f(X)}
	\end{equation*}
	Da abzählbare Schnitte von Einsmengen stets Einsmengen sind, d.\,h.
	\begin{equation*}
		\forall i \in \N : \, \P(E_i)=1
		\follows \P\brackets{\bigcap_{i \in \N} E_i}=1
		\qquad \forall \menge{E_i} \subseteq \Omega \mit \P(E_i)=1
	\end{equation*}
	folgt 
	\begin{equation*}
		1 = \P\brackets{\menge{X_n \ntoinf X \text{ und $f$ stetig in } X}} \le\P\brackets{\menge{f(X_n) \ntoinf  f(X)}} \le 1
	\end{equation*}
	und somit schon $\P\brackets{\menge{f(X_n) \ntoinf  f(X)}} = 1$.
\end{proof}

\begin{satz}[Konvergenz-Kriterium]\label{satz: 3.9}
	\begin{equation*}
		X_n \ntoinf X \quad \P\text{-fast sicher}
		\equivalent
		\forall\epsilon > 0: \lim_{n \to \infty} \P\brackets{\sup_{m\ge n} d(X_m,X) > \epsilon} = 0
	\end{equation*}
\end{satz}
\begin{proof}
	Man ersetze im Beweis für den Fall reeller Zufallsvariablen $\abs{X_n-X}$ durch $d(X_n,X)$.
	Und beachte, dass alle Schlussfolgerungen bestehen bleiben. Hier ausführlich:
	\begin{align*}
		X_n \overset{n\to \infty}{\longrightarrow} X~\P \text{-fast sicher}
		&\equivalent 1 = \P\brackets{\bigcup_{0 < \epsilon} \bigcap_{m  \in  \N} \bigcup_{n \ge m} d(X_n, X) < \epsilon} \\
		&\equivalent 0 = \P\brackets{\bigcap_{0 < \epsilon} \bigcup_{m  \in  \N} \bigcap_{n \ge m} d(X_n, X) \ge \epsilon} \\
		&\equivalent 0 = \P\brackets{\bigcup_{m  \in  \N} \bigcap_{n \ge m} d(X_n, X) \ge \epsilon} \qquad \forall \epsilon > 0
	\end{align*}
	Da für $m \to  \infty$ die Mengen $\menge{\bigcap_{n \ge m} d(X_n, X) \ge \epsilon}$ kleiner werden,
	ist dies äquivalent zu
	\begin{align*}
		\forall \epsilon > 0 \enskip \forall \delta > 0 \enskip \exists\, m  \in \N: \quad \P\brackets{\bigcap_{n \ge m} d(X_n, X) \ge \epsilon} &< \delta \\
		\equivalent \forall \epsilon > 0: \quad \lim_{m \to  \infty} \P\brackets{\bigcap_{n \ge m} d(X_n, X) \ge \epsilon} &= 0 \\
		\equivalent \forall \epsilon > 0: \quad \lim_{m \to  \infty} \P\brackets{\sup_{n \ge m} d(X_n, X) \ge \epsilon} &= 0
	\end{align*}
\end{proof}

Ein sehr nützliches Kriterium ist Folgendes:
\begin{satz}\label{satz: 3.10}
	\begin{align*}
		\forall \epsilon > 0: \quad 
		\sum_{n \in \N_{\ge1}}\P\brackets{d(X_n,X)>\epsilon} < \infty
		\quad \implies \quad  X_n \ntoinf X \quad \P\text{-fast sicher}
	\end{align*}
\end{satz}
\begin{*bemerkung}
	Die $\P(d(X_n, X) > \epsilon)$ werden in der Statistik \begriff{Fehlerwahrscheinlichkeit} oder \begriff{tail probability} genannt.
	Um Fehlerwahrscheinlichkeiten abzuschätzen, also die Voraussetzung für diesen Satz für einen speziellen Fall zu zeigen, nutzt man häufig sogenannte Maximalungleichungen wie die Markov-Ungleichung und die Tschebychew-Ungleichung.
\end{*bemerkung}

\begin{proof}
	Setze $A_n(\epsilon) \defeq \menge{d(X_n,X)>\epsilon} \in \A$ wegen \cref{satz: 3.5}.
	Dann folgt aus dem \textit{ersten Borel-Cantelli-Lemma} $\displaystyle \P\brackets{\limsup_{n\to\infty} A_n(\epsilon)} = 0$ für alle $\epsilon > 0$. Mit
	\begin{align*}
		\liminf_{n \to \infty}\brackets{A_n(\epsilon)^\complement}
		\overset{\text{Def}}{=}
		\bigcup_{m \in \N}\bigcap_{n \ge  m} \brackets{A_n(\epsilon)}^\complement
		&= \brackets{\bigcap_{m \in \N} \bigcup_{n \ge m} \menge{ d(X_n,X) > \epsilon}}^\complement \\
		\overset{\text{Def}}&{=} \brackets{\limsup_{n \to \infty} A_n(\epsilon)}^\complement
	\end{align*}
	folgt dann
	\begin{equation*}
		1 = \P\brackets{\brackets{\limsup_{n\to\infty} A_n(\epsilon)}^\complement}
		= \P\brackets{\liminf_{n\to\infty} \brackets{A_n(\epsilon)^\complement}}
		\qquad \forall \epsilon > 0
	\end{equation*}
	Da abzählbare Durchschnitte von Eins-Mengen (also Mengen mit $\P$-Maß $1$) wieder Eins-Mengen sind, folgt schließlich:
	\begin{equation*}
		\P\brackets{\underbrace{
			\bigcap_{0 < \epsilon \in \Q}\bigcup_{m \in \N}\bigcap_{n\ge m}\menge{d(X_n,X)\le\epsilon}
		}_{\menge{ X_n\to X} = \menge{ d(X_n,X) \to 0}}}=1
	\end{equation*}
\end{proof}

Weitere Eigenschaften der fast sicheren Konvergenz von Zufallsvariablen in metrischen Räumen finden sich z.\,B.\ in \cite[Kapitel 8.2]{gaensslerstute1977Wahrscheinlichkeitstheorie}.
%

	
    \vfill \hfill $\square$
    
    \nocite{*} % erstellt Literaturverzeichnus ohne Verweise
   	\bibliographystyle{abbrvdin}
    \bibliography{literatur}
\end{document}
